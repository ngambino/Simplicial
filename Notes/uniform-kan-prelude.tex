

%% Math packages
\usepackage{amssymb,amsmath,amsthm}
\usepackage{mathtools}
\usepackage{stmaryrd}
\usepackage{enumerate}
\usepackage[all,cmtip,2cell]{xy} % cmtip gives arrow heads like in Shulman's papers
\usepackage{fouridx}
\usepackage{bbm}
\usepackage{scalerel}

%% General packages
\usepackage[utf8]{inputenc}
%\usepackage[usenames,dvipsnames]{xcolor}
\usepackage{geometry}
\usepackage{verbatim}
\usepackage{enumitem}
\usepackage{xparse}
\usepackage{xspace}
\usepackage[draft=false]{hyperref}
\usepackage{cleveref}
\usepackage{xcolor}

%%%%%%%%%%%%%%%%%%%%%%%%%%%%%%%%%%%%%%%%%%%%%%%%%%%%%%%%%%%%%%%%%%
%% Adapted from the HoTT book sources.

%% Theorem environments
\def\defthm#1#2#3#4{
  \newtheorem{#1}[theorem]{#3}
  \newtheorem*{#1*}{#3}
  \newtheorem{#2}[theorem]{#4}
  \newtheorem*{#2*}{#4}
  \crefname{#1}{#3}{#4}
  \crefname{#2}{#4}{#4}  
}

%%% Lambda abstractions.
% Each variable being abstracted over is a separate argument.  If
% there is more than one such argument, they *must* be enclosed in
% braces.  Arguments can be untyped, as in \lam{x}{y}, or typed with a
% colon, as in \lam{x:A}{y:B}. In the latter case, the colons are
% automatically noticed and (with current implementation) the space
% around the colon is reduced.  You can even give more than one variable
% the same type, as in \lam{x,y:A}.
\makeatletter
\def\lam#1{{\lambda}\@lamarg#1:\@endlamarg\@ifnextchar\bgroup{.\,\lam}{.\,}}
\def\@lamarg#1:#2\@endlamarg{\if\relax\detokenize{#2}\relax #1\else\@lamvar{\@lameatcolon#2},#1\@endlamvar\fi}
\def\@lamvar#1,#2\@endlamvar{(#2\,{:}\,#1)}
% \def\@lamvar#1,#2{{#2}^{#1}\@ifnextchar,{.\,{\lambda}\@lamvar{#1}}{\let\@endlamvar\relax}}
\def\@lameatcolon#1:{#1}
\let\lamt\lam
% This version silently eats any typing annotation.
\def\lamu#1{{\lambda}\@lamuarg#1:\@endlamuarg\@ifnextchar\bgroup{.\,\lamu}{.\,}}
\def\@lamuarg#1:#2\@endlamuarg{#1}
\makeatother

%%%%%%%%%%%%%%%%%%%%%%%%%%%%%%%%%%%%%%%%%%%%%%%%%%%%%%%%%%%%%%%%%%%%%%%%%%%%%%%%
%% Fancy pullback corners

\NewDocumentCommand\fancypullbackcore{m m}{
  \POS{
    % Setup coordinate system for left upper corner.
    "c00";"c01":"c10"::
    % Calculate axis offsets
    "c00"+/v(1,0){#1}/="a",
    "c00"+/v(0,1){#1}/="b",
    % Find tip candidates as intercepts with diagonal.
    "c00";"c11":
    "a";p+"c11"-"c01";x="sa",
    "b";p+"c11"-"c10";x="sb",
    % Calculate tip as mean of candidates.
    % FIXME: Should be geometric instead of arithmetic.
    % FIXME: A better selection criterion might be the area enclosed
    %        by the pullback symbol and the upper left corner.
    "sa"+"sb"-"c00"="t",
    % Setup coordinate system for left upper corner.
    "c00";"c01":"c10"::
    % Find intersections of pullback symbol with left and upper sides.
    "t";p+"c01"-"c11";x="x",
    "t";p+"c10"-"c11";y="y",
    % Setup coordinate system for pullback symbol.
    "t";"x":"y"::
    % Draw pullback symbol.
    (0,0);({#2},0)**\dir{-},
    (0,0);(0,{#2})**\dir{-}}}

% Pullback/pushout corners for arbitrary quadrangles.
% The pullback symbol will be drawn at the upper left corner, given by the current position.
% Use as \fancypullback{<x>}{<y>}{<z>}[<dist>][<size>] where:
%   <x>    position of upper right corner of the pullback square;
%   <y>    position of lower left corner of the pullback square;
%   <z>    position of lower right corner of the pullback square,
%          the quandrangle defaulting to a parallelogram if not specified.
%   <dist> measure of distance of the pullback symbol from the upper left corner,
%          defaulting to 0.2cm if not specified;
%   <size> relative size of the pullback symbol ranging from 0 to 1,
%          defaulting to 0.6 if not specified.
%
% Example:  \fancypullback{[dd]}{[dr]}{[ddr]}
\NewDocumentCommand\fancypullback{m m g o o}{
  \save{
    p="c00",
    {#1}="c01",
    {#2}="c10"}
  \IfValueTF{#3}
    {\POS{{#3}="c11"}}
    {\POS{"c01"+"c10"-"c00"="c11"}}
  \fancypullbackcore
    {\IfValueTF{#4}{#4}{0.2cm}}
    {\IfValueTF{#5}{#5}{0.6}}
  \restore}

% Version of the previous command where the corners are specified using
% the hops coordinate format from xymatrix excluding the square brackets.
%
% Example:  \fancypullbackhops{u}{ll}
\NewDocumentCommand\fancypullbackhops{m m g o o}{
  \IfValueTF{#3}
    {\pullbackcorewrapper{[#1]}{[#2]}{[#3]}[#4][#5]}
    {\pullbackcorewrapper{[#1]}{[#2]}[#4][#5]}}

% for adjunctions
% use as \ar@{}@<-0.24em>[..]|{\dir{_|_}}
\newdir{_|_}{^\dir{|-}}

%%%%%%%%%%%%%%%%%%%%%%%%%%%%%%%%%%%%%%%%%%%%%%%%%%%%%%%%%%%%%%%%%%%%%%%%%%%%%%%%
%% Nicola's stuff

\newdir{ >}{{}*!/-7pt/@{>}}
\newdir{m}{->}
%\newdir{m}{{}*!/-1pt/@{o}}
\newcommand{\xycenter}[1]{\begin{gathered}\xymatrix{#1}\end{gathered}}
%%% Pullback symbols
\newcommand{\pullback}[1]{\save*!/#1-1.2pc/#1:(-1,1)@^{|-}\restore}
\newcommand{\ulpullback}{\pullback{ul}}
\newcommand{\dlpullback}{\pullback{dl}}
\newcommand{\urpullback}{\pullback{ur}}
\newcommand{\drpullback}{\pullback{dr}}

\makeatletter
\newcommand{\DeclareAbbrevation}[2]{\newcommand{#1}{\@ifnextchar{.}{#2}{#2.\@\xspace}}}
\makeatother

\DeclareAbbrevation{\ie}{i.e}
\DeclareAbbrevation{\eg}{e.g}
\DeclareAbbrevation{\cf}{cf}
\DeclareAbbrevation{\etc}{etc}
\DeclareAbbrevation{\resp}{resp}
\DeclareAbbrevation{\etal}{et al}
\DeclareAbbrevation{\ibid}{ibid}


\newcommand{\defeq}{=_{\operatorname{def}}}
\newcommand{\co}{\colon}
\newcommand{\iso}{\cong} 
\newcommand{\rev}{\mathit{\vee}}
\newcommand{\op}{{\operatorname{op}}}
\newcommand{\catequiv}{\simeq} 
\newcommand{\cateq}{\simeq} 
\newcommand{\coend}{\int}


\newcommand{\cal}[1]{\mathcal{#1}}

\newcommand{\cat}[1]{\mathbb{#1}}
\newcommand{\catA}{\cat{A}}
\newcommand{\catB}{\cat{B}}
\newcommand{\catC}{\cat{C}}
\newcommand{\catD}{\cat{D}}
\newcommand{\catK}{\cat{K}}
\newcommand{\catM}{\cat{M}}

%% 

\newcommand{\SSet}{\mathbf{SSet}}
\newcommand{\CSet}{\mathbf{CSet}}
\newcommand{\UU}{\overline{\mathsf{W}}}
\newcommand{\U}{\mathsf{W}}
\newcommand{\Weq}{\operatorname{Eq}}
\newcommand{\Set}{\mathbf{Set}}
\newcommand{\SET}{\mathbf{SET}}
\newcommand{\N}{\mathbb{N}}

\newcommand{\Sd}{\operatorname{Sd}}
\newcommand{\Ex}{\operatorname{Ex}}

%%%%%%%%%%%%%%%%%%%%%%%%%%%%%%%%%%%%%%%%%%%%%%%%%%%%%%%%%%%%%%%%%%%%%%%%%%%%%%%%
%% My stuff

\UseAllTwocells

\DeclarePairedDelimiter\parens\lparen\rparen
\DeclarePairedDelimiter\floors\lfloor\rfloor
\DeclarePairedDelimiter\ceils\lceil\rceil
\DeclarePairedDelimiter\bracks\lbrack\rbrack
\DeclarePairedDelimiter\brackss\llbrack\rrbrack
\DeclarePairedDelimiter\verts\lvert\rvert
\DeclarePairedDelimiter\vertss\lVert\rVert
\DeclarePairedDelimiter\angles\langle\rangle
\DeclarePairedDelimiter\braces\lbrace\rbrace
\DeclarePairedDelimiterX\set[2]\lbrace\rbrace{#1 \mathrel{\delimsize\vert} #2}
\DeclarePairedDelimiter\ucorns\ulcorner\urcorner

\newcommand{\arghole}{-}
\newcommand{\brarghole}{(\arghole)}

\newcommand{\cc}{\mathbin{\circ}}
\newcommand{\id}{\operatorname{id}}
\newcommand{\const}{\operatorname{const}}
\newcommand{\canonical}{\mathord{!}}
\newcommand{\inl}{\operatorname{inl}}
\newcommand{\inr}{\operatorname{inr}}

\newcommand{\cod}{\operatorname{cod}}
\newcommand{\dom}{\operatorname{dom}}
\newcommand{\colim}{\operatorname{colim}}
\newcommand{\Id}{\operatorname{Id}}
\newcommand{\Lan}{\operatorname{Lan}}
\newcommand{\Ran}{\operatorname{Ran}}
\newcommand{\alg}{\operatorname{alg}}
\newcommand{\coalg}{\operatorname{coalg}}
\newcommand{\Alg}{\operatorname{Alg}}
\newcommand{\Coalg}{\operatorname{Coalg}}
\newcommand{\Cat}{\mathbf{Cat}}
\newcommand{\CAT}{\mathbf{CAT}}
\newcommand{\MonCAT}{\mathbf{MonCAT}}
\newcommand{\MonCat}{\mathbf{MonCat}}
\newcommand{\Presheaf}{\operatorname{Presheaf}}
\newcommand{\SSetCart}{\SSet_{\cart}^{\to}}
\newcommand{\cart}{\text{\normalfont{cart}}}
\newcommand{\PtdEndo}{\mathbf{PtdEndo}}
\newcommand{\CoptdEndo}{\mathbf{CoptdEndo}}
\newcommand{\Mnd}{\mathbf{Mnd}}
\newcommand{\Cmd}{\mathbf{Cmd}}
\newcommand{\Mon}{\mathbf{Mon}}
\newcommand{\Comon}{\mathbf{Comon}}
\newcommand{\Bialg}{\mathbf{Bialg}}
\newcommand{\Adju}{\mathbf{Adj}}
\newcommand{\Fun}{\mathbf{Fun}}
\newcommand{\AWFS}{\mathbf{AWFS}}
\newcommand{\LAWFS}{\mathbf{LAWFS}}
\newcommand{\RAWFS}{\mathbf{RAWFS}}

\newcommand{\join}{\mathbin{\star}}
\newcommand{\hatjoin}{\mathbin{\hat{\star}}}
\newcommand{\hattimes}{\mathbin{\hat{\times}}}
\newcommand{\hatcirc}{\mathbin{\widehat{\circ}}}
\newcommand{\hatexp}{\operatorname{\widehat{exp}}}
\newcommand{\hathom}{\operatorname{\widehat{hom}}}

\newcommand{\classcofib}{\mathcal{C}}
\newcommand{\classfib}{\mathcal{F}}
\newcommand{\classweq}{\mathcal{W}}
\newcommand{\classleft}{\mathcal{L}}
\newcommand{\classright}{\mathcal{R}}

\newcommand{\ptA}{\mathbf{a}}
\newcommand{\ptB}{\mathbf{b}}


\newcommand{\ret}{\mathbf{R}}
\newcommand{\retA}{\mathit{x}}
\newcommand{\retB}{\mathit{y}}

% Stolen from Mike Shulman.
\newcommand{\xto}{\xrightarrow}


\newcommand{\sql}{{}^L}
\newcommand{\sqr}{{}^R}
\newcommand{\squ}{{}^U}
\newcommand{\sqd}{{}^D}

\newcommand{\squl}{{}^{UL}}
\newcommand{\squr}{{}^{UR}}
\newcommand{\sqdl}{{}^{DL}}
\newcommand{\sqdr}{{}^{DR}}

\newcommand{\sqhori}{{}^{H}}
\newcommand{\sqvert}{{}^{V}}

\newcommand{\Slice}{\operatorname{Slice}}

\newcommand{\initial}{\bot}
\newcommand{\terminal}{\top}

\newcommand{\liftl}[1]{{\fourIdx{\pitchfork}{}{}{}{\smash{#1}\vphantom{I}}}}
\newcommand{\liftr}[1]{{\fourIdx{}{}{\pitchfork}{}{\smash{#1}\vphantom{I}}}}
\newcommand{\rliftl}[2]{{\fourIdx{\pitchfork}{}{}{#1}{#2}}}
\newcommand{\rliftr}[2]{{\fourIdx{}{}{\pitchfork}{#1}{#2}}}

\newsavebox{\mybox}
\newcommand{\scaledreflect}[1]{%
  \ThisStyle{\ifmmode%
    \savebox{\mybox}{$\SavedStyle#1$}%
    \reflectbox{\usebox{\mybox}}%
  \else%
    \savebox{\mybox}{#1}%
    \reflectbox{\usebox{\mybox}}%
  \fi%
}}

\newcommand{\obackslash}{\mathbin{\scaledreflect{\oslash}}}

\newcommand{\unit}{\top}
\newcommand{\hatunit}{\hat{\unit}}
\newcommand{\hatotimes}{\mathbin{\widehat{\otimes}}}
\newcommand{\hatobackslash}{\mathbin{\widehat{\obackslash}}}

\newcommand{\eval}{\operatorname{ev}}
\newcommand{\hateval}{\widehat{\eval}}

\newcommand{\cchat}{\mathbin{\widehat{\cc}}}

\newcommand{\interval}{I}
\newcommand{\intervall}{\ell}
\newcommand{\intervalr}{r}

\newcommand{\cyl}{C}
\newcommand{\ccyl}{\varepsilon}


\hyphenation{e-pi-mor-phism}
\hyphenation{e-pi-mor-phisms}
\hyphenation{mo-no-mor-phism}
\hyphenation{mo-no-mor-phisms}


\makeatletter
\def\ignorespacesandallpars{%
  \@ifnextchar\par
    {\expandafter\ignorespacesandallpars\@gobble}%
    {}%
}
\makeatother

%\newcommand{\para}[1]{\smallskip\noindent\textbf{{#1}.\ }\ignorespacesandallpars}

% usage: \note[author]{color}{content}
\newcommand{\note}[3][]{\def\auth{#1}\textcolor{#2}{{[\ifx\auth\empty\else\auth: \fi{#3}]}}}

\newcommand{\notesh}{\note{green}}
\newcommand{\noten}{\note{blue}}



\newcommand{\alert}[3][]{\def\auth{#1}\textcolor{#2}{{\ifx\auth\empty\else\auth: \fi{#3}}}}

\newcommand{\caution}{\alert{red}}

\newcommand{\Psh}{\mathrm{Psh}}

\newcommand{\yon}{\mathrm{y}}

% Reedy stuff

\newcommand{\R}{\mathrm{R}}
% \newcommand{\deg}{\operatorname{d}}
\newcommand{\Rp}{\R^+}
\newcommand{\Rm}{\R^-}
\newcommand{\Sk}{\operatorname{Sk}}
\newcommand{\Cosk}{\operatorname{Cosk}}
\newcommand{\obj}{\operatorname{Obj}}
\newcommand{\Rhat}{\operatorname{Psh}(\R)}

\newcommand{\fakeslice}{\sslash}

%\newcommand{\Join}{\operatorname*{\star}}
\newcommand{\Hatjoin}{\operatorname*{\scalerel*{\hat{\star}}{\sum}}}

\let\Otimes\bigotimes
\newcommand{\Hatotimes}{\operatorname*{\scalerel*{\hat{\otimes}}{\sum}}}

\newcommand{\isContr}{\operatorname{isContr}}
\newcommand{\Lift}{\operatorname{Lift}}

\newcommand{\triv}{\textup{triv}}

\newcommand{\leftapp}[2]{{#1} \mathbin{\otimes} {#2}}
\newcommand{\rightapp}[2]{{#1} \mathbin{\odot} {#2}}
\newcommand{\hatleftapp}[2]{{#1} \mathbin{\widehat{\otimes}} {#2}}
\newcommand{\hatrightapp}[2]{{#1} \mathbin{\widehat{\odot}} {#2}}

\newcommand{\Adj}{\operatorname{Adj}}

\newcommand{\Codensity}{\operatorname{Codensity}}

\newcommand{\fib}{\twoheadrightarrow}
\newcommand{\cof}{\rightarrowtail}
\newcommand{\trivfib}{\mathrel{\mathrlap{\hspace{0pt}\raisebox{5pt}{$\scriptscriptstyle\triv$}}\mathord{\twoheadrightarrow}}}
\newcommand{\trivcof}{\mathrel{\mathrlap{\hspace{0pt}\raisebox{5pt}{$\scriptscriptstyle\triv$}}\mathord{\rightarrowtail}}}
\newcommand{\we}{\mathrel{\mathrlap{\hspace{1pt}\raisebox{4pt}{$\scriptscriptstyle\sim$}}\mathord{\rightarrow}}}
\newcommand{\lwe}{\mathrel{\mathrlap{\hspace{4pt}\raisebox{4pt}{$\scriptscriptstyle\sim$}}\mathord{\leftarrow}}}

\newcommand{\cofib}{\mathbf{C}}
\newcommand{\trivcofib}{\mathbf{TC}}
\newcommand{\fibcat}{\mathbf{F}}
\newcommand{\trivfibcat}{\mathbf{TF}}
\newcommand{\weakequiv}{\mathbf{W}}

\setlist[enumerate]{label=(\roman*)}
