\documentclass[reqno,a4paper,oneside]{amsart}



%% Math packages
\usepackage{amssymb,amsmath,amsthm}
\usepackage{mathtools}
\usepackage{stmaryrd}
\usepackage{enumerate}
\usepackage[all,cmtip,2cell]{xy} % cmtip gives arrow heads like in Shulman's papers
\usepackage{fouridx}
\usepackage{bbm}
\usepackage{scalerel}

%% General packages
\usepackage[utf8]{inputenc}
%\usepackage[usenames,dvipsnames]{xcolor}
\usepackage{geometry}
\usepackage{verbatim}
\usepackage{enumitem}
\usepackage{xparse}
\usepackage{xspace}
\usepackage[draft=false]{hyperref}
\usepackage{cleveref}
\usepackage{xcolor}

%%%%%%%%%%%%%%%%%%%%%%%%%%%%%%%%%%%%%%%%%%%%%%%%%%%%%%%%%%%%%%%%%%
%% Adapted from the HoTT book sources.

%% Theorem environments
\def\defthm#1#2#3#4{
  \newtheorem{#1}[theorem]{#3}
  \newtheorem*{#1*}{#3}
  \newtheorem{#2}[theorem]{#4}
  \newtheorem*{#2*}{#4}
  \crefname{#1}{#3}{#4}
  \crefname{#2}{#4}{#4}  
}

%%% Lambda abstractions.
% Each variable being abstracted over is a separate argument.  If
% there is more than one such argument, they *must* be enclosed in
% braces.  Arguments can be untyped, as in \lam{x}{y}, or typed with a
% colon, as in \lam{x:A}{y:B}. In the latter case, the colons are
% automatically noticed and (with current implementation) the space
% around the colon is reduced.  You can even give more than one variable
% the same type, as in \lam{x,y:A}.
\makeatletter
\def\lam#1{{\lambda}\@lamarg#1:\@endlamarg\@ifnextchar\bgroup{.\,\lam}{.\,}}
\def\@lamarg#1:#2\@endlamarg{\if\relax\detokenize{#2}\relax #1\else\@lamvar{\@lameatcolon#2},#1\@endlamvar\fi}
\def\@lamvar#1,#2\@endlamvar{(#2\,{:}\,#1)}
% \def\@lamvar#1,#2{{#2}^{#1}\@ifnextchar,{.\,{\lambda}\@lamvar{#1}}{\let\@endlamvar\relax}}
\def\@lameatcolon#1:{#1}
\let\lamt\lam
% This version silently eats any typing annotation.
\def\lamu#1{{\lambda}\@lamuarg#1:\@endlamuarg\@ifnextchar\bgroup{.\,\lamu}{.\,}}
\def\@lamuarg#1:#2\@endlamuarg{#1}
\makeatother

%%%%%%%%%%%%%%%%%%%%%%%%%%%%%%%%%%%%%%%%%%%%%%%%%%%%%%%%%%%%%%%%%%%%%%%%%%%%%%%%
%% Fancy pullback corners

\NewDocumentCommand\fancypullbackcore{m m}{
  \POS{
    % Setup coordinate system for left upper corner.
    "c00";"c01":"c10"::
    % Calculate axis offsets
    "c00"+/v(1,0){#1}/="a",
    "c00"+/v(0,1){#1}/="b",
    % Find tip candidates as intercepts with diagonal.
    "c00";"c11":
    "a";p+"c11"-"c01";x="sa",
    "b";p+"c11"-"c10";x="sb",
    % Calculate tip as mean of candidates.
    % FIXME: Should be geometric instead of arithmetic.
    % FIXME: A better selection criterion might be the area enclosed
    %        by the pullback symbol and the upper left corner.
    "sa"+"sb"-"c00"="t",
    % Setup coordinate system for left upper corner.
    "c00";"c01":"c10"::
    % Find intersections of pullback symbol with left and upper sides.
    "t";p+"c01"-"c11";x="x",
    "t";p+"c10"-"c11";y="y",
    % Setup coordinate system for pullback symbol.
    "t";"x":"y"::
    % Draw pullback symbol.
    (0,0);({#2},0)**\dir{-},
    (0,0);(0,{#2})**\dir{-}}}

% Pullback/pushout corners for arbitrary quadrangles.
% The pullback symbol will be drawn at the upper left corner, given by the current position.
% Use as \fancypullback{<x>}{<y>}{<z>}[<dist>][<size>] where:
%   <x>    position of upper right corner of the pullback square;
%   <y>    position of lower left corner of the pullback square;
%   <z>    position of lower right corner of the pullback square,
%          the quandrangle defaulting to a parallelogram if not specified.
%   <dist> measure of distance of the pullback symbol from the upper left corner,
%          defaulting to 0.2cm if not specified;
%   <size> relative size of the pullback symbol ranging from 0 to 1,
%          defaulting to 0.6 if not specified.
%
% Example:  \fancypullback{[dd]}{[dr]}{[ddr]}
\NewDocumentCommand\fancypullback{m m g o o}{
  \save{
    p="c00",
    {#1}="c01",
    {#2}="c10"}
  \IfValueTF{#3}
    {\POS{{#3}="c11"}}
    {\POS{"c01"+"c10"-"c00"="c11"}}
  \fancypullbackcore
    {\IfValueTF{#4}{#4}{0.2cm}}
    {\IfValueTF{#5}{#5}{0.6}}
  \restore}

% Version of the previous command where the corners are specified using
% the hops coordinate format from xymatrix excluding the square brackets.
%
% Example:  \fancypullbackhops{u}{ll}
\NewDocumentCommand\fancypullbackhops{m m g o o}{
  \IfValueTF{#3}
    {\pullbackcorewrapper{[#1]}{[#2]}{[#3]}[#4][#5]}
    {\pullbackcorewrapper{[#1]}{[#2]}[#4][#5]}}

% for adjunctions
% use as \ar@{}@<-0.24em>[..]|{\dir{_|_}}
\newdir{_|_}{^\dir{|-}}

%%%%%%%%%%%%%%%%%%%%%%%%%%%%%%%%%%%%%%%%%%%%%%%%%%%%%%%%%%%%%%%%%%%%%%%%%%%%%%%%
%% Nicola's stuff

\newdir{ >}{{}*!/-7pt/@{>}}
\newdir{m}{->}
%\newdir{m}{{}*!/-1pt/@{o}}
\newcommand{\xycenter}[1]{\begin{gathered}\xymatrix{#1}\end{gathered}}
%%% Pullback symbols
\newcommand{\pullback}[1]{\save*!/#1-1.2pc/#1:(-1,1)@^{|-}\restore}
\newcommand{\ulpullback}{\pullback{ul}}
\newcommand{\dlpullback}{\pullback{dl}}
\newcommand{\urpullback}{\pullback{ur}}
\newcommand{\drpullback}{\pullback{dr}}

\makeatletter
\newcommand{\DeclareAbbrevation}[2]{\newcommand{#1}{\@ifnextchar{.}{#2}{#2.\@\xspace}}}
\makeatother

\DeclareAbbrevation{\ie}{i.e}
\DeclareAbbrevation{\eg}{e.g}
\DeclareAbbrevation{\cf}{cf}
\DeclareAbbrevation{\etc}{etc}
\DeclareAbbrevation{\resp}{resp}
\DeclareAbbrevation{\etal}{et al}
\DeclareAbbrevation{\ibid}{ibid}


\newcommand{\defeq}{=_{\operatorname{def}}}
\newcommand{\co}{\colon}
\newcommand{\iso}{\cong} 
\newcommand{\rev}{\mathit{\vee}}
\newcommand{\op}{{\operatorname{op}}}
\newcommand{\catequiv}{\simeq} 
\newcommand{\cateq}{\simeq} 
\newcommand{\coend}{\int}


\newcommand{\cal}[1]{\mathcal{#1}}

\newcommand{\cat}[1]{\mathbb{#1}}
\newcommand{\catA}{\cat{A}}
\newcommand{\catB}{\cat{B}}
\newcommand{\catC}{\cat{C}}
\newcommand{\catD}{\cat{D}}
\newcommand{\catK}{\cat{K}}
\newcommand{\catM}{\cat{M}}

%% 

\newcommand{\SSet}{\mathbf{SSet}}
\newcommand{\CSet}{\mathbf{CSet}}
\newcommand{\UU}{\overline{\mathsf{W}}}
\newcommand{\U}{\mathsf{W}}
\newcommand{\Weq}{\operatorname{Eq}}
\newcommand{\Set}{\mathbf{Set}}
\newcommand{\SET}{\mathbf{SET}}
\newcommand{\N}{\mathbb{N}}

\newcommand{\Sd}{\operatorname{Sd}}
\newcommand{\Ex}{\operatorname{Ex}}

%%%%%%%%%%%%%%%%%%%%%%%%%%%%%%%%%%%%%%%%%%%%%%%%%%%%%%%%%%%%%%%%%%%%%%%%%%%%%%%%
%% My stuff

\UseAllTwocells

\DeclarePairedDelimiter\parens\lparen\rparen
\DeclarePairedDelimiter\floors\lfloor\rfloor
\DeclarePairedDelimiter\ceils\lceil\rceil
\DeclarePairedDelimiter\bracks\lbrack\rbrack
\DeclarePairedDelimiter\brackss\llbrack\rrbrack
\DeclarePairedDelimiter\verts\lvert\rvert
\DeclarePairedDelimiter\vertss\lVert\rVert
\DeclarePairedDelimiter\angles\langle\rangle
\DeclarePairedDelimiter\braces\lbrace\rbrace
\DeclarePairedDelimiterX\set[2]\lbrace\rbrace{#1 \mathrel{\delimsize\vert} #2}
\DeclarePairedDelimiter\ucorns\ulcorner\urcorner

\newcommand{\arghole}{-}
\newcommand{\brarghole}{(\arghole)}

\newcommand{\cc}{\mathbin{\circ}}
\newcommand{\id}{\operatorname{id}}
\newcommand{\const}{\operatorname{const}}
\newcommand{\canonical}{\mathord{!}}
\newcommand{\inl}{\operatorname{inl}}
\newcommand{\inr}{\operatorname{inr}}

\newcommand{\cod}{\operatorname{cod}}
\newcommand{\dom}{\operatorname{dom}}
\newcommand{\colim}{\operatorname{colim}}
\newcommand{\Id}{\operatorname{Id}}
\newcommand{\Lan}{\operatorname{Lan}}
\newcommand{\Ran}{\operatorname{Ran}}
\newcommand{\alg}{\operatorname{alg}}
\newcommand{\coalg}{\operatorname{coalg}}
\newcommand{\Alg}{\operatorname{Alg}}
\newcommand{\Coalg}{\operatorname{Coalg}}
\newcommand{\Cat}{\mathbf{Cat}}
\newcommand{\CAT}{\mathbf{CAT}}
\newcommand{\MonCAT}{\mathbf{MonCAT}}
\newcommand{\MonCat}{\mathbf{MonCat}}
\newcommand{\Presheaf}{\operatorname{Presheaf}}
\newcommand{\SSetCart}{\SSet_{\cart}^{\to}}
\newcommand{\cart}{\text{\normalfont{cart}}}
\newcommand{\PtdEndo}{\mathbf{PtdEndo}}
\newcommand{\CoptdEndo}{\mathbf{CoptdEndo}}
\newcommand{\Mnd}{\mathbf{Mnd}}
\newcommand{\Cmd}{\mathbf{Cmd}}
\newcommand{\Mon}{\mathbf{Mon}}
\newcommand{\Comon}{\mathbf{Comon}}
\newcommand{\Bialg}{\mathbf{Bialg}}
\newcommand{\Adju}{\mathbf{Adj}}
\newcommand{\Fun}{\mathbf{Fun}}
\newcommand{\AWFS}{\mathbf{AWFS}}
\newcommand{\LAWFS}{\mathbf{LAWFS}}
\newcommand{\RAWFS}{\mathbf{RAWFS}}

\newcommand{\join}{\mathbin{\star}}
\newcommand{\hatjoin}{\mathbin{\hat{\star}}}
\newcommand{\hattimes}{\mathbin{\hat{\times}}}
\newcommand{\hatcirc}{\mathbin{\widehat{\circ}}}
\newcommand{\hatexp}{\operatorname{\widehat{exp}}}
\newcommand{\hathom}{\operatorname{\widehat{hom}}}

\newcommand{\classcofib}{\mathcal{C}}
\newcommand{\classfib}{\mathcal{F}}
\newcommand{\classweq}{\mathcal{W}}
\newcommand{\classleft}{\mathcal{L}}
\newcommand{\classright}{\mathcal{R}}

\newcommand{\ptA}{\mathbf{a}}
\newcommand{\ptB}{\mathbf{b}}


\newcommand{\ret}{\mathbf{R}}
\newcommand{\retA}{\mathit{x}}
\newcommand{\retB}{\mathit{y}}

% Stolen from Mike Shulman.
\newcommand{\xto}{\xrightarrow}


\newcommand{\sql}{{}^L}
\newcommand{\sqr}{{}^R}
\newcommand{\squ}{{}^U}
\newcommand{\sqd}{{}^D}

\newcommand{\squl}{{}^{UL}}
\newcommand{\squr}{{}^{UR}}
\newcommand{\sqdl}{{}^{DL}}
\newcommand{\sqdr}{{}^{DR}}

\newcommand{\sqhori}{{}^{H}}
\newcommand{\sqvert}{{}^{V}}

\newcommand{\Slice}{\operatorname{Slice}}

\newcommand{\initial}{\bot}
\newcommand{\terminal}{\top}

\newcommand{\liftl}[1]{{\fourIdx{\pitchfork}{}{}{}{\smash{#1}\vphantom{I}}}}
\newcommand{\liftr}[1]{{\fourIdx{}{}{\pitchfork}{}{\smash{#1}\vphantom{I}}}}
\newcommand{\rliftl}[2]{{\fourIdx{\pitchfork}{}{}{#1}{#2}}}
\newcommand{\rliftr}[2]{{\fourIdx{}{}{\pitchfork}{#1}{#2}}}

\newsavebox{\mybox}
\newcommand{\scaledreflect}[1]{%
  \ThisStyle{\ifmmode%
    \savebox{\mybox}{$\SavedStyle#1$}%
    \reflectbox{\usebox{\mybox}}%
  \else%
    \savebox{\mybox}{#1}%
    \reflectbox{\usebox{\mybox}}%
  \fi%
}}

\newcommand{\obackslash}{\mathbin{\scaledreflect{\oslash}}}

\newcommand{\unit}{\top}
\newcommand{\hatunit}{\hat{\unit}}
\newcommand{\hatotimes}{\mathbin{\widehat{\otimes}}}
\newcommand{\hatobackslash}{\mathbin{\widehat{\obackslash}}}

\newcommand{\eval}{\operatorname{ev}}
\newcommand{\hateval}{\widehat{\eval}}

\newcommand{\cchat}{\mathbin{\widehat{\cc}}}

\newcommand{\interval}{I}
\newcommand{\intervall}{\ell}
\newcommand{\intervalr}{r}

\newcommand{\cyl}{C}
\newcommand{\ccyl}{\varepsilon}


\hyphenation{e-pi-mor-phism}
\hyphenation{e-pi-mor-phisms}
\hyphenation{mo-no-mor-phism}
\hyphenation{mo-no-mor-phisms}


\makeatletter
\def\ignorespacesandallpars{%
  \@ifnextchar\par
    {\expandafter\ignorespacesandallpars\@gobble}%
    {}%
}
\makeatother

%\newcommand{\para}[1]{\smallskip\noindent\textbf{{#1}.\ }\ignorespacesandallpars}

% usage: \note[author]{color}{content}
\newcommand{\note}[3][]{\def\auth{#1}\textcolor{#2}{{[\ifx\auth\empty\else\auth: \fi{#3}]}}}

\newcommand{\notesh}{\note{green}}
\newcommand{\noten}{\note{blue}}



\newcommand{\alert}[3][]{\def\auth{#1}\textcolor{#2}{{\ifx\auth\empty\else\auth: \fi{#3}}}}

\newcommand{\caution}{\alert{red}}

\newcommand{\Psh}{\mathrm{Psh}}

\newcommand{\yon}{\mathrm{y}}

% Reedy stuff

\newcommand{\R}{\mathrm{R}}
% \newcommand{\deg}{\operatorname{d}}
\newcommand{\Rp}{\R^+}
\newcommand{\Rm}{\R^-}
\newcommand{\Sk}{\operatorname{Sk}}
\newcommand{\Cosk}{\operatorname{Cosk}}
\newcommand{\obj}{\operatorname{Obj}}
\newcommand{\Rhat}{\operatorname{Psh}(\R)}

\newcommand{\fakeslice}{\sslash}

%\newcommand{\Join}{\operatorname*{\star}}
\newcommand{\Hatjoin}{\operatorname*{\scalerel*{\hat{\star}}{\sum}}}

\let\Otimes\bigotimes
\newcommand{\Hatotimes}{\operatorname*{\scalerel*{\hat{\otimes}}{\sum}}}

\newcommand{\isContr}{\operatorname{isContr}}
\newcommand{\Lift}{\operatorname{Lift}}

\newcommand{\triv}{\textup{triv}}

\newcommand{\leftapp}[2]{{#1} \mathbin{\otimes} {#2}}
\newcommand{\rightapp}[2]{{#1} \mathbin{\odot} {#2}}
\newcommand{\hatleftapp}[2]{{#1} \mathbin{\widehat{\otimes}} {#2}}
\newcommand{\hatrightapp}[2]{{#1} \mathbin{\widehat{\odot}} {#2}}

\newcommand{\Adj}{\operatorname{Adj}}

\newcommand{\Codensity}{\operatorname{Codensity}}

\newcommand{\fib}{\twoheadrightarrow}
\newcommand{\cof}{\rightarrowtail}
\newcommand{\trivfib}{\mathrel{\mathrlap{\hspace{0pt}\raisebox{5pt}{$\scriptscriptstyle\triv$}}\mathord{\twoheadrightarrow}}}
\newcommand{\trivcof}{\mathrel{\mathrlap{\hspace{0pt}\raisebox{5pt}{$\scriptscriptstyle\triv$}}\mathord{\rightarrowtail}}}
\newcommand{\we}{\mathrel{\mathrlap{\hspace{1pt}\raisebox{4pt}{$\scriptscriptstyle\sim$}}\mathord{\rightarrow}}}
\newcommand{\lwe}{\mathrel{\mathrlap{\hspace{4pt}\raisebox{4pt}{$\scriptscriptstyle\sim$}}\mathord{\leftarrow}}}

\newcommand{\cofib}{\mathbf{C}}
\newcommand{\trivcofib}{\mathbf{TC}}
\newcommand{\fibcat}{\mathbf{F}}
\newcommand{\trivfibcat}{\mathbf{TF}}
\newcommand{\weakequiv}{\mathbf{W}}

\setlist[enumerate]{label=(\roman*)}



\newtheorem{theorem}{Theorem}[section]
\newtheorem*{theorem*}{Theorem}
\crefname{theorem}{Theorem}{Theorems}

\defthm{corollary}{corollaries}{Corollary}{Corollaries}
\defthm{lemma}{lemmata}{Lemma}{Lemmata}
\defthm{proposition}{propositions}{Proposition}{Propositions}
\defthm{exercise}{exercises}{Exercise}{Exercises}

\theoremstyle{definition}

\defthm{definition}{definitions}{Definition}{Definitions}

\defthm{remark}{remarks}{Remark}{Remarks}
\defthm{example}{examples}{Example}{Examples}
\defthm{question}{questions}{Question}{Questions}
\defthm{assumption}{assumptions}{Assumption}{Assumptions}

\crefname{section}{Section}{Sections}
\crefname{subsection}{Subsection}{Subsections}

\crefname{equation}{}{}
\numberwithin{equation}{section}

\usepackage[parfill]{parskip} 
\message{<Paul Taylor's Proof Trees, 2 August 1996>}
%% Build proof tree for Natural Deduction, Sequent Calculus, etc.
%% WITH SHORTENING OF PROOF RULES!
%% Paul Taylor, begun 10 Oct 1989
%% *** THIS IS ONLY A PRELIMINARY VERSION AND THINGS MAY CHANGE! ***
%%
%% 2 Aug 1996: fixed \mscount and \proofdotnumber
%%
%%      \prooftree
%%              hyp1            produces:
%%              hyp2
%%              hyp3            hyp1    hyp2    hyp3
%%      \justifies              -------------------- rulename
%%              concl                   concl
%%      \thickness=0.08em
%%      \shiftright 2em
%%      \using
%%              rulename
%%      \endprooftree
%%
%% where the hypotheses may be similar structures or just formulae.
%%
%% To get a vertical string of dots instead of the proof rule, do
%%
%%      \prooftree                      which produces:
%%              [hyp]
%%      \using                                  [hyp]
%%              name                              .
%%      \proofdotseparation=1.2ex                 .name
%%      \proofdotnumber=4                         .
%%      \leadsto                                  .
%%              concl                           concl
%%      \endprooftree
%%
%% Within a prooftree, \[ and \] may be used instead of \prooftree and
%% \endprooftree; this is not permitted at the outer level because it
%% conflicts with LaTeX. Also,
%%      \Justifies
%% produces a double line. In LaTeX you can use \begin{prooftree} and
%% \end{prootree} at the outer level (however this will not work for the inner
%% levels, but in any case why would you want to be so verbose?).
%%
%% All of of the keywords except \prooftree and \endprooftree are optional
%% and may appear in any order. They may also be combined in \newcommand's
%% eg "\def\Cut{\using\sf cut\thickness.08em\justifies}" with the abbreviation
%% "\prooftree hyp1 hyp2 \Cut \concl \endprooftree". This is recommended and
%% some standard abbreviations will be found at the end of this file.
%%
%% \thickness specifies the breadth of the rule in any units, although
%% font-relative units such as "ex" or "em" are preferable.
%% It may optionally be followed by "=".
%% \proofrulebreadth=.08em or \setlength\proofrulebreadth{.08em} may also be
%% used either in place of \thickness or globally; the default is 0.04em.
%% \proofdotseparation and \proofdotnumber control the size of the
%% string of dots
%%
%% If proof trees and formulae are mixed, some explicit spacing is needed,
%% but don't put anything to the left of the left-most (or the right of
%% the right-most) hypothesis, or put it in braces, because this will cause
%% the indentation to be lost.
%%
%% By default the conclusion is centered wrt the left-most and right-most
%% immediate hypotheses (not their proofs); \shiftright or \shiftleft moves
%% it relative to this position. (Not sure about this specification or how
%% it should affect spreading of proof tree.)
%
% global assignments to dimensions seem to have the effect of stretching
% diagrams horizontally.
%
%%==========================================================================

\def\introrule{{\cal I}}\def\elimrule{{\cal E}}%%
\def\andintro{\using{\land}\introrule\justifies}%%
\def\impelim{\using{\Rightarrow}\elimrule\justifies}%%
\def\allintro{\using{\forall}\introrule\justifies}%%
\def\allelim{\using{\forall}\elimrule\justifies}%%
\def\falseelim{\using{\bot}\elimrule\justifies}%%
\def\existsintro{\using{\exists}\introrule\justifies}%%

%% #1 is meant to be 1 or 2 for the first or second formula
\def\andelim#1{\using{\land}#1\elimrule\justifies}%%
\def\orintro#1{\using{\lor}#1\introrule\justifies}%%

%% #1 is meant to be a label corresponding to the discharged hypothesis/es
\def\impintro#1{\using{\Rightarrow}\introrule_{#1}\justifies}%%
\def\orelim#1{\using{\lor}\elimrule_{#1}\justifies}%%
\def\existselim#1{\using{\exists}\elimrule_{#1}\justifies}

%%==========================================================================

\newdimen\proofrulebreadth \proofrulebreadth=.05em
\newdimen\proofdotseparation \proofdotseparation=1.25ex
\newdimen\proofrulebaseline \proofrulebaseline=2ex
\newcount\proofdotnumber \proofdotnumber=3
\let\then\relax
\def\hfi{\hskip0pt plus.0001fil}
\mathchardef\squigto="3A3B
%
% flag where we are
\newif\ifinsideprooftree\insideprooftreefalse
\newif\ifonleftofproofrule\onleftofproofrulefalse
\newif\ifproofdots\proofdotsfalse
\newif\ifdoubleproof\doubleprooffalse
\let\wereinproofbit\relax
%
% dimensions and boxes of bits
\newdimen\shortenproofleft
\newdimen\shortenproofright
\newdimen\proofbelowshift
\newbox\proofabove
\newbox\proofbelow
\newbox\proofrulename
%
% miscellaneous commands for setting values
\def\shiftproofbelow{\let\next\relax\afterassignment\setshiftproofbelow\dimen0 }
\def\shiftproofbelowneg{\def\next{\multiply\dimen0 by-1 }%
\afterassignment\setshiftproofbelow\dimen0 }
\def\setshiftproofbelow{\next\proofbelowshift=\dimen0 }
\def\setproofrulebreadth{\proofrulebreadth}

%=============================================================================
\def\prooftree{% NESTED ZERO (\ifonleftofproofrule)
%
% first find out whether we're at the left-hand end of a proof rule
\ifnum  \lastpenalty=1
\then   \unpenalty
\else   \onleftofproofrulefalse
\fi
%
% some space on left (except if we're on left, and no infinity for outermost)
\ifonleftofproofrule
\else   \ifinsideprooftree
        \then   \hskip.5em plus1fil
        \fi
\fi
%
% begin our proof tree environment
\bgroup% NESTED ONE (\proofbelow, \proofrulename, \proofabove,
%               \shortenproofleft, \shortenproofright, \proofrulebreadth)
\setbox\proofbelow=\hbox{}\setbox\proofrulename=\hbox{}%
\let\justifies\proofover\let\leadsto\proofoverdots\let\Justifies\proofoverdbl
\let\using\proofusing\let\[\prooftree
\ifinsideprooftree\let\]\endprooftree\fi
\proofdotsfalse\doubleprooffalse
\let\thickness\setproofrulebreadth
\let\shiftright\shiftproofbelow \let\shift\shiftproofbelow
\let\shiftleft\shiftproofbelowneg
\let\ifwasinsideprooftree\ifinsideprooftree
\insideprooftreetrue
%
% now begin to set the top of the rule (definitions local to it)
\setbox\proofabove=\hbox\bgroup$\displaystyle % NESTED TWO
\let\wereinproofbit\prooftree
%
% these local variables will be copied out:
\shortenproofleft=0pt \shortenproofright=0pt \proofbelowshift=0pt
%
% flags to enable inner proof tree to detect if on left:
\onleftofproofruletrue\penalty1
}

%=============================================================================
% end whatever box and copy crucial values out of it
\def\eproofbit{% NESTED TWO
%
% various hacks applicable to hypothesis list 
\ifx    \wereinproofbit\prooftree
\then   \ifcase \lastpenalty
        \then   \shortenproofright=0pt  % 0: some other object, no indentation
        \or     \unpenalty\hfil         % 1: empty hypotheses, just glue
        \or     \unpenalty\unskip       % 2: just had a tree, remove glue
        \else   \shortenproofright=0pt  % eh?
        \fi
\fi
%
% pass out crucial values from scope
\global\dimen0=\shortenproofleft
\global\dimen1=\shortenproofright
\global\dimen2=\proofrulebreadth
\global\dimen3=\proofbelowshift
\global\dimen4=\proofdotseparation
\global\count255=\proofdotnumber
%
% end the box
$\egroup  % NESTED ONE
%
% restore the values
\shortenproofleft=\dimen0
\shortenproofright=\dimen1
\proofrulebreadth=\dimen2
\proofbelowshift=\dimen3
\proofdotseparation=\dimen4
\proofdotnumber=\count255
}

%=============================================================================
\def\proofover{% NESTED TWO
\eproofbit % NESTED ONE
\setbox\proofbelow=\hbox\bgroup % NESTED TWO
\let\wereinproofbit\proofover
$\displaystyle
}%
%
%=============================================================================
\def\proofoverdbl{% NESTED TWO
\eproofbit % NESTED ONE
\doubleprooftrue
\setbox\proofbelow=\hbox\bgroup % NESTED TWO
\let\wereinproofbit\proofoverdbl
$\displaystyle
}%
%
%=============================================================================
\def\proofoverdots{% NESTED TWO
\eproofbit % NESTED ONE
\proofdotstrue
\setbox\proofbelow=\hbox\bgroup % NESTED TWO
\let\wereinproofbit\proofoverdots
$\displaystyle
}%
%
%=============================================================================
\def\proofusing{% NESTED TWO
\eproofbit % NESTED ONE
\setbox\proofrulename=\hbox\bgroup % NESTED TWO
\let\wereinproofbit\proofusing
\kern0.3em$
}

%=============================================================================
\def\endprooftree{% NESTED TWO
\eproofbit % NESTED ONE
% \dimen0 =     length of proof rule
% \dimen1 =     indentation of conclusion wrt rule
% \dimen2 =     new \shortenproofleft, ie indentation of conclusion
% \dimen3 =     new \shortenproofright, ie
%                space on right of conclusion to end of tree
% \dimen4 =     space on right of conclusion below rule
  \dimen5 =0pt% spread of hypotheses
% \dimen6, \dimen7 = height & depth of rule
%
% length of rule needed by proof above
\dimen0=\wd\proofabove \advance\dimen0-\shortenproofleft
\advance\dimen0-\shortenproofright
%
% amount of spare space below
\dimen1=.5\dimen0 \advance\dimen1-.5\wd\proofbelow
\dimen4=\dimen1
\advance\dimen1\proofbelowshift \advance\dimen4-\proofbelowshift
%
% conclusion sticks out to left of immediate hypotheses
\ifdim  \dimen1<0pt
\then   \advance\shortenproofleft\dimen1
        \advance\dimen0-\dimen1
        \dimen1=0pt
%       now it sticks out to left of tree!
        \ifdim  \shortenproofleft<0pt
        \then   \setbox\proofabove=\hbox{%
                        \kern-\shortenproofleft\unhbox\proofabove}%
                \shortenproofleft=0pt
        \fi
\fi
%
% and to the right
\ifdim  \dimen4<0pt
\then   \advance\shortenproofright\dimen4
        \advance\dimen0-\dimen4
        \dimen4=0pt
\fi
%
% make sure enough space for label
\ifdim  \shortenproofright<\wd\proofrulename
\then   \shortenproofright=\wd\proofrulename
\fi
%
% calculate new indentations
\dimen2=\shortenproofleft \advance\dimen2 by\dimen1
\dimen3=\shortenproofright\advance\dimen3 by\dimen4
%
% make the rule or dots, with name attached
\ifproofdots
\then
        \dimen6=\shortenproofleft \advance\dimen6 .5\dimen0
        \setbox1=\vbox to\proofdotseparation{\vss\hbox{$\cdot$}\vss}%
        \setbox0=\hbox{%
                \advance\dimen6-.5\wd1
                \kern\dimen6
                $\vcenter to\proofdotnumber\proofdotseparation
                        {\leaders\box1\vfill}$%
                \unhbox\proofrulename}%
\else   \dimen6=\fontdimen22\the\textfont2 % height of maths axis
        \dimen7=\dimen6
        \advance\dimen6by.5\proofrulebreadth
        \advance\dimen7by-.5\proofrulebreadth
        \setbox0=\hbox{%
                \kern\shortenproofleft
                \ifdoubleproof
                \then   \hbox to\dimen0{%
                        $\mathsurround0pt\mathord=\mkern-6mu%
                        \cleaders\hbox{$\mkern-2mu=\mkern-2mu$}\hfill
                        \mkern-6mu\mathord=$}%
                \else   \vrule height\dimen6 depth-\dimen7 width\dimen0
                \fi
                \unhbox\proofrulename}%
        \ht0=\dimen6 \dp0=-\dimen7
\fi
%
% set up to centre outermost tree only
\let\doll\relax
\ifwasinsideprooftree
\then   \let\VBOX\vbox
\else   \ifmmode\else$\let\doll=$\fi
        \let\VBOX\vcenter
\fi
% this \vbox or \vcenter is the actual output:
\VBOX   {\baselineskip\proofrulebaseline \lineskip.2ex
        \expandafter\lineskiplimit\ifproofdots0ex\else-0.6ex\fi
        \hbox   spread\dimen5   {\hfi\unhbox\proofabove\hfi}%
        \hbox{\box0}%
        \hbox   {\kern\dimen2 \box\proofbelow}}\doll%
%
% pass new indentations out of scope
\global\dimen2=\dimen2
\global\dimen3=\dimen3
\egroup % NESTED ZERO
\ifonleftofproofrule
\then   \shortenproofleft=\dimen2
\fi
\shortenproofright=\dimen3
%
% some space on right and flag we've just made a tree
\onleftofproofrulefalse
\ifinsideprooftree
\then   \hskip.5em plus 1fil \penalty2
\fi
}

%==========================================================================
% IDEAS
% 1.    Specification of \shiftright and how to spread trees.
% 2.    Spacing command \m which causes 1em+1fil spacing, over-riding
%       exisiting space on sides of trees and not affecting the
%       detection of being on the left or right.
% 3.    Hack using \@currenvir to detect LaTeX environment; have to
%       use \aftergroup to pass \shortenproofleft/right out.
% 4.    (Pie in the sky) detect how much trees can be "tucked in"
% 5.    Discharged hypotheses (diagonal lines).


%\usepackage{lineno}
% \linenumbers

\usepackage{setspace}
\doublespacing

\setlength{\parskip}{0.7ex}

\title[]{Revised notes on the equivalence extension property}

\author{Nicola Gambino}

\date{3rd December 2018}

\newcommand{\C}{\mathsf{C}}
\newcommand{\TC}{\mathsf{TC}}
\newcommand{\F}{\mathsf{F}}
\newcommand{\TF}{\mathsf{TF}}
\newcommand{\W}{\mathsf{W}}

\begin{document}

\begin{abstract} These draft notes summarize discussions with Steve Awodey and Thierry Coquand during the Oberwolfach Seminar `Mathematical Logic: Proof Theory and Constructive Mathematics'
in November 2015. The discussion tried to clarify some steps in Christian Sattler's proof of the Equivalence Extension Property. Revised for joint work with Simon Henry.
\end{abstract} 


\maketitle


\section{Glueing} 

We wish to define the glueing functors. We begin by introducing their domain and codomain categories. 
For this, let us fix a cospan $(A, p, f)$, given by a pair of arrows with common codomain:
\[
\xymatrix{
 & A \ar[d]^{p} \\
Y \ar[r]_f & X \rlap{.}}
\]
In what follows, we write $(A, f)$ for such a cospan and consider $A$ as an object of $\cal{E}_{/X}$ leaving the map~$p$ implicit.  
We define the category $\mathsf{Span}(A, f)$ of spans over $(A, f)$. Its objects $(B, q, g)$ are spans of the form
\[
\xymatrix{
B \ar[r]^g \ar[d]_{q}   & A \\
X & }
\]
such that the diagram 
\begin{equation}
\label{equ:sq-span}
\begin{gathered}
\xymatrix{
B \ar[d]_{q} \ar[r]^{g}  & A \ar[d]^{p} \\
X \ar[r]_f & Y }
\end{gathered}
\end{equation}
commutes. In analogy with what we have done above, we write simply $(B,g)$ instead of~$(B,q,g)$ and consider $B$ as an object of $\cal{E}_{/Y}$ without mentioning the map $q$.  
The maps of $\mathsf{Span}(Y_1, i)$ are defined in the evident way. Note that  $\mathsf{Span}(A, f)$ has a terminal object, given by the pullback
\[
\xymatrix{
Y \times_X A \ar[r] \ar[d] & A \\
Y & }
\]

Next, we define the category $\cal{E}_{/p}$ of maps into $p \co A \to X$. Its objects $(A', p', u)$ consist of an object $A'$, a map $p' \co A \to X$ and a map $u \co A' \to A$ in $\cal{E}_{/X}$. 
Again, the maps of this category are evident. This category is canonically isomorphic to $\cal{E}_{/A}$, but the description given above will be useful later on, when we impose conditions
on $p' \co A' \to X$.


The glueing functor associated to the cospan $(A, f)$ has the form
\[
 \mathsf{Glue} \co \mathsf{Span}(A, f)  \to \cal{E}_{/p} \rlap{.}
\]
In what follows, we  denote the action of the glueing functor as
\[
\begin{gathered}
\xymatrix{
B \ar[d]_{q} \ar[r]^{g} & A \\
Y & }
\end{gathered} \quad
\longmapsto \quad
\begin{gathered}
\xymatrix{
\mathsf{Glue}(B, g) \ar[rr]^-{\mathsf{unglue}(B,g)}  \ar[dr] & &  A \ar[dl]^{p} \\
 & X & }
 \end{gathered}
 \]


\medskip

As we will see \cref{thm:glueing-classifies-sections}, for an object $(B, g) \in \mathsf{Span}(A, f)$, the sections of the map  $\mathsf{Glue}(B,g)  \to X$ correspond bijectively to
pairs of commuting sections of the square~\eqref{equ:sq-span}, \ie 
maps $a \co X \to A$ and $b \co Y \to B$ that are sections of~$p$ and~$q$, respectively, and such that 
$g b  = a f $. We represent this situation diagrammatically as follows:
\[
\xymatrix{ 
B \ar[d]_{g} \ar[r]^{g}  & A \ar[d]^{p} \\
\ar@/^2pc/@{.>}[u]^{b} Y \ar[r]_{f} & X  \ar@/_2pc/@{.>}[u]_{a} \rlap{.}}
\]



We now define the action of the glueing functor on objects. Let $(B,g) \in \mathsf{Span}(A, f)$. First, we consider the pullback of $p \co A \to X$ along $f \co Y \to X$ and the induced dotted map $\bar{g} \defeq 
(q,g)$ in 
\[
\xymatrix{
B \ar@/^1.5pc/[drr]^g \ar@/_1.5pc/[ddr]_{q} \ar@{.>}[dr]^{\bar{g}}  & & \\ 
 &   f_* (A) \ar[d]_{f_*{p}}  \ar[r]^{j_1}   & A \ar[d]^{p} \\
  & Y \ar[r]_{f}  & X \rlap{.}} 
\]


Next, we consider $\bar{g} \co B \to f^*(A)$ as a map in $\cal{E}_{/Y}$ and apply the functor 
$f_* \co \cal{E}_{/Y} \to \cal{E}_{/X}$ to it, so as to obtain the 
map $f^*(\bar{g}) \co  f_* (B)  \to f_* f^* (A)$ in $\cal{E}_{/X}$. 



{\bf Notation:} 
\begin{itemize}
\item $A[f] = f^*(A) = Y \times_X A \, ,$
\item $A^Y  = f_* f^*(A)$
\end{itemize}

\smallskip



Finally, we take the pullback of the map $f^*(\bar{g}) \co  g_* \, (B)  \to f_* f^* \, A $ that we just constructed along the component of unit of the adjunction
$f^* \dashv f_*$ at~$p \co A \to X$, which is a map $\eta \co A \to A^Y$ in~$\cal{E}_{/X}$. This gives us the diagram 

\begin{equation}
\label{equ:total-diagram}
\begin{gathered}
\xymatrix{
 B \ar[dr]^{\bar{f}} \ar@/_1pc/[ddr]_{q} \ar@/^1pc/[drr]^g &         &       \eta^* f_* \, B \drpullback                \ar[d] \ar[rr]  & & f_* B \ar[d]^{f_{*}(\bar{g})}  \\ 
      &             i_* \, A \ar[r]^{j_1}  \ar[d]  \drpullback &        A                \ar[dr]   \ar[rr]^\eta              &  & A^Y \ar[dl] \\
      &                       Y \ar[rr]_f                &  			          & X	 \rlap{.} & }
\end{gathered}
\end{equation}
This diagram contains all the constructions performed so far and we will refer to it frequently in what follows.  We then define the glueing functor by letting: 
\[
\begin{gathered}
\xymatrix{
\mathsf{Glue}(B,g) \ar[rr]^-{\mathsf{unglue}} \ar[dr]  & & A \ar[dl]^{p}   \\
 & X & }
 \end{gathered} \quad \defeq \quad
 \begin{gathered}
 \xymatrix{
  \eta^* f_* \, B \ar[rr]^{\eta^* f_{*} \bar{g}} \ar[dr] & & A \ar[dl]^{p} \\
   & X \rlap{.} & }
   \end{gathered}
 \]
 As usual, we often consider $\mathsf{Glue}(B,g)$ as an object of $\cal{E}_{/X}$ leaving the map into $X$  implicit. 
 
 
 \medskip


\begin{proposition} \label{thm:glueing-classifies-sections} There is a bijection between 
\begin{itemize}
\item sections of $\mathsf{Glue}(B, g) \to X$,
\item pairs of sections $a$ and $b$ of $p \co A \to X$ and $q \co B \to Y$, respectively, such that 
$g b = a f$.
\end{itemize}
\end{proposition} 

\begin{proof} To be added.
\end{proof} 

Note that, according to the bijection in \cref{thm:glueing-classifies-sections}, the map $\mathsf{unglue} \co \mathsf{Glue}(B, g) \to X$ corresponds to the second projection mapping
a commuting pair of sections $(a, b)$ to $b$. Also note that \cref{thm:glueing-classifies-sections} expresses that the category-theoretic glueing construction considered here is analogous 
to the type-theoretic construction, which has rules

\[
\begin{prooftree}
X, f  \vdash A \quad
X \vdash B \co \mathsf{type} \quad
X , f  \vdash \bar{g} \co B \to f^*\, A \quad
\justifies
X \vdash \mathsf{Glue}(B,g) 
\end{prooftree} \qquad
\begin{prooftree}
X \vdash a \co A \quad
X, f  \vdash b \co B \quad
X, f \vdash \bar{g}(b) = a \co f^* \, A
\justifies
X \vdash \mathsf{glue}(a,b) \co \mathsf{Glue}(B,g) 
\end{prooftree}
\]



Just as the ordinary pushforward functors, which classify sections of maps, are right adjoint, so is the glueing functors. 

\begin{proposition} Glueing functors are right adjoint. Hence, they preserve limits and in particular pullbacks.
\end{proposition} 

\begin{proof} This needs checking. 
\end{proof} 

The next proposition makes use of map $j_1 \co f^*(A) \to A$ that is part of the diagram in~\eqref{equ:total-diagram}. 

\begin{proposition} The glueing functor is naturally isomorphic to the functor 
\[
\xymatrix{
(j_1)_*(-,-) \co \mathsf{Span}(A, f) \ar[r] &  \cal{E}_{/A}}
\]
mapping $(B, g)$ to $(j_1)_*(g,q) \co {j_1}_*(B) \to A$. 
\end{proposition}

\begin{proof} By the definition of the glueing functors, it suffices to show that, for $t \co B \to f^*(A)$ in~$\cal{E}_{/A}$, $\eta^* f_*(t) \co \eta^* f_* \, B \to A$ in $\cal{E}_{/A}$ has the same universal property 
of $(j_1)_*(t) \co j_*(B) \to A$. This is a straightforward 
diagram-chasing proof.
\end{proof} 

\section{The equivalence extension property} 

We continue to work with the fixed cospan $(A, f)$, but now assume that $p \co A \to X$ is a fibration and 
$f \co A \to B$ is a cofibration. Let us consider $(B, g) \in \mathsf{Span}(A, f)$ such
that $q \co B \to Y$ is a fibration and $\bar{g} \co B \to Y \times_X A$ is an equivalence, where, as before $\bar{g} \defeq (q,g)$. Note that this gives us the arrows in the diagram below: 
\[
\xymatrix{
  B
  % \ar@{.>}[rr]
  \ar[dr]^{\bar{g}}
  \ar[dd]_{q}
%  \fancypullback{[dd]}{[rrr]}{[ddrrr]}[0.3cm][0.6]
&&
 %  \bullet
  % \ar@{.>}[dr]
 % \ar@{.>}[dd]|{\hole}
&\\&
  Y \times_X A 
  \ar[rr]^(.45){j_1}
  \ar[dl]
%  \fancypullback{[dl]}{[rr]}{[dr]}[0.3cm][0.6]
&&
  A
  \ar[dl]^{p}
\\
  Y
  \ar[rr]_f
&&
  X \rlap{.} 
&
}
\]

Next, we factor $\bar{g} \co  B \to Y \times_X A$ in $\cal{E}_{/Y}$ via its mapping path object $M$, as follows:
\[
\xymatrix{ 
 B \ar[r]^{s} & M \ar[r]^-t  &  Y \times_X A \rlap{.} }
 \]
 Here, $M$ is obtained by the diagram 
 \begin{equation}
 \label{equ:def-of-mapping-path}
 \begin{gathered}
\xymatrix{
B \ar[r] \ar[d]_s \drpullback  \ar@/_2pc/[dd]_{1_B} & Y \times_X A \ar[d] \ar[r]^-{j_1} \drpullback & A \ar[d]  \ar@/^2pc/[dd]^{1_{Y_1}} \\
M \ar[r] \ar[d] \drpullback & Y \times_X {A}^I \ar[d] \ar[r] \drpullback& A^I \ar[d] \\ 
B \ar[r] \ar[d] & Y \times_X A \ar[r] \ar[d]  \drpullback & A \ar[d]^p  \\
Y \ar@{=}[r] & Y \ar[r]_f & X \rlap{.}  }
\end{gathered}
\end{equation}
 If we apply the glueing functor, we get 
\[
\xymatrix{
  B \ar[dr]^{s} \ar@{.>}[rr] 
  \ar[dddr]_{q} & & \mathsf{Glue}(B,g) \ar[dr]^{s'} & & \\
  % \ar@{.>}[rr]
%  \fancypullback{[dd]}{[rrr]}{[ddrrr]}[0.3cm][0.6]
& M \ar[dr]^t \ar@{.>}[rr] & & \mathsf{Glue}(M) \ar[dr]^{\ell}
 %  \bullet
  % \ar@{.>}[dr]
 % \ar@{.>}[dd]|{\hole}
& \\& & 
  Y \times_X A 
  \ar[rr]^(.35){j_1}
  \ar[dl]^{f^* p}
%  \fancypullback{[dl]}{[rr]}{[dr]}[0.3cm][0.6]
&&
  A
  \ar[dl]^{p}
\\
&   Y
  \ar[rr]_f
&&
  X \rlap{.} 
&
}
\]
 
Here, we used that the glueing functor preserves limits and in particular the terminal object to deduce that 
\begin{equation}
\label{equ:usethis}
\mathsf{Glue}(Y \times_X A) \cong A
\end{equation}



 \begin{center}
\framebox{ 
{\bf Notation:} \quad
$\overline{B} = \mathsf{Glue}(B, g) \, , \qquad \ \overline{M} = \mathsf{Glue}(M)$ } 
\end{center}

\caution{Note: for the rest of the argument, we need to show
\begin{itemize}
\item $t$ is a trivial fibration, which should follow from $\bar{g}$ being an equivalence.
\item  a counterpart of Sattler's Lemma 4.16, asserting that a map between fibrant objects is an equivalence if and only if it factors as a section of a trivial fibration followed
by a trivial fibration.
\end{itemize}}



We need to show that~$s'$ factors as a section of a trivial fibration followed by a trivial fibration. 
For this, one constructs a map $u$ as follows 
\begin{equation}
\label{equ:define-u}
\begin{gathered}
\xymatrix{
A \ar@/_1pc/[ddr]_{u_1}  \ar@/^1pc/[drr]^{1_A} \ar@{.>}[dr]^{u} & & \\
  & (A^I)^Y \times_{A^Y} A \ar[r]^(.6){v} \ar[d] & A \ar[d]^{\eta} \\ 
 & (A^I)^Y \ar[r]_{\partial_1^Y} & A^Y }
 \end{gathered}
 \end{equation}
 and then proves separately the following two claims:
\begin{enumerate}
\item[(a)] $s'$ is the pullback of the map $u$
\item[(b)] $u$ factors as a section of a trivial fibration over $Y_1^A$ followed by a trivial fibration. 
\end{enumerate} 

 \begin{center}
\framebox{ \medskip 
{\bf Notation:} $P \defeq  (A^I)^Y \times_{A^Y} A$}
\end{center} 

\noindent 
{\bf {Proof of claim (a)}.} Observe that  
\begin{align*}
P & = (A^I)^Y \times_{A^Y} A \\
 & = \eta^* (A^I)^Y \\
 & = \eta^* f_* f^* (A^I) \\
 & = \eta^* f_* (Y \times_X A^I) \\
 & =  \mathsf{Glue}(Y \times_X A^I)
\end{align*}

%\noten{The previous version of these notes contained a mistake, describing $P$ as 
% $\mathsf{Glue}(Y_1^I)$ rather than $\mathsf{Glue}(A \times_B Y_1^A)$.}



We wish to show that $s'$ fits into a pullback diagram of the form
\begin{equation*}
\label{equ:key}
\begin{gathered}
\xymatrix{
\overline{B} \ar[d]_{s'}  \ar[r] &  A \ar[d]^{u} \\
N \ar[r] & P \rlap{.} }
\end{gathered} 
\end{equation*}
By definition of $\bar{B}$, $N$ and $P$ and the isomorphism in~\eqref{equ:usethis}, this diagram is given by 
\begin{equation*}
\label{equ:key-as-glueing}
\begin{gathered}
\xymatrix{
\mathsf{Glue}(B) \ar[d]   \ar[r]   & \mathsf{Glue}(Y \times_X A) \ar[d]  \\
\mathsf{Glue}(M) \ar[r]   & \mathsf{Glue}(Y \times_X A^I) }
\end{gathered} 
\end{equation*}
Since the glueing functor preserves pullbacks, it suffices to show that 
\[
\xymatrix{
B \ar[d]   \ar[r]   & Y \times_X A \ar[d]  \\
M \ar[r]   & Y \times_X A^I \rlap{.}}
\] 
is a pullback. But this diagram is the top left  square in the diagram~\eqref{equ:def-of-mapping-path},
which is a pullback by construction. 

%\noten{Because of the incorrect description of the object $P$ in the previous draft mentioned above, the conclusion of the proof involved another step, asserting that composition preserved pullbacks. %But that seems incorrect as well.}



\medskip

{\bf Proof of claim (b).} Observe that $u$ factors as follows:
\[
\xymatrix{
A \ar[rr]^-u  \ar[dr]_{r}  & &  (A^I)^Y \times_{A^Y} A \\
 & A^I \ar[ur]_{t}  & }
 \]
Here $r$  the `constant path' map, which is a section to a trivial fibration. The map $t$ is the pullback exponential of the cofibration $f \co Y \to X$
with the trivial fibration $A^I \to A$ and so it is a trivial fibration by the dual of the pushout-product property.


%\noten{The definition of $u$ above seems to prove something stronger than what is stated in claim (b), namely that $u$ itself is a section of a trivial fibration. Indeed, consider the diagram 
%in~\eqref{equ:define-u} and observe that $\delta_1$ is
%a trivial fibration, that $(-)^A$ preserves trivial fibrations and that pullback preserves trivial fibrations. Hence, the map $v$ in~\eqref{equ:define-u}
%is a trivial fibration and $u$ is a section of $v$ by definition. The paper by Christian suggests that being a section over $Y_1^A$ is important, but without any hint of where this is used. In any case, it % seems that also $u$ is a section of $v$ over $Y_1^A$.}


\end{document}