\documentclass[reqno,10pt,a4paper,oneside,draft]{amsart}
\setcounter{tocdepth}{1}
\usepackage{../Auxiliary/prelude}



\title[]{Notes on simplicial homotopy theory}

\begin{document}

\begin{abstract}
Some notes on simplicial homotopy theory
\end{abstract}

\author{Nicola Gambino}

\author{Simon Henry}


 \date{\today}

\maketitle

\section{Introduction}

\newpage

\section{Constructive simplicial homotopy theory}

\begin{itemize}
\item Basics on $\SSet$ and notation. 
\item We use $\Delta[n]$, $\partial \Delta[n]$ and $\Lambda^{k}[n]$ and following Joyal.
\item Recall the existence of a weak model structure on $\SSet$:
\begin{itemize}
\item Fibrations  = Kan fibrations
\item Cofibrations = Levelwise complemented monomorphisms such that degeracies are decidable on the complement of the image. 
\item Cofibrant objects are simplicial sets in which degeneracies are decidable.
\end{itemize}
\item It would be good to have an explicit definition of the cofibrant replacement functor.
\item Recall the pushout product property.
\end{itemize}






\begin{proposition} Cofibrations are closed under pullbacks.
\end{proposition}

\begin{proof} To be added.
\end{proof} 



\begin{lemma} For $0 \leq k \leq n$, the horn inclusion $i \co \Lambda^k[n] \to \Delta[n]$ is a retract of
the pushout product
\[
i \hattimes \kcyl \co \big( \Lambda^k[n] \times \Delta[1] \big) \cup (\Delta[n] \times {k}) \to \Delta[n] \times \Delta[1] \, .
\]
\end{lemma} 

\begin{proposition} A map is a Kan fibration if and only if it has the right lifting property with respect to the pushout products $i \hattimes \kcyl$.
\end{proposition} 


\newpage

\section{Dependent products}

\begin{itemize}
\item We wish to establish a restricted form of the Frobenius condition.
\item We follow Gambino-Sattler, but without making use of the assumption that every object is cofibrant.
\end{itemize}

\newpage

\section{The universe}

Recall that we work in a constructive set theory with two universes $\mathsf{u}_1$ and $\mathsf{u}_2$
and that we refer to elements of $\mathsf{u}_1$ as small sets. We then define a simplicial
set $X$ to be \emph{small}


\begin{definition} \hfill 
\label{thm:small}
\begin{enumerate}[(i)]
\item We say that a simplicial set $X$ is \emph{small}  if $X_n$ is a small set for every $[n] \in \Delta$. 
\item We say that a map $f \co Y \to X$ in $\SSet$ is \emph{small} if for every $x \co \Delta[n] 
\to X$ the simplicial set~$Y_x$ fitting in the pullback square
\[
\xymatrix{
Y_x \ar[r] \ar[d] \drpullback & Y \ar[d]^{f} \\
\Delta[n] \ar[r]_x & X }
\]
is small.
\end{enumerate}
\end{definition} 

By the results in~\cite{hofmann-streicher-universes} for arbitrary presheaf categories, small maps in
 $\SSet$ admit a weak classifier, \ie a small map $\rho \co \VV \to \V$ such 
that for every small map $f \co Y \to X$ there exists  a pullback diagram of the form
\[
\xymatrix{
Y \ar[r] \ar[d]_f \drpullback  & \VV \ar[d]^\rho \\
X \ar[r] & \V }
\]
Letting $X = \Delta[n]$ in this diagram suggests to define $\V_n$ as the set of all small maps with codomain~$\Delta[n]$. In this way, however, one does not obtain a presheaf since the transition functions
will satisfy the functorial laws only up to isomorphism rather than equality. To remedy this, the $n$-simplices of $\V$ are defined instead to be the functors $F \co (\Delta/[n])^{\op} \to \Set$ such that 
the corresponding map of simplicial sets $ \mathsf{El}(F) \to \Delta[n]$ is small. MORE TO BE ADDED.

\medskip

Following~\cite{cisinski-univalence,voevodsky-simplicial-model}, we consider the pullback 
\[
\xymatrix{
\UU \ar[r] \ar[d]_\pi \drpullback  & \VV \ar[d]^\rho \\
\U \ar[r] & \V }
\]
where $\U \subseteq \V$ is defined by letting 
\[
\U_n = \{ F \in \V_n \ | \ \mathsf{El}(F) \to \Delta[n] \text{ is a small Kan fibration } \}
\]


\begin{proposition} \label{thm:universe-u}  \hfill 
\begin{enumerate}[(i)] 
\item $\pi \co \UU \to \U$ is a small Kan fibration.
\item $\pi \co \UU \to \U$ classifies small Kan fibrations, \ie 
for every small Kan fibration $f \co Y \to X$ there exists  a pullback diagram of the form
\[
\xymatrix{
Y \ar[r] \ar[d]_f \drpullback & \UU \ar[d]^\pi \\
X \ar[r] & \U }
\]
\item  The simplicial set $\UU$ is cofibrant.
\end{enumerate}
\end{proposition}

\begin{proof} We prove the three claims separately.
\begin{enumerate}[(i)] 
\item Should follow by locality.
\item Should be immediate.
\item See handwritten notes. Key step is the constructive version of the Eilenberg-Zilber lemma. \qedhere
\end{enumerate}
\end{proof} 

However, the simplicial set $\U$ does not appear to be cofibrant and hence it does not seem possible to show that  $\pi \co \UU \to \U$ is a weak classifier for small Kan fibrations with cofibrant codomain. In order to remedy this, we consider the cofibrant replacement $\U_c$ of $\U$, which comes equipped
with a trivial fibration $p \co \U_c \to \U$, and the pullback
\[
\xymatrix{
\UU_c \ar[d]_{\pi_c} \ar[r] \drpullback & \UU \ar[d]^{\pi}  \\
\U_c \ar[r]_p & \U}
\]
We can now prove that $\pi_c \co \UU_c \to \U_c$ has the desired properties.


\begin{proposition} \label{thm:universe-uc} 
\hfill 
\begin{enumerate}[(i)] 
\item $\pi_c \co \UU_c \to \U_c$ is a small Kan fibration with fibrant codomain. 
\item The map $\pi_c \co \UU_c \to \U_c$ classifies small Kan fibrations with cofibrant domain, \ie 
for every small Kan fibration $f \co Y \to X$ with $X$ cofibrant there exists  a pullback diagram of the form
\[
\xymatrix{
Y \ar[r] \ar[d]_f & \UU_c \ar[d]^{\pi_c} \\
X \ar[r] & \U_c }
\]
\item The simplicial set $\UU_c$ is cofibrant.
\end{enumerate}
\end{proposition}

\begin{proof} Part~(i) follows from part~(i) of~\cref{thm:universe-u}. For part~(ii), 
let $f \co Y \to X$ be a small Kan fibration with $X$ cofibrant. Since $f$ is a
small Kan fibration, we know 
from~\cref{thm:universe-u} that there is a pullback diagram of the form 
\[
\xymatrix{
Y \ar[r] \ar[d]_f \drpullback & \UU \ar[d]^{\pi} \\
X \ar[r] & \U }
\]
Since $X$ is cofibrant, we have the lifting diagram
\[
\xymatrix{
0 \ar[r] \ar[d] & \U_c \ar[d]^{p} \\
X \ar[r] \ar@{.>}[ur] & \U }
\]
which shows that the map $X \to \U$ factors via $\U_c$.  We then obtain the diagram
\[
\xymatrix{
Y \ar[r] \ar[d]_f &  \UU_c \ar[r]  \ar[d]^{\pi_c} \drpullback & \UU \ar[d]^{\pi} \\
X \ar[r] & \U_c \ar[r]_p &  \U }
\]
Here, the right-hand side square and the rectangle are pullbacks and therefore the left-hand
side square is also a pullback, as required. Part (iii) follows from the fact that both $\U_c$ and
$\UU$ are cofibrant, the latter being part (iii) of~\cref{thm:universe-u}.
\end{proof} 








Note that we have not shown yet that $\U_c$ fibrant. This will be done in~\cref{sec:fibuu}, as a consequence of the equivalence extension property for fibrations, which we establish in~\cref{sec:equep}.

\newpage

\section{The equivalence extension property}
\label{sec:equep}

\begin{itemize}
\item Here we follow Kapulkin and Lumsdaine.
\end{itemize}

\newpage

\section{Fibrancy and univalence of the universe}
\label{sec:fibuu}

\begin{itemize}
\item Fibrancy should follow directly from equivalence extension property, without using `composition vs filling' but rather retract property for horns (see notes).
\item Once we have established fibrancy of $U_c$, then one can prove univalence by showing that 
$t \co \mathsf{Weq}(U_c) \to U_c$ is a trivial fibration. 
\item Question: do we need to know that $\mathsf{Weq}(U_c)$ is a cofibrant object to get univalence? 
\end{itemize}


\newpage

\section{Semantics}


\begin{itemize}
\item This should be essentially straightforward, following Kapulkin and Lumsdaine, but we may need to modify the notion of a $\Pi$-structure to accommodate the cofibrant replacements that we take for $\Pi$.
\end{itemize}

\nocite{*}


\bibliographystyle{plain}
\bibliography{../Auxiliary/bibliography}

\end{document}