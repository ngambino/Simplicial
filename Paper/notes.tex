\documentclass[reqno,10pt,a4paper,oneside,draft]{amsart}
\setcounter{tocdepth}{1}
\usepackage{../Auxiliary/prelude}
\message{<Paul Taylor's Proof Trees, 2 August 1996>}
%% Build proof tree for Natural Deduction, Sequent Calculus, etc.
%% WITH SHORTENING OF PROOF RULES!
%% Paul Taylor, begun 10 Oct 1989
%% *** THIS IS ONLY A PRELIMINARY VERSION AND THINGS MAY CHANGE! ***
%%
%% 2 Aug 1996: fixed \mscount and \proofdotnumber
%%
%%      \prooftree
%%              hyp1            produces:
%%              hyp2
%%              hyp3            hyp1    hyp2    hyp3
%%      \justifies              -------------------- rulename
%%              concl                   concl
%%      \thickness=0.08em
%%      \shiftright 2em
%%      \using
%%              rulename
%%      \endprooftree
%%
%% where the hypotheses may be similar structures or just formulae.
%%
%% To get a vertical string of dots instead of the proof rule, do
%%
%%      \prooftree                      which produces:
%%              [hyp]
%%      \using                                  [hyp]
%%              name                              .
%%      \proofdotseparation=1.2ex                 .name
%%      \proofdotnumber=4                         .
%%      \leadsto                                  .
%%              concl                           concl
%%      \endprooftree
%%
%% Within a prooftree, \[ and \] may be used instead of \prooftree and
%% \endprooftree; this is not permitted at the outer level because it
%% conflicts with LaTeX. Also,
%%      \Justifies
%% produces a double line. In LaTeX you can use \begin{prooftree} and
%% \end{prootree} at the outer level (however this will not work for the inner
%% levels, but in any case why would you want to be so verbose?).
%%
%% All of of the keywords except \prooftree and \endprooftree are optional
%% and may appear in any order. They may also be combined in \newcommand's
%% eg "\def\Cut{\using\sf cut\thickness.08em\justifies}" with the abbreviation
%% "\prooftree hyp1 hyp2 \Cut \concl \endprooftree". This is recommended and
%% some standard abbreviations will be found at the end of this file.
%%
%% \thickness specifies the breadth of the rule in any units, although
%% font-relative units such as "ex" or "em" are preferable.
%% It may optionally be followed by "=".
%% \proofrulebreadth=.08em or \setlength\proofrulebreadth{.08em} may also be
%% used either in place of \thickness or globally; the default is 0.04em.
%% \proofdotseparation and \proofdotnumber control the size of the
%% string of dots
%%
%% If proof trees and formulae are mixed, some explicit spacing is needed,
%% but don't put anything to the left of the left-most (or the right of
%% the right-most) hypothesis, or put it in braces, because this will cause
%% the indentation to be lost.
%%
%% By default the conclusion is centered wrt the left-most and right-most
%% immediate hypotheses (not their proofs); \shiftright or \shiftleft moves
%% it relative to this position. (Not sure about this specification or how
%% it should affect spreading of proof tree.)
%
% global assignments to dimensions seem to have the effect of stretching
% diagrams horizontally.
%
%%==========================================================================

\def\introrule{{\cal I}}\def\elimrule{{\cal E}}%%
\def\andintro{\using{\land}\introrule\justifies}%%
\def\impelim{\using{\Rightarrow}\elimrule\justifies}%%
\def\allintro{\using{\forall}\introrule\justifies}%%
\def\allelim{\using{\forall}\elimrule\justifies}%%
\def\falseelim{\using{\bot}\elimrule\justifies}%%
\def\existsintro{\using{\exists}\introrule\justifies}%%

%% #1 is meant to be 1 or 2 for the first or second formula
\def\andelim#1{\using{\land}#1\elimrule\justifies}%%
\def\orintro#1{\using{\lor}#1\introrule\justifies}%%

%% #1 is meant to be a label corresponding to the discharged hypothesis/es
\def\impintro#1{\using{\Rightarrow}\introrule_{#1}\justifies}%%
\def\orelim#1{\using{\lor}\elimrule_{#1}\justifies}%%
\def\existselim#1{\using{\exists}\elimrule_{#1}\justifies}

%%==========================================================================

\newdimen\proofrulebreadth \proofrulebreadth=.05em
\newdimen\proofdotseparation \proofdotseparation=1.25ex
\newdimen\proofrulebaseline \proofrulebaseline=2ex
\newcount\proofdotnumber \proofdotnumber=3
\let\then\relax
\def\hfi{\hskip0pt plus.0001fil}
\mathchardef\squigto="3A3B
%
% flag where we are
\newif\ifinsideprooftree\insideprooftreefalse
\newif\ifonleftofproofrule\onleftofproofrulefalse
\newif\ifproofdots\proofdotsfalse
\newif\ifdoubleproof\doubleprooffalse
\let\wereinproofbit\relax
%
% dimensions and boxes of bits
\newdimen\shortenproofleft
\newdimen\shortenproofright
\newdimen\proofbelowshift
\newbox\proofabove
\newbox\proofbelow
\newbox\proofrulename
%
% miscellaneous commands for setting values
\def\shiftproofbelow{\let\next\relax\afterassignment\setshiftproofbelow\dimen0 }
\def\shiftproofbelowneg{\def\next{\multiply\dimen0 by-1 }%
\afterassignment\setshiftproofbelow\dimen0 }
\def\setshiftproofbelow{\next\proofbelowshift=\dimen0 }
\def\setproofrulebreadth{\proofrulebreadth}

%=============================================================================
\def\prooftree{% NESTED ZERO (\ifonleftofproofrule)
%
% first find out whether we're at the left-hand end of a proof rule
\ifnum  \lastpenalty=1
\then   \unpenalty
\else   \onleftofproofrulefalse
\fi
%
% some space on left (except if we're on left, and no infinity for outermost)
\ifonleftofproofrule
\else   \ifinsideprooftree
        \then   \hskip.5em plus1fil
        \fi
\fi
%
% begin our proof tree environment
\bgroup% NESTED ONE (\proofbelow, \proofrulename, \proofabove,
%               \shortenproofleft, \shortenproofright, \proofrulebreadth)
\setbox\proofbelow=\hbox{}\setbox\proofrulename=\hbox{}%
\let\justifies\proofover\let\leadsto\proofoverdots\let\Justifies\proofoverdbl
\let\using\proofusing\let\[\prooftree
\ifinsideprooftree\let\]\endprooftree\fi
\proofdotsfalse\doubleprooffalse
\let\thickness\setproofrulebreadth
\let\shiftright\shiftproofbelow \let\shift\shiftproofbelow
\let\shiftleft\shiftproofbelowneg
\let\ifwasinsideprooftree\ifinsideprooftree
\insideprooftreetrue
%
% now begin to set the top of the rule (definitions local to it)
\setbox\proofabove=\hbox\bgroup$\displaystyle % NESTED TWO
\let\wereinproofbit\prooftree
%
% these local variables will be copied out:
\shortenproofleft=0pt \shortenproofright=0pt \proofbelowshift=0pt
%
% flags to enable inner proof tree to detect if on left:
\onleftofproofruletrue\penalty1
}

%=============================================================================
% end whatever box and copy crucial values out of it
\def\eproofbit{% NESTED TWO
%
% various hacks applicable to hypothesis list 
\ifx    \wereinproofbit\prooftree
\then   \ifcase \lastpenalty
        \then   \shortenproofright=0pt  % 0: some other object, no indentation
        \or     \unpenalty\hfil         % 1: empty hypotheses, just glue
        \or     \unpenalty\unskip       % 2: just had a tree, remove glue
        \else   \shortenproofright=0pt  % eh?
        \fi
\fi
%
% pass out crucial values from scope
\global\dimen0=\shortenproofleft
\global\dimen1=\shortenproofright
\global\dimen2=\proofrulebreadth
\global\dimen3=\proofbelowshift
\global\dimen4=\proofdotseparation
\global\count255=\proofdotnumber
%
% end the box
$\egroup  % NESTED ONE
%
% restore the values
\shortenproofleft=\dimen0
\shortenproofright=\dimen1
\proofrulebreadth=\dimen2
\proofbelowshift=\dimen3
\proofdotseparation=\dimen4
\proofdotnumber=\count255
}

%=============================================================================
\def\proofover{% NESTED TWO
\eproofbit % NESTED ONE
\setbox\proofbelow=\hbox\bgroup % NESTED TWO
\let\wereinproofbit\proofover
$\displaystyle
}%
%
%=============================================================================
\def\proofoverdbl{% NESTED TWO
\eproofbit % NESTED ONE
\doubleprooftrue
\setbox\proofbelow=\hbox\bgroup % NESTED TWO
\let\wereinproofbit\proofoverdbl
$\displaystyle
}%
%
%=============================================================================
\def\proofoverdots{% NESTED TWO
\eproofbit % NESTED ONE
\proofdotstrue
\setbox\proofbelow=\hbox\bgroup % NESTED TWO
\let\wereinproofbit\proofoverdots
$\displaystyle
}%
%
%=============================================================================
\def\proofusing{% NESTED TWO
\eproofbit % NESTED ONE
\setbox\proofrulename=\hbox\bgroup % NESTED TWO
\let\wereinproofbit\proofusing
\kern0.3em$
}

%=============================================================================
\def\endprooftree{% NESTED TWO
\eproofbit % NESTED ONE
% \dimen0 =     length of proof rule
% \dimen1 =     indentation of conclusion wrt rule
% \dimen2 =     new \shortenproofleft, ie indentation of conclusion
% \dimen3 =     new \shortenproofright, ie
%                space on right of conclusion to end of tree
% \dimen4 =     space on right of conclusion below rule
  \dimen5 =0pt% spread of hypotheses
% \dimen6, \dimen7 = height & depth of rule
%
% length of rule needed by proof above
\dimen0=\wd\proofabove \advance\dimen0-\shortenproofleft
\advance\dimen0-\shortenproofright
%
% amount of spare space below
\dimen1=.5\dimen0 \advance\dimen1-.5\wd\proofbelow
\dimen4=\dimen1
\advance\dimen1\proofbelowshift \advance\dimen4-\proofbelowshift
%
% conclusion sticks out to left of immediate hypotheses
\ifdim  \dimen1<0pt
\then   \advance\shortenproofleft\dimen1
        \advance\dimen0-\dimen1
        \dimen1=0pt
%       now it sticks out to left of tree!
        \ifdim  \shortenproofleft<0pt
        \then   \setbox\proofabove=\hbox{%
                        \kern-\shortenproofleft\unhbox\proofabove}%
                \shortenproofleft=0pt
        \fi
\fi
%
% and to the right
\ifdim  \dimen4<0pt
\then   \advance\shortenproofright\dimen4
        \advance\dimen0-\dimen4
        \dimen4=0pt
\fi
%
% make sure enough space for label
\ifdim  \shortenproofright<\wd\proofrulename
\then   \shortenproofright=\wd\proofrulename
\fi
%
% calculate new indentations
\dimen2=\shortenproofleft \advance\dimen2 by\dimen1
\dimen3=\shortenproofright\advance\dimen3 by\dimen4
%
% make the rule or dots, with name attached
\ifproofdots
\then
        \dimen6=\shortenproofleft \advance\dimen6 .5\dimen0
        \setbox1=\vbox to\proofdotseparation{\vss\hbox{$\cdot$}\vss}%
        \setbox0=\hbox{%
                \advance\dimen6-.5\wd1
                \kern\dimen6
                $\vcenter to\proofdotnumber\proofdotseparation
                        {\leaders\box1\vfill}$%
                \unhbox\proofrulename}%
\else   \dimen6=\fontdimen22\the\textfont2 % height of maths axis
        \dimen7=\dimen6
        \advance\dimen6by.5\proofrulebreadth
        \advance\dimen7by-.5\proofrulebreadth
        \setbox0=\hbox{%
                \kern\shortenproofleft
                \ifdoubleproof
                \then   \hbox to\dimen0{%
                        $\mathsurround0pt\mathord=\mkern-6mu%
                        \cleaders\hbox{$\mkern-2mu=\mkern-2mu$}\hfill
                        \mkern-6mu\mathord=$}%
                \else   \vrule height\dimen6 depth-\dimen7 width\dimen0
                \fi
                \unhbox\proofrulename}%
        \ht0=\dimen6 \dp0=-\dimen7
\fi
%
% set up to centre outermost tree only
\let\doll\relax
\ifwasinsideprooftree
\then   \let\VBOX\vbox
\else   \ifmmode\else$\let\doll=$\fi
        \let\VBOX\vcenter
\fi
% this \vbox or \vcenter is the actual output:
\VBOX   {\baselineskip\proofrulebaseline \lineskip.2ex
        \expandafter\lineskiplimit\ifproofdots0ex\else-0.6ex\fi
        \hbox   spread\dimen5   {\hfi\unhbox\proofabove\hfi}%
        \hbox{\box0}%
        \hbox   {\kern\dimen2 \box\proofbelow}}\doll%
%
% pass new indentations out of scope
\global\dimen2=\dimen2
\global\dimen3=\dimen3
\egroup % NESTED ZERO
\ifonleftofproofrule
\then   \shortenproofleft=\dimen2
\fi
\shortenproofright=\dimen3
%
% some space on right and flag we've just made a tree
\onleftofproofrulefalse
\ifinsideprooftree
\then   \hskip.5em plus 1fil \penalty2
\fi
}

%==========================================================================
% IDEAS
% 1.    Specification of \shiftright and how to spread trees.
% 2.    Spacing command \m which causes 1em+1fil spacing, over-riding
%       exisiting space on sides of trees and not affecting the
%       detection of being on the left or right.
% 3.    Hack using \@currenvir to detect LaTeX environment; have to
%       use \aftergroup to pass \shortenproofleft/right out.
% 4.    (Pie in the sky) detect how much trees can be "tucked in"
% 5.    Discharged hypotheses (diagonal lines).



\title[]{Towards a constructive  simplicial model \\ of Homotopy Type Theory}

\begin{document}

\begin{abstract}
We provide a partial solution to the open problem of defining a constructive version of 
Voevodsky's simplicial model of Homotopy Type Theory. This is done by establishing
constructive counterparts of the main results of simplicial homotopy theory
necessary to define model. In particular, 
 we show constructively that the weak factorisation system of trivial cofibrations and Kan fibrations satisfies a
restricted version of the Frobenius property, that the model structure satisfies the weak equivalence extension property
for weak equivalences between fibrations with cofibrant domain and that there exists a univalent 
 fibration with bifibrant base that classifies small fibrations between bifibrant objects.
\end{abstract}

\author{Nicola Gambino}
\address{School of Mathematics, University of Leeds, Leeds LS2 9JT, United Kingdom}
\email{n.gambino@leeds.ac.uk}


\author{Simon Henry}
\address{Department of Mathematics and Statistics, Masaryk University, Brno, Czeck Republic}
\email{henrys@math.muni.cz} 


 \date{\today}
 
 

\maketitle



\section*{Introduction} 

This paper investigates Voevodsky's model of Homotopy  Type Theory (HoTT), \ie Martin-L\"of type theory~\cite{nordstrom-petersson-smith:ml} extended with the univalence axiom~\cite{voevodsky:univalence}, in the category of simplicial sets~\cite{voevodsky-simplicial-model}. This model is of fundamental importance since it informs a new approach to the development of mathematics
and its computer-assisted formalisation, guided by the notion of the homotopy level of a type and by the univalence axiom, which makes precise the idea that isomorphic structures should be considered as equal. 

The original definition of the simplicial model  was carried out working within classical Zermelo-Fraenkel set theory extended with the axiom of choice (ZFC) and two inaccessible cardinals. Given the wide gap in proof-theoretic strength between Martin-L\"of type theory and ZFC~\cite{GrifforE:strsml} and the fact that the former is constructive  while the latter is classical,
 the question of whether the simplicial model could be defined working constructively (\ie without the use of the law of excluded middle or the axiom of choice) arose immediately after the definition of the simplicial
 model and was one of the
 central questions investigated during the thematic programme on Univalent Foundations at the Institute for Advanced Study in 2012/13. In spite of the significant efforts devoted to answering this question,
the issue is still open to date.

The aim of this paper is to provide a partial solution to this problem.  In order to
explain our results and the novel aspects of our work, let us recall from~\cite{voevodsky-simplicial-model} that the main results of  homotopy theory necessary 
to define the simplicial model of HoTT are 
the existence of a Quillen
model structure on the category of simplicial sets whose fibrations are the Kan fibrations, 
 the fact that dependent product along a fibration preserves fibrations 
 the existence of a fibration $\pi \co \UU \to \U$ that classifies small fibrations, 
the fibrancy of $\U$ and, finally, the univalence of the fibration $\pi$. These results
allow us to define an interpretation of type theory in simplicial sets by mapping type-theoretic contexts
into Kan complexes and dependent types as fibrations $p  \co A \to X$, where
$X$ is a Kan complex. This interpretation suffers from well-known coherence issues,
which can be solved in different ways~\cite{voevodsky-simplicial-model,LumsdaineP:locuoc}.



A fundamental obstruction to the constructive development of a version of the simplicial model was discovered by
Bezem, Coquand and Parmann in~\cite{coquand-non-constructivity-kan}, where it was shown that
it is not possible to prove constructively that for a simplicial set $A$ and a
Kan complex $B$, the exponential~$B^A$ is again a Kan complex.
Because of this, Coquand and his collaborators  defined homotopy-theoretic models of HoTT in categories of cubical sets~\cite{coquand-cubical-sets}, opening a very profitable new research direction, cf.~\cite{awodey-cubical,PittsAM:aximct} for example. Apart from switching 
from simplicial sets to cubical sets, they also switched from ordinary fibrations, defined by a right lifting property,
to uniform fibrations, defined as maps equipped with additional algebraic structure which provides
chosen fillers for lifting problems and is required to satisfy some naturality conditions, as in the
theory of algebraic weak factorisation systems~\cite{garner:small-object-argument}. While 
categories of cubical sets considered in this line of work allow us to define constructively models of HoTT 
and admit a Quillen
model structure~\cite{SattlerC:equepu}, none of them
 is known to be Quillen equivalent to simplicial sets or topological spaces.

As shown in~\cite{gambino2017frobenius}, it is possible to develop  a constructive theory of uniform fibrations
also in simplicial sets and prove that dependent product along a uniform fibration preserves
uniform fibrations and in particular that for a simplicial set $A$ and an 
 algebraic Kan complex $B$, the simplicial set $B^A$ admits the structure of an algebraic Kan complex. However,  uniform fibrations
are not as well-behaved in simplicial sets as in cubical sets,
since they do not admit a classifier, essentially because 
representables  in simplicial sets are not closed under under products while they are in cubical sets~\cite{SattlerC:faiaut}. 





A breakthrough has been obtained recently by the second-named author in~\cite{henry2019qms}, where, building on his previous work on weak model structures~\cite{henry2018wms},
it is shown constructively that the category of simplicial sets admits a Quillen model structure in which the
fibrations are the Kan fibrations. Moreover, this model structure is shown to be cartesian and proper.  Crucially for our goals here, 
the cofibrations of this model structure are not given by all the monomorphisms, but only by the monomorphisms $i \co A \to B$ that are levelwise complemented 
and for which the degeneracy of $n$-simplices is decidable in $B_n \setminus A_n$ for every $[n] \in \Delta$. In particular, the cofibrant objects are not all the simplicial sets, but only those with decidable degeneracies. However, this model structure coincides with
the standard one as soon as the law of excluded middle is assumed. Two other  
proofs of the  existence of this constructive
model structure are obtained in~\cite{GambinoN:anocp} without relying on the results in~\cite{henry2018wms}. 


The present paper extends this work to obtain constructive counterparts of all the other main results of 
homotopy theory necessary to define a constructive simplicial interpretation of HoTT. In particular, our main results are the following:
\begin{itemize}
\item \cref{thm:MainPathObject}, asserting that, for a fibration with cofibrant domain $p \co A \to X$, the mapping path space 
gives a factorisation of the diagonal $\delta_p \co A \to A \times_X A$ as a trivial cofibration followed by a fibration;
\item \cref{cor:Pi_types_are_fibrant}, asserting that dependent product along fibrations with cofibrant
domain preserves fibrations, which implies that, for a cofibrant simplicial set $A$ and a Kan complex $B$, the exponential $B^A$ is a Kan complex; 
\item \cref{thm:universe-uc}, asserting the existence of a small fibration $\pi_c \co
\UU_c \to \U_c$ that classifies small fibrations between cofibrant objects;
\item \cref{thm:fibrancy-of-u-and-uc}, asserting that the simplicial set $\U_c$ is a cofibrant Kan complex;
\item \cref{thm:univalence-of-u-and-uc}, asserting that the fibration $\pi_c  \co
\UU_c \to \U_c$ is univalent.
\end{itemize}



These results allow us to define a constructive interpretation of HoTT in the category of simplicial sets. 
In this new interpretation, type-theoretic
contexts are mapped into bifibrant objects, \ie cofibrant Kan complexes.   Dependent types are interpreted as fibrations $p \co A \to X$ where $X$ is a cofibrant Kan complex and $A$ is cofibrant. This choice is informed by the fact that, for a simplicial set~$X$, the slice category~$\SSet_{/X}$ admits a model structure in which fibrations and cofibrations are the maps that are fibrations and cofibrations in~$\SSet$, respectively. Then, the bifibrant objects in~$\SSet_{/X}$ are the fibrations $p \co A \to X$ with cofibrant domain. 
Our main results above are then intended to show how to interpret the rules for identity types, 
$\Sigma$-types, $\Pi$-types and a type theoretic universe satisfying the univalence axiom. 
For example, \cref{thm:MainPathObject} shows that identity types can be interpreted as 
mapping path spaces. The classical and the constructive interpretations of HoTT in simplicial sets are compared in~\cref{tab:interpretations}. 


\begin{table}[htb]
\begin{center}
\begin{tabular}{|c|c|c|}
\hline 
{\em Type theory} & {\em Classical simplicial interpretation} & {\em Constructive simplicial interpretation} \\ \hline 
 $X$ context  & $X$ Kan complex & $X$  cofibrant Kan complex \\[3ex]
 $X \vdash A \co \text{type}$ & \begin{tabular}{c} $p \co A \to X$ Kan fibration  \\ with $X$ 
Kan complex  \end{tabular}  & \begin{tabular}{c} $p \co A \to X$ Kan fibration with \\  $X$ cofibrant
Kan complex, $A$ cofibrant  \end{tabular} \\[3ex] 
  $X \vdash a \co A$ &  $\begin{gathered} \xymatrix{X \ar[rr]^a \ar[dr]_{1_X}  & & A  \ar[dl]^{p} \\
 & X & }\end{gathered}$ & $\begin{gathered} \xymatrix{X \ar[rr]^a \ar[dr]_{1_X}  & & A  \ar[dl]^{p} \\
 & X & }\end{gathered}$ \\ \hline
\end{tabular}
\end{center}
\medskip
\caption{Two simplicial interpretations of type theory.} 
\label{tab:interpretations}
\end{table}

The key novelty of our approach is the use of the homotopy-theoretic notion of cofibrancy
to handle the logical notion of decidability of degenacies. In particular, we use of the cofibrant  replacement functor
to interpret $\Pi$-types and the type-theoretic universe.  For $\Pi$-types, given a fibration $p \co A \to X$ with cofibrant domain, the result of applying the dependent product along $p$ to a fibration $q \co B \to A$ with cofibrant domain produces a map  $\Pi_p(q) \co \Pi_A(B) \to X$, which is a fibration by~\cref{cor:Pi_types_are_fibrant}, but whose domain
is not necessarily cofibrant. In order to remedy this, we define the interpretation of $\Pi$-types to be given by a cofibrant replacement of
$\Pi_p(q) \co \Pi_A(B) \to X$, which is now a fibration with cofibrant domain. 
Interestingly, this definition validates 
a judgemental $\beta$-rule and a propositional $\eta$-rule (see~\cref{rem:pi-types} for details), a combination that arises naturally when  Martin-L\"of type theory is presented
within the Logical Framework~\cite{nordstrom-petersson-smith:ml}, as well as in
versions of the Coq assistant, including the one used for the original development of  univalent foundations
library~\cite{voevodsky:library}. For example, the proof that the univalence axiom implies the principle of function extensionality was originally obtained assuming only a propositional $\eta$-rule for $\Pi$-types. See also~\cite{GarnerR:strdp}.

For the interpretation of a  type-theoretic universe satisfying the univalence axiom, we proceed
in two steps. First, we construct a small fibration $\pi \co \UU \to \U$ that classifies small fibrations with cofibrant fibers and prove that $\U$ is fibrant. 
Then, we consider a cofibrant replacement~$\U_c$ of~$\U$, which comes equipped with a trivial fibration $\varepsilon \co \U_c \to \U$, so that 
we obtain small fibration $\pi_c \co \UU_c \to \U_c$ with bifibrant codomain and cofibrant domain by pullback along $\varepsilon$. Our final result, \cref{thm:univalence-of-u-and-uc}, shows that
$\pi \co \UU \to \U$ and $\pi_c \co \UU_c \to \U_c$ are univalent fibrations. 



We refrain from claiming that our work provides a complete solution to the problem of defining a constructive simplicial model of type theory since our interpretation  does not validate strictly the type-theoretic
rules for substitution. These rules come in two groups:
one expressing associativity and unitality of substitution and another expressing commutation of substitution with pullbacks (corresponding to the Beck-Chevalley conditions). While 
there are well-known techniques to address these issues~\cite{voevodsky-simplicial-model,LumsdaineP:locuoc}, they do not seem to apply in our  context.

Thus, we only need an appropriate coherence or strictification result for the structure defined in this paper, 
which we leave the investigation of such a coherence result to future work.

Let us conclude these introductory remarks by mentioning that the proofs of our main results are inspired by arguments  in the existing literature, especially those in~\cite{voevodsky-simplicial-model}, but require a systematic
and careful analysis of the cofibrancy assumptions needed to carry over these proofs in a constructive
setting, so as to support the constructive simplicial interpretation of type theory. For example,
when discussing a universe, we are not interested in defining a small fibration that classifies small fibrations as in~\cite{voevodsky-simplicial-model}, but
rather a small fibration that classifies small fibrations between cofibrant objects. 




\noindent 
\textbf{Outline of the paper.} \cref{sec:preliminaries} recalls the constructive version of the 
Kan-Quillen model structure from~\cite{henry2019qms}. \cref{sec:basrp} establishes some basic results on pullbacks. \cref{sec:pats} shows how path spaces provide factorisations
as trivial cofibration followed by fibration. \cref{sec:Pi-types} proves the restricted Frobenius property. \cref{sec:equep}
establishes the weak equivalence extension property.
\cref{sec:unifbb} introduces a fibration $\pi_c \co \UU_c \to \U_c$ that classifies small fibrations
between cofibrant Kan complexes. \cref{sec:fibrancy-and-univalence} proves that $\U_c$ is bifibrant and that $\pi_c \co \UU_c \to \U_c$ is univalent, using the weak equivalence extension property.




\noindent
\textbf{Remarks on constructivity.} We work in  Constructive Zermelo-Fraenkel set theory (CZF),
a set theory based on intuitionistic logic~\cite{AczelP:typtic-I}. When discussing smallness of sets,
simplicial sets and fibrations, we assume the existence of a regular set $\mathsf{u}$, as defined 
in~\cite{AczelP:typtic-II}, and say that a set is small if it is an element of $\mathsf{u}$.
For further background on CZF, see~\cite{AczelP:notcst}. 
Throughout the paper, we adopt a slight abuse of language and say that a map has a right (or left) lifting property to mean that there exists a function mapping elements of the class of lifting problems to a chosen solution. 




\noindent 
\textbf{Acknowledgements.} Nicola Gambino gratefully acknowledges the support of
EPSRC under grant EP/M01729X/1 and the US Air Force Office for Scientific Research under 
agreement FA8655-13-1-3038 and the hospitality of the 
School of Mathematics of the University of Manchester and the Centre for Advanced Study in Oslo,
where the paper was written while on study leave. He is also grateful to Steve Awodey, Marc Bezem, Thierry Coquand, Andr\'e Joyal, Christian Sattler, Karol Szumi{\l}o  for useful conversations. 









\section{Preliminaries} 
\label{sec:preliminaries}




We write $\Delta$ for the simplicial category. The objects of $\Delta$ are written as $[n]$, for $n \geq 0$.
We write $\SSet \defeq [\Delta^{\op}, \Set]$ for the category of simplicial sets. For $n \geq 0$, $\Delta[n] \in \SSet$ is the representable simplicial set associated to $[n] \in \Delta$. Given a map $f \co Y \to X$ in $\SSet$, we write $f^* \co \SSet_{/X} \to \SSet_{/Y}$
for the associated pullback functor. The functor $f^*$ has a left adjoint, defined
by composition, which we write $\Sigma_{f} \co \SSet_{/Y} \to \SSet_{/X}$ and refer
to as the \emph{dependent sum} along $f$. Since $\SSet$ is locally cartesian closed, pullback
along $f$ has also a right adjoint, which we write 
$\Pi_f \co \SSet_{/Y} \to \SSet_{/X}$ and refer to as the \emph{dependent
product}  along $f$. The action of these functors on a
map $g \co Z \to Y$ will be written $\Sigma_f(g) \co \Sigma_Y(Z) \to X$ and 
$\Pi_f(g) \co \Pi_Y(Z) \to X$, respectively. Since  $\Sigma_f$ is defined by composition,~$\Sigma_Y(Z) \defeq Z$ and~$\Sigma_f(g) = g f $. 

As a special case of a well-known result for presheaf 
categories for every~$[n] \in \Delta$ there is an equivalence of categories
\begin{equation}
\label{equ:psh-slice-sset}
\SSet_{/\Delta[n]} \simeq  [ {\Delta_{/[n]}}^{\op}, \Set]   \, .
\end{equation}
For $F \co {\Delta_{/[n]}}^{\op} \to \Set$, we write $\pi_F \co \int F \to \Delta[n]$
for the corresponding object of~$\SSet_{/ \Delta[n]}$. Here, $\int F$ is the
simplicial set whose $m$-simplices are pairs $(\theta, x)$ where $\theta \co [m] \to [n]$
is a map in~$\Delta$ and $x \in F(\theta)$. The components of the natural
transformation $\pi_F$ are the first projections.




For $[n] \in \Delta$, we write $i^n \co  \partial \Delta[n] \to \Delta[n]$ for the boundary inclusion into the $n$-simplex and, for $1 \leq i \leq n$,  
$h^i_n  \co \Lambda^i[n] \to \Delta[n]$  for the $i$-th horn inclusion into the $n$-simplex. The simplicial set $\Delta[1]$ is an interval object in $\SSet$, with endpoint inclusions~$\kcyl \co \braces{ k} \to \Delta[1]$ defined by~$\kcyl \defeq h^1_k$. Throughout this paper, we shall work  with the constructive 
Kan-Quillen model structure $(\Weq, \Cof, \Fib)$ on $\SSet$ defined in~\cite{henry2019qms}. 
For the convenience of the reader, we recall the main aspects of this model structure below.
For this, let $\TrivCof \defeq \Weq \cap \Cof$ and $\TrivCof \defeq \Weq \cap \Fib$ be the classes of trivial cofibrations and trivial fibrations,
respectively. 

The weak factorisation system $(\Cof, \TrivFib)$ of cofibrations and trivial fibrations  is cofibrantly generated by the set $\cal{I} \defeq \{ i^n \co  \partial \Delta[n] \to \Delta[n] \ | \ n \geq 0 \}$  of boundary 
inclusions, \ie 
\[
\big( \Cof, \TrivFib \big) = ( \mathsf{Sat}(\cal{J}) \, , \cal{J}^\pitchfork) \, .
\]
As shown in \cite[proposition 5.1.4]{henry2018wms} a map $f \co Y \to X$ is a cofibration if  and only if 
it is a levelwise complemented monomorphism and the degeneracy of the simplices of $X_n ~\setminus~Y_n$ is decidable for every $[n] \in \Delta$. Thus, a simplicial set $X$ is
cofibrant if  degeneracy of the simplices of $X$ is decidable.
Note that a map between cofibrant objects is a cofibration
if and only if it is a levelwise complemented monomorphism. 
Cofibrant simplicial sets are of particular importance for our development because of their decidability property, which can be used to establish counterparts of classical results valid for all simplicial sets. An example is the Eilenberg-Zilber lemma~\cite[5.1.2]{henry2018wms}, asserting that in a cofibrant simplicial set~$X$, any cell $x \in X$ can be written uniquely as $p^*(y)$, where $y$ is a non-degenerate cell of $X$ and $p$ is a degeneracy. 
\medskip

The weak factorisation system $(\TrivCof, \Fib)$ of  trivial cofibrations and fibrations is cofibrantly 
generated by the set $\cal{J} \defeq \{ h^k_n  \co \Lambda^k[n] \to \Delta[n]  \ | \ 0 \leq k \leq n \}$ of horn 
inclusions, \ie 
\[
(\TrivCof, \Fib) = ( \mathsf{Sat}(\cal{J}) \, , \cal{J}^\pitchfork)
\] 
on $\SSet$. For a map $f \co Y \to X$, we denote the action of the pullback $f^*$ on a fibration~$p \co A \to X$ as
\[
\xymatrix{
A[f] \ar[r] \ar[d]_{p[f]} & A \ar[d]^{p} \\
Y \ar[r]_f & X \, .}
\]


For $X \in \SSet$, we write $\Fib_{/X}$ for the full subcategory of the slice category $\SSet_{/X}$ spanned by the fibrations with codomain $X$. For $X \in \SSet$, we write $\BFFib_{/X}$ for  the full subcategory of  $\Fib_{/X}$ spanned by fibrations with cofibrant domain. For a simplicial set $X$, we write $\mathbb{L}(X)$ for its cofibrant replacement and $\mathbb{R}(X)$ for its
fibrant replacement. These objects come equipped with a trivial fibration $\varepsilon_X \co \mathbb{L}(X) \to X$ and a trivial cofibration $\eta_X \co 
X \to \mathbb{R}(X)$, respectively.

\medskip

The model structure $(\Weq, \Cof, \Fib)$ is proper, \ie both left and right proper~\cite[proposition 2.2.9 and 3.5.2]{henry2019qms}
and the weak factorization systems $(\Cof, \TrivFib)$ and $(\TrivCof, \Fib)$
satisfy the so-called pushout product property~\cite[proposition 5.1.5 and corollary 5.2.3]{henry2018wms}. Recall that, given two maps $f \co Y \rightarrow X$ and $g \co B \rightarrow A$ their \emph{pushout product} $f \hattimes g$ is defined as the unique dotted map in the diagram
\[
\xymatrix{
Y \times B \ar[r] \ar[d] &  X \times B \ar[d] \ar@/^1.5pc/[ddr] \\
Y \times A \ar[r]  \ar@/_1.5pc/[drr] & \displaystyle \big( Y \times A ) +_{Y \times B} \big( X \times B ) \ar@{.>}[dr]  \\
 & & X \times A }
 \]
The pushout-product property is  the statement that  if $f$ and $g$ are cofibrations then so is $f \hattimes g$
and, if additionally either $f$ and $g$ is a weak equivalence, then so is $f \hattimes g$.
Note that when~$f$ and~$g$ are monomorphisms, so in particular when they are cofibrations, the pushout in the diagram above is just an union of subobjects and the pushout product of~$f$ and~$g$ is just the inclusion:
\[  
f \hattimes g \co (Y \times A) \cup (X \times B) \rightarrow X \times A \, .
\]
Dually, using  exponentials instead of products and pullbacks instead of pushouts, for maps $f \co Y \rightarrow X$ and~$p \co B \rightarrow A$, the \myemph{pullback exponential}  $\langle f \, , p \rangle$ is defined as the unique dotted arrow in the diagram:
\[
\xymatrix{
 B^X \ar@{.>}[dr] \ar@/^1.5pc/[drr] \ar@/_1.5pc/[ddr] \\
& B^Y \times_{A^Y} A^X \ar[r] \ar[d] &  A^X \ar[d]  \\
& B^Y \ar[r] & A^Y  \\
 }
 \]
By adjointness (see~\cite{joyal-tierney-segal} for details), a map $f$ has the left lifting property agains $\langle p \, , q \rangle $ if and only if $f \hattimes p$ has the left lifting property against $q$.
Therefore, the pushout-property  implies its dual version, asserting that  if $f \co Y \to X$ is a cofibration and $p \co B \to A$ a fibration then $\langle f \, , p \rangle$ is a fibration and if, additionally,  either $p$ or $f$ is a weak equivalence, then so so is $\langle f \, , p \rangle$.

\medskip

We also make use of another weak factorisation system $(\mathsf{L}, \mathsf{R})$ on $\SSet$ introduced in~\cite[section 3.1]{henry2019qms}. This is useful to establish decidability conditions; in particular, we will use it to prove~\cref{proposition:PathObjectCofibrant} and~\cref{Lemma:ForTheExtProperty} here. By definition, $(\mathsf{L}, \mathsf{R})$ is the
weak factorisation system cofibrantly generated by the set of degeneracy maps $\sigma \co \Delta[m] \to \Delta[n]$, \ie the maps induced by surjections $[m] \rightarrow [n]$ in $\Delta$. We refer to maps in $\mathsf{L}$ as the \emph{degeneracy quotients} and the
maps in $\mathsf{R}$ as the \emph{degeneracy-detecting} maps. Since degeneracy maps are (split) epimorphisms in $\SSet$, the weak factorization system $(\mathsf{L}, \mathsf{R})$ is actually a unique factorization system. Since the degeneracy-detecting maps are the maps with the (unique) right lifting property against degeneracy map,  they are exactly the simplicial morphisms $f \co X \rightarrow Y$ such that for $x \in X$, $f(x)$ is degenerated if and only if $x$ is degenerated. The degeneracy quotient maps are
instead the pushouts of coproducts of degeneracy maps, \ie the maps of the form $X \rightarrow X[(x_i,\sigma_i])$ where $x_i \in X_{n-1}$ are a family of cells,  $\sigma \co [n_i] \twoheadrightarrow [m_i]$ a family of degeneracies and $X[(x_i,\sigma_i)]$ is obtained from from $X$ by freely making $x_i$ in the image of $\sigma_i $ for each $i$. For the convenience of the reader, we recall
 the results  about this factorization system in \cite{henry2019qms} that will be used here.


\begin{lemma}
\label{lem:decidability_lift_degen_quo}
Let $p \co A \rightarrow B$ be a degeneracy quotient between finite decidable simplicial set, $X$  a cofibrant simplicial set and $f \co
A \rightarrow X$ be any morphism. Then it is decidable whether $f$ can be factored through $p$ or not. \qed
\end{lemma}

\begin{proof} See~\cite[Lemma~3.18]{henry2019qms}.
\end{proof} 


The proof of the result above can be outlined in a few words. Since $B = A[(a_i,\sigma_i)]$ for a finite collection of cells $a_i$,  $f$ factors through $p$ if and only if $f(a_i) =\sigma_i^*  x_i$ for each $i$, but when $X$ is cofibrant one can decide for each $i$ if this is the case or not. As there is only a finite number of such $i$, one can decide if it is the case for all $i$ or not.


\begin{proposition}
\label{prop:Degen_quotient_are_pullback_stable}
The class of degeneracy quotients is stable under pullback. 
\end{proposition}

\begin{proof} See~\cite[Proposition 3.1.11]{henry2019qms}.
\end{proof}

By~\cref{prop:Degen_quotient_are_pullback_stable}, if $f \co A \rightarrow B$ be a degeneracy quotient  then $f \times X \co A \times X \rightarrow B \times X$ is again a degeneracy quotient for every $X$.


\section{The comprehension category of cofibrant simplicial sets}
\label{sec:basrp}

The main goal of this section is to introduce the comprehension category of cofibrant simplicial sets. The
results of simplicial homotopy theory that we obtain in the following sections will show how this 
comprehension category supports various type-theoretic constructions. We refer to~\cite{JacobsB:catltt} for the definition of a comprehension category and basic results and to~\cite{LumsdaineP:locuoc} for the
definitions categorical counterparts of the type-theoretic constructs.  


\begin{lemma}\label{lem:cofibrant_fiber_product} \hfill 
\begin{enumerate}[$(i)$] 
\item Let $A \, , B$ be cofibrant simplicial sets. Then their product $A \times B$ is cofibrant.
\item Let $f \co A \to X$ and $g \co B \to X$ be maps with  cofibrant domain. Then their
fiber product $A \times_X B$, fitting in the pullback diagram
\[
\xymatrix{
A \times_X B \ar[r]^-{q} \ar[d]_-{p} & B \ar[d]^g \\
A \ar[r]_f & X \, ,}
\]
is cofibrant.
\end{enumerate}
\end{lemma}

\begin{proof} Part~(i) follows immediately from the fact that the model structure on simplicial sets is cartesian, i.e. satisfies the pushout-product conditions.

% let $x = (a,b) \in (A \times B)_n$, if $A$ and $B$ are cofibrant one can apply the Eilenberg-Zilber lemma and write $a = p^*(a')$ and $b = q^*(b')$ with $a'$ and $b'$ non-degenerate cells and $p \co [n] \to [m]$ and $q \co [n] \to [k]$ are two degeneracies. We claim that $(a,b)$ is degenerate if and only if the map $(a ,b )  \co [n] \rightarrow [m] \times [k]$ is not monic, which is a decidable condition since $(a,b)$ is a map between finite decidable sets.

% Indeed, the pair of map $(a,b)$ corresponding to a degenerate cell $x=r^*(x')$ for any non-trivial degeneracy $r \co [n] \to [n']$, is  $\big( p'(r),q'(r) \big)$ where $(p',q')$ are the maps corresponding to $x'$, and this is never monic as it factor into $r$, hence for a degenerate cell the condition above is always satisfied. Conversely, if $(p,q)\co [n] \rightarrow [m] \times [k]$ is not monic then as it is an increasing map there is some $i$ such that $p(i)=p(i+1)$ and $q(i)=q(i+1)$, hence $(p,q)$ can be factored into the $i$-th degeneracy map: $p=p' d_i$ and $q=q' d_i$ and in this case $x= d_i^* (p'^*(b'),q'^*(c'))$ is indeed a degenerate cell.
For part~(ii), $A \times_X B$ is a sub-simplicial set of $A \times B$ hence a cell of~$A \times_X B$ is degenerate if and only if it is degenerate as a cell of $A \times B$, hence degeneracy
in~$A \times_X B$ is indeed decidable. 
\end{proof}


The next proposition introduces the  comprehension category  cofibrant simplicial sets.
Recall that we write~$\SSet_{\mathsf{cof}}$ for the full subcategory of $\SSet$ spanned by
cofibrant simplicial sets. 


\begin{proposition} \label{thm:compcat}
The category $\SSet_{\mathsf{cof}}$ is the base of a comprehension category
of the form
\begin{equation*}
\begin{gathered}
\xymatrix{
\mathbf{Fib}_{ \mathsf{cof}} \ar[dr]_{p} \ar[rr]^{\chi} & & \SSet^{\to}_{\mathsf{cof}} \ar[dl]^{\mathrm{cod}} \\ 
 & \SSet_{\mathsf{cof}} &  }
 \end{gathered}
 \end{equation*}
 where $\mathbf{Fib}_{ \mathsf{cof}}$ is the full subcategory of $\SSet^{\to}_{\mathsf{cof}}$ spanned by fibrations.
\end{proposition} 

\begin{proof} Recall that the category $\SSet^\to_{\mathsf{cof}}$ is the arrow category of $\SSet_{\mathsf{cof}}$, whose objects are
maps $p \co A \to X$ between cofibrant objects. Then, the codomain
functor 
\[
\mathrm{cod} \co \SSet^{\to}_{\mathsf{cof}} \to \SSet_{\mathsf{cof}}
\] 
is a Grothendieck fibration by part~(ii) of~\cref{lem:cofibrant_fiber_product}. 
Since $\mathbf{Fib}_{ \mathsf{cof}}$ is the full subcategory of $\SSet^{\to}_{\mathsf{cof}}$ spanned by fibrations and the comprehension functor $\chi$ is the inclusion, $p = \mathrm{cod} \circ \chi$
and~$\chi$ preserves pullbacks.
\end{proof}



\begin{remark} \label{thm:sigma-types} It is immediate to see that the comprehension category of cofibrant simplicial sets supports $\Sigma$-types. since for a fibration $p \co A \to X$ between cofibrant objects, dependent sum along $p$  functor~$\Sigma_p \co \SSet_{/A} \to \SSet_{/X}$, which is defined by
composition with $p$, maps fibrations to fibrations.
\end{remark}

\medskip


We conclude this section with some auxiliary results on cofibrant objects and cofibrations
that will be useful in the following. 


\begin{proposition} \label{thm:cof-pbk}  Let $p \co A \to X$  be a map with cofibrant domain.
Then pullback along $p$, \[
p^* \co \SSet_{/X} \to \SSet_{/A} \, , 
\]
preserves cofibrations. 
\end{proposition}

\begin{proof} Let $f \co Y \to X$ be a cofibration and consider the pullback
\[
\xymatrix{
A[f]  \ar@{>->}[r] \ar@{>->}[d]_-{p[f]} &  A \ar[d]^{p} \\
Y \ar@{>->}[r]_{f} \ & X}
\]
We need to show that the monomorphism $A[f] \rightarrowtail A$ is a cofibration. 
Since $A$ is cofibrant, it suffices to show that it is a levelwise complemented. For $a \in A_n$, we have that $a \in A[f]_n$ if and only if $p(a) \in Y_n$. Since $f \co Y \rightarrowtail X$ is a levelwise complemented monomorphism, this is decidable.
\end{proof} 




\begin{proposition} \label{prop:X^kCofibrant}  \hfill 
\begin{enumerate}[$(i)$]
\item Let $X$ be cofibrant and $K$ finite and decidable. Then $X^K$ is cofibrant.
\item Let $f \co A \rightarrow B$ be a degeneracy quotient of finite decidable simplicial sets and $X$  cofibrant. Then $X^f \co X^B \rightarrow X^A$ is a cofibration between cofibrant objects.
\end{enumerate}
\end{proposition}

\begin{proof}  For part~(i), recall that an $n$-cell $\Delta[n] \rightarrow X^K$ is a morphism $\Delta[n] \times K \rightarrow X$. Now let $\sigma \co \Delta[n] \rightarrow \Delta[m]$ be a degeneracy map. Then the map 
$\sigma \times K \co \Delta[n] \times K \rightarrow \Delta[m] \times K$ is again degeneracy quotient.
Since it is a map between finite decidable simplicial sets, the question of whether a map $\Delta[n] \times K \rightarrow X$ factor through $\Delta[n] \times K \rightarrow \Delta[m] \times K$ is decidable
by~\cref{lem:decidability_lift_degen_quo}. But this is exactly the question of degeneracy of cells in $X^K$.
Similarly, for part (ii), since an $n$-cell of $X^A$ is a morphism $\Delta[n] \times A \rightarrow X$, it belongs to $X^B$ if and only it can be factored as $\Delta[n] \times A \rightarrow \Delta[n] \times B \rightarrow X$. This is decidable by \cref{lem:decidability_lift_degen_quo} because $\Delta[n] \times A \rightarrow \Delta[n] \times B$ is a degeneracy quotient between finite decidable simplicial sets by part (i).
\end{proof} 








\section{Identity types as path spaces}
\label{sec:pats}


In this section we begin to show how the the comprehension category of~\cref{thm:compcat} supports 
various type-theoretic constructs by considering identity types. Following the fundamental insight in~\cite{awodey-warren:homotopy-idtype}, in order to equip the comprehension category of~\cref{thm:compcat} with identity types it suffices to show that
for every fibration $p \co A \to X$ with cofibrant domain, there is a  fibration with cofibrant domain~$\partial \co \Id_A \to A \times_X A$ and a trivial cofibration $\mathsf{refl}_A \co A \to \Id_A$ that provide a factorisation of the diagonal map $\delta_p \co A \to A \times_X A$~\cite{LumsdaineP:locuoc}. While such a factorisation is guaranteed to exist by the model structure on $\SSet$, here we show that such a factorisation can be obtained by letting~$\Id_A$ be a suitable mapping path space. 
This is useful in order to support the intuition of elements of identity types as paths that is central to Homotopy Type Theory~\cite{hottbook}. 



\begin{lemma} \hfill 
 \label{thm:exponentials}
\begin{enumerate}[(i)] 
\item Let $X$ be cofibrant and $A$ be fibrant.  Then $A^X$ is fibrant.
\item Let $f \co Y \rightarrow X$ be a cofibration and $A$ be fibrant. Then $A^f \co A^X \rightarrow A^Y$ is a fibration.
\item Let $f \co Y \rightarrow X$ be a trivial cofibration and $A$ be fibrant.  Then $A^f \co A^X \rightarrow A^Y$ is a trivial fibration.
\item Let $X$ be cofibrant and $p \co B \rightarrow A$ be a (trivial) fibration. Then $p^X \co B^X \rightarrow A^X$ is also a (trivial) fibration.
\end{enumerate}
\end{lemma}

\begin{proof} The claims follow easily from the pushout product property of the model structure.
\end{proof}
 
Interestingly, the cofibrancy assumptions of part~(i) of~\cref{thm:exponentials} allows
us also to prove the claim constructively following the combinatorial proof in~\cite{MayJP:simoat}, exploiting the decidability of degeneracy in $X$ instead of appealing to the law of excluded middle.




For a simplicial set $A$, we define its path object by letting $\Path(A) \defeq A^{\Delta[1]}$. 
There are evident boundary map $\partial \co \Path(A) \to A \times A$, giving
the endpoints of a path. We write $\partial_0$, $\partial_1 \co \Path(A) \to A$ for the compositions of $\partial$
with the two projections and $r \co A \to \Path(A)$ for the `constant path' map.



\begin{proposition} \label{thm:id-types-for-types}
Assume that $A$ is fibrant. Then,
\begin{enumerate}[(i)] 
\item $\Path(A)$ is fibrant,
\item the boundary map $\partial  \co \Path(A) \rightarrow A \times A$ is a fibration,
\item the composite of $\partial \co \Path(A) \rightarrow A \times A$ with either projection is a trivial fibration,
\item the map $r \co A \rightarrow \Path(A)$ induced by the unique map $\Delta[1] \rightarrow \Delta[0]$ is a weak equivalence.
\end{enumerate}
\end{proposition} 

\begin{proof}
Part~(i) is just a special case of part~(i) of \cref{thm:exponentials}. For part~(ii), apply part~(ii) of \cref{thm:exponentials} to the cofibration $i^1 \co \partial \Delta[1]  \hookrightarrow \Delta[1]$. For part~(iii), apply part~(iii) of \cref{thm:exponentials} to the horn inclusions $h^k_n \co \Lambda^k[1]  \rightarrow \Delta[1]$. Part~(iv) follows from the 3-for-2 property for weak equivalences applied to $A \rightarrow A^{\Delta[1]} \rightarrow A$. Indeed, the
composite is the identity and the second factor has just been proved to be a trivial fibration.
\end{proof}


In order to interpret identity types, we need $\Path(A)$ to be cofibrant and the map $r \co A \rightarrow \Path(A)$ to be a trivial cofibration. This is achieved by the following proposition.

\begin{proposition}\label{proposition:PathObjectCofibrant}
Let $A$ be cofibrant. Then $\Path(A)$ is cofibrant and the map~$r \co A \rightarrow \Path(A)$ is a cofibration.
\end{proposition}



\begin{proof}
This follows from point $(iii)$ of proposition \ref{prop:X^kCofibrant}, applied to the cofibrant simplicial set $X$, and the degeneracy map between finite simplicial sets $\Delta[1] \rightarrow \Delta[0]$.


% Former Self-contained proof:  An $n$-cell of $\Path(A) = A^{\Delta[1]}$ is the same as a morphism $\Delta[1] \times \Delta[n] \rightarrow A$. It is classical simplicial combinatorics that $\Delta[1] \times \Delta[n]$ is a simplicial set generated by $(n+1)$ distinct non-degenerate $(n+1)$-dimensional cell $\alpha_i: \Delta[n+1] \rightarrow \Delta[1] \times \Delta[n]$ for $0 \leqslant i \leqslant n$ which are defined as (the nerve of) the map $\alpha_i$ from $\{0,\dots,(n+1)\}$ to $\{0,1\} \times \{0,\dots,n\}$ such that $\alpha_i(k)=(0,k)$ if $k \leqslant i$ and $\alpha_i(k)=(1,k-1)$ if $k>i$. They satisfies only the relation $d_i^* \alpha_i = d_{i}^* \alpha_{i-1}$ for each $1 \leqslant i \leqslant n$.

% So an $n$-cell $\sigma$ of $A^{\Delta[1]}$ is the same as collection of $n+1$-cells $\sigma(\alpha_i) \in A([n])$ for $i=0,\dots,n$ subject to the relations $d_i^* \sigma \alpha_i = d_{i}^* \sigma \alpha_{i-1}$ for $i=1,\dots,n$. It can then be checked that:
% \begin{itemize}
% \item $\sigma$ is ``degenerate at $j$'', \ie is of the form $s_j ^* \sigma'$ for $s_j : [n] \rightarrow [n-1]$ the map that repeat $j$ twice if and only if for each $i$, $\sigma \alpha_i $ is degenerate at $j+1$ for $i \leqslant j+1$ and at $j$ for $i \geqslant j$ (and at both if $i=j$ of $j+1$).
% \item $\sigma$ is in the image of $A \rightarrow A^{\Delta[1]}$ if and only if $\sigma(\alpha_i)$ is degenerate at $i$ for all $i$.
% \end{itemize}


% For example, for the second condition, if $\sigma$ is the image of an $n$-cell $t \in A[n]$ then $\sigma \alpha_i = s_i^* x$ for all $i$, and conversely, if $\sigma \alpha_i$ are all of the form $s_i^* x_i$ then the relation $d_i^* \sigma \alpha_i = d_{i}^* \sigma \alpha_{i-1}$ implies that $x_{i-1}=x_i$ for all $i$ and hence $\sigma$ is indeed the image of the cell $x=x_0= \dots = x_n$. The proof of the first claim is very similar.

% Now because $A$ is cofibrant for any given cell of $A$ one can decide whether it is degenerated or not, and by which degeneracy, so these the two question above are decidable as finite conjunction of decidable questions. As $\sigma$ is degenerated if and only if it is degenerated at $j$ for some $j$ this concludes the proof.
\end{proof}






We now define mapping path spaces and  extend~\cref{thm:id-types-for-types} and~\cref{proposition:PathObjectCofibrant}. Given a map $p \co A \to X$, we define $\Path(p)$ 
via the pullback diagram
\[
\xymatrix{
\Path(p) \ar[r] \ar[d] & \Path(A) \ar[d] \\
X \ar[r]_-{r_X} & \Path(X) }
\]
The structural maps $r_p \co A \rightarrow \Path(p)$ and $\partial \co \Path(p) \rightarrow A \times_{X} A$ are produced by the diagram:
\[
\xymatrix{
& A \ar[rr] \ar[dd] & & \Path(A) \ar[rr] \ar[dd] & & A \times A \ar[dd] \\
A \ar[ur] \ar[rr]^(.6){r}  \ar[dd] & & \Path(p) \ar[ur] \ar[rr]^(.65){\partial} \ar[dd] & & A \times_{X} A \ar[ur] \ar[dd] \\
& X \ar[rr] & & \Path(X) \ar[rr] & & X \times X \\
X \ar[ur] \ar[rr] & & X \ar[rr] \ar[ur] & & X \ar[ur] \\ 
}
\]
where the three square in the vertical/diagonal direction are pullbacks. As before, the maps
$\partial_0 \, , \partial_1 \co \Path(A) \to A$ are defined the composites of $\partial$
with the two projections. 


\begin{theorem}
\label{thm:MainPathObject}
Assume that $p \co A \rightarrow X $ is a fibration, with $A$ cofibrant and $X$  bifibrant. Then,
\begin{enumerate}[(i)] 
\item \label{thm:MainPathObject:IdBifib} $\Path(p)$ is bifibrant, 
\item the map $\Path(p) \rightarrow X$ is a fibration,
\item the map $\partial \co \Path(p) \rightarrow A \times_{X} A$ is a fibration,
\item $\partial_k \co \Path(p) \rightarrow A \times_{X} A$  is a trivial fibration, for $k \in \braces{0 \, , 1 }$,
\item the map $r_p \co A \rightarrow \Path(p)$ is a trivial cofibration.
\end{enumerate}
\end{theorem}

\begin{proof} The map $\Path(p) \rightarrow X$ is a pullback of the maps $\Path(A) \rightarrow \Path(X)$ along $X \rightarrow \Path(X)$. Hence, since $\Path(A) \rightarrow \Path(X)$ is a fibration (by part~(iv) 
of~\cref{thm:exponentials}), the map $\Path(p) \rightarrow X$ is a fibration, in particular $\Path(p)$ is fibrant. Since $X$ is cofibrant by assumption and $\Path(A)$ is  cofibrant by \cref{proposition:PathObjectCofibrant}, we have that $\Path(p)$ is also cofibrant by~\cref{lem:cofibrant_fiber_product}. 

By the dual of the pushout-product property, the map $\langle \partial \Delta[n] \hookrightarrow \Delta[n] ,  A \rightarrow X \rangle$ is a fibration. This map is 
\[ 
\Path(A) \rightarrow (A \times A) \times_{X \times X} \Path(X)
\] 
Moreover, in the diagram
\[
\xymatrix{
\Path(p) \ar[r] \ar[d] & A \times_{X} A \ar[d] \ar[r] & X \ar[d] \\
\Path(A) \ar[r] & \Path(X) \times_{X \times X}  (A \times A) \ar[r]  & \Path(X)
}
 \]
the right hand square is easily seen to be a pullback and the total rectangle is the pullback defining $\Path(p)$, hence the left hand square is also a pullback. Since the bottom left map is a fibration,  $\Path(p) \rightarrow A \times_{X} A$ is a fibration as well.

By a similar argument, for $k \in \braces{ 0 \, , 1}$, the map $\langle \Lambda^k[n] \hookrightarrow \Delta[n] \, , A \rightarrow X\rangle$ is a trivial fibration. Indeed, this is the map $\Path(A) \rightarrow  A  \times_X \Path(X)$ which fits into the pullback diagrams
\[
\xymatrix{
\Id_A \ar[r] \ar[d] & A  \ar[d] \ar[r] & X \ar[d] \\
\Path(A) \ar[r] & A \times_X \Path(X) \ar[r]  & \Path(X)
}
 \]
which  shows that the canonical maps $\partial_k \co \Path(p) \rightarrow A$ are trivial fibrations.

We conclude by showing that the map $A \rightarrow \Path(p)$ is levelwise complemented. Indeed, it fits into a factorization 
\[
A \rightarrow \Path(p) \rightarrow \Path(A)
\] 
of a map which has been proved to be a levelwise complemented inclusion in~\cref{proposition:PathObjectCofibrant} and therefore for any cell of $\Path(p)$ one can decide if it is in $A$ or not by considering it as a cell of $\Path(A)$. Since $A$ and $\Path(A)$ are cofibrant, this shows that $A \rightarrow \Path(p)$ is a cofibration. The 3-for-2 property for weak equivalences applied to $A \rightarrow \Path(p) \rightarrow A$ show that  $A \rightarrow \Path(p)$ is moreover a weak equivalence, hence a trivial cofibration.
\end{proof}




\section{$\Pi$-types via cofibrant replacements}
\label{sec:Pi-types}

The aim of this section is to prove the results of simplicial homotopy theory necessary to show that
the comprehension category of~\cref{thm:compcat} supports $\Pi$-types. 
In order to do this, we should consider a fibration with cofibrant domain $p \co A \to X$  and define an operation mapping a fibration with cofibrant domain $q \co B \to A$  to a new fibration with cofibrant domain~$\widetilde{\Pi}_p(q) \co \widetilde{\Pi}_A(B) \to X$. together with additional data~\cite{LumsdaineP:locuoc}. 
Given such a map $q \co B \to A$, we proceed
in two steps. First, we apply the dependent product along $p$, 
\[
\Pi_p \co \SSet_{/A} \to \SSet_{/X} \, , 
\]
to $q \co B \to A$ and obtain a map $\Pi_p(q) \co \Pi_A(B) \to X$, which we will show to be again a fibration. Since this fibration does not seem to have  cofibrant domain in general, we then apply a
cofibrant replacement in $\SSet_{/X}$ to $\Pi_p(q) \co \Pi_A(B) \to X$ so as to obtain a map with all the desired properties. Remarkably, the result will support a categorical counterpart of the propositional
$\eta$-rule.

Below, we show that, for a fibration $p \co A \rightarrow X$ with cofibrant domain, the dependent product along $p$ preserves fibrations. By adjointness, this is equivalent to showing that its left adjoint, \ie 
 pullback along $f$, preserves trivial cofibrations. This amounts to proving a restricted version of the Frobenius property~\cite{BergB:topsmi}, obtained by considering pullbacks along fibrations with
cofibrant domain rather than general fibrations. By the results in~\cite{coquand-non-constructivity-kan}  these cofibrancy assumptions are essential in our constructive setting.  


\begin{theorem}\label{cor:Pi_types_are_fibrant}
Let $p \co A \rightarrow X$ be a fibration with cofibrant domain,
\begin{enumerate}[$(i)$]
\item The pullback functor $p^* \co \SSet_{/X} \to \SSet_{/A}$ preserves trivial cofibrations.
\item The dependent product $\Pi_p \co \SSet_{/A} \to \SSet_{/X}$ preserves  fibrations.
\end{enumerate}
\end{theorem}

\begin{proof} Part~(i) follows from \cref{thm:cof-pbk} which show that it preserves cofibrations, together with the right properness of the simplicial model structure which shows that it preserve weak equivalences.
 Part~(ii) follows from part~(i) as $\Pi_p$ is right adjoint to $p^*$.
\end{proof}


It is also possible to give a direct proof of this claim based on the same strategy as in \cite{gambino2017frobenius} instead of using the right properness of the model structure on simplicial sets. See~\cite{GambinoN:anocp} for details.


 
\begin{remark}  \label{rem:pi-types}
We can define the interpretation of $\Pi$-types using~\cref{cor:Pi_types_are_fibrant} and a cofibrant replacement, as follows. 
Recall that, for maps $p \co A \to X$ and $q \co B \to A$,  the dependent product $\Pi_p(q) \co \Pi_A(B)
\to X$ is equipped with a map
\[
\mathsf{app} \co \Pi_A(B) \times_A  A \to B
\] 
in $\SSet_{/A}$ which is universal in the sense that, for every  $Y \to X$, the function
\[
\begin{array}{rcl} 
 \SSet_{/X}[ Y , \Pi_A(B)] & \longrightarrow &  \SSet_{/A}[Y \times_A A, B]  \\
  f & \longmapsto & \mathsf{app}(f \times_A 1_A) 
  \end{array} 
 \]
 is a bijection. This means that we have a function $\lambda$ in the opposite direction such that  
 \begin{equation}
 \label{equ:betaeta}
 \mathsf{app}(\lambda(b) \times_A 1_A) = b   \, , \quad
 \lambda( \mathsf{app}(f \times_A 1_A)) = f \, ,
 \end{equation}
 for every $b \co Y \times_A A \to B$ and $f \co Y \to \Pi_A(B)$.  These equations correspond to the
 well-known judgemental $\beta$-rule and $\eta$-rule for $\Pi$-types, respectively.
 
 When $p$ and $q$ are fibrations and $A$ is cofibrant, the map 
 $\Pi_p(q) \co \Pi_A(B) \to X$ is a fibration by \cref{cor:Pi_types_are_fibrant} but $\Pi_A(B)$ is not cofibrant
 in general. Thus, we interpret  $\Pi$-types as the 
 cofibrant replacement  of $\Pi_A(B)$, which is given by a cofibrant simplicial set
 $\mathbb{L}(\Pi_A(B)$  equipped with
 a trivial fibration $\varepsilon \co \mathbb{L}(\Pi_A(B)) \to \Pi_A(B)$. 
We then define $\widetilde{\mathsf{app}} \co   \mathbb{L}(\Pi_A(B)) \times_A A \to B$ by letting
\[
\widetilde{\mathsf{app}}  \defeq \mathsf{app} \circ (\varepsilon \times_A 1_A) \, .
\]
For a bifibrant simplicial set $Y$ and maps $Y \to X$,  $b \co Y \times_A A \to B$, we define $\widetilde{\lambda}(b) \co Y \to \mathbb{L}(B^A)$ to be the
diagonal filler
\[
\xymatrix{
0 \ar[r] \ar[d] & \mathbb{L}(\Pi_A(B))  \ar[d]^\varepsilon \\
Y \ar[r]_{\lambda(b)} \ar@{.>}[ur] & \Pi_A(B)}
\]
which exists since $Y$ is cofibrant and $\varepsilon$ is a trivial fibration. It follows immediately that
\[
 \widetilde{\mathsf{app}}(\widetilde{\lambda}(b) \times_A 1_A) = b \, ,
\]
so the $\beta$-rule holds as an equality. Instead, for $f \co X \to \mathbb{L}(\Pi_A(B))$, we have a homotopy
\[
\eta_f  \co \widetilde{\lambda}( \widetilde{\mathsf{app}}(f \times_A 1_A)) \sim  f  \, ,
\]
which is constructed as the diagonal filler in the following diagram
\[
\xymatrix@C=2cm{
\partial \Delta[1] \times Y \ar[r]^-{[f, \widetilde{\lambda}(f \times 1_A)]} \ar[d] & \mathbb{L}(\Pi_A(B)) \ar[d]^\varepsilon \\
\Delta[1] \times Y \ar[r] \ar@{.>}[ur] & \Pi_A(B) }
\]
where the bottom map is the constant homotopy given by the equality in the $\eta$-rule in~\eqref{equ:betaeta}.
\end{remark}



\section{The weak equivalence extension property}
\label{sec:equep}

The main goal of this section is to prove the so-called weak equivalence extension property, which will be the key to prove the univalence of the classifying fibrations considered in~\cref{sec:unifbb,sec:fibrancy-and-univalence}.  For this, we follow closely the approach in \cite{voevodsky-simplicial-model}, but exploiting crucially the cofibrancy requirements that are part of our set-up.



\begin{lemma}\label{Lemma:ForTheExtProperty} Let  $f \co Y \rightarrow X$ be a cofibration between
cofibrant objects. 
\begin{enumerate}[$(i)$]
\item The dependent product along $f$, $\Pi_f \co \SSet_{/Y} \rightarrow \SSet_{/X}$, preserves trivial fibrations.
\item The counit of the adjunction $f^* \dashv \Pi_f$ is a natural isomorphism.
\item If $q \co B \to Y$ is a map with cofibrant domain, then $\Pi_f(q) \co \Pi_Y(B) \to X$  is so.
\item Trivial fibrations extend along $f$, \ie given a trivial fibration $q
 \co B \to Y$  as in the solid diagram
\[
\xymatrix{
B \ar@{.>}[r]^{g} \ar[d]_{q} \drpullback  & A \ar@{.>}[d]^{p} \\
Y \ar[r]_-f &  X \, ,}
\]
then there exists a trivial fibration $p \co A \rightarrow X$ which fits in the dotted pullback square above. Moreover if $B$ is cofibrant  then 
$A$ can be chosen to be 
cofibrant as well.
\end{enumerate}
\end{lemma}

 
\begin{proof} We prove the different parts separately. 

For part~(i), since the dependent product functor $\Pi_f$ is right adjoint to the pullback functor~$f^*$ and trivial fibrations are the maps with the the right lifting property with respect to cofibrations, $\Pi_f$ preserves trivial fibrations if and only if $f^*$ preserves cofibrations. But this follows by~\cref{thm:cof-pbk}.

For part~(ii),  since $f$ is a monomorphism,  $\Sigma_f \co \SSet_{/X} \rightarrow \SSet_{/Y}$ is fully faithful and hence the unit $\eta \co 1 \rightarrow f^*  \Sigma_{f}$ is an isomorphism. By adjointness, the counit $\varepsilon \co f^* \Pi_f \rightarrow 1$ is also an isomorphism. 

For part (iii), Let $y \co  \Delta[n] \rightarrow \Pi_Y(B)$ be a  $k$-cell, we will show that for a given degeneracy $\sigma \co [n] \rightarrow [k]$ it is decidable if $y$ is ``$\sigma$-degenerated'', i.e. if $y$ factors through $\sigma \co \Delta[n] \rightarrow \Delta[k]$. As degeneracy is decidable in $Y$, one can freely assume that the image of $y$ in $Y$ is $\sigma$-degenerated (as if it is not the case, $y$ is not $\sigma$-degenerated) hence one has a solid diagram:

\[
\xymatrix{
\Delta[n] \ar[d]_{\sigma} \ar[r] & \Pi_Y(B) \ar[d] \\
\Delta[k] \ar@{.>}[ur]^? \ar[r] & X  \\
}
\]
Because of the adjunction between dependent product and the pullback along $f$, the existence of a lift as above is equivalent to the existence of a lift in:
\[
\xymatrix{
f^* \Delta[n] \ar[d]_{f^* \sigma} \ar[r]^x & B \ar[d] \\
f^* \Delta[k] \ar@{.>}[ur]^? \ar[r] & Y  \\
}
\]
The objects $f^* \Delta[n]$ and $f^* \Delta[k]$ are decidable simplicial subset of $\Delta[n]$ and $\Delta[k]$ because $f$ is itself a levelwise complemented monomorphism, hence they are both finite decidable simplicial sets. The map $f^* \sigma$ is degeneracy quotient by \cref{prop:Degen_quotient_are_pullback_stable}), hence \cref{lem:decidability_lift_degen_quo} shows that the existence of such a lift is decidable (as degeneracy quotients are epimorphisms, the existence of lift making the upper triangle commutes is equivalent to the existence of lift making the square commutes).


%    let $v$ be a $n$-cell in $\Pi_Y(A)$.
% If the image of $v$ is in $Y$ then $v$ is a cell of $A \subseteq \Pi_Y(A)$, in which case it is decidable whether $v$ is degenerate or not.
% Since $Y$ is levelwise complemented in $X$, one can assume that $v$ is not in the image of $Y$. In this case it is decidable if the image of $v$ in $X$ is degenerate or not.
% In fact, by the Eilenberg-Zilber lemma one can also decide for each given degeneracy if the image of $v$ is degenerate for this precise degeneracy or not. 

% Let $\sigma \co [n] \to [k]$ be any degeneracy, we will show that it is decidable whether $v$ is $\sigma$-degenerate, \ie if $v =\sigma^*(v')$ for some $v'$. Note that if $v$ is $\sigma$-degenerate then its image in $Y$ is as well. As this is a decidable question, one can freely assume that the image of $v$ in $Y$ is $\sigma$-degenerate, i.e. for the form $\sigma^* x$ for some $x \in X$ (and not in $Y$). One can form the pullback squares
% \[
% \xymatrix{
% V \ar@{>->}[r] \drpullback \ar[d] & \Delta[n] \ar@{->}[d]^{\sigma} \\
% V_{\sigma} \ar@{>->}[r] \drpullback \ar[d] & \Delta[k] \ar@{->}[d]^x \\
% Y \ar@{>->}[r]_{f} &  X }
% \]
% Given its image in $X$, the cell $v \co \Delta[n] \rightarrow \Pi_Y(A)$ is uniquely determined by the data of a morphism $\lambda \co V\rightarrow Y$. The cell $v$ is $\sigma$-degenerate if and only if $\lambda$ factors in $V_{\sigma}$ (such a factorization being unique if it exists). For any $J \subset [n]$, the $J$-face of a cell is said to be $\sigma$-degenerate if and only if it is degenerate for the (potentially trivial) degeneracy: 
% $\sigma_{|J} \co J \rightarrow \sigma(J)$. We claim that $\lambda$ factor into $V_{\sigma}$ if and only for all $i \co [f] \hookrightarrow [n]$ that belong to $V$ (and $V$ is decidable so there is only a finite cardinal of them), $i^* \lambda$ is $\sigma$-degenerate (which is dediable). Indeed $V$ is the gluing of all the $\sigma \circ i$ for such faces, for each individual face $i$ one has a factorization into its image in $V_{\sigma}$ if and only if $i^* \lambda$ is $\sigma$-degenerate, and as such factorization are unique they patch together on $V_{\sigma}$ is they all exists.

For part~(iv), given a trivial fibration $q \co B \to Y$, define $p \co A \to X$ to  be $\Pi_f(q) \co \Pi_Y(B) \to Y$. This map is a trivial fibration by part~(i) and the square is a pullback by part (ii). The final remark about the cofibrancy of $A$ follows from part~(iii). \qedhere
\end{proof}





\begin{proposition}[Weak equivalence extension property]
\label{Prop:Homotopy_ext_prop}
Let 
\[
\xymatrix{
B \ar[r]^g \ar[d]_q & A \ar[d]^p \\
Y \ar[r]_f & X}
\]
be a commutative diagram with $p \co A \to X$ and $q \co B \to Y$ be fibrations with cofibrant domains, $f \co Y \to X$ 
a cofibration and such that the map $u \co B \to A[f]$ defined by $u \defeq (q, g)$, fitting the diagram 
of solid maps
\[ 
\xymatrix{
 B
  \ar@{.>}[rr]
  \ar[dr]^{u}
  \ar[dd]_(.3){q}
&&
  \bar{B}
  \ar@{.>}[dr]^{v}
  \ar@{.>}[dd]_(.3){\bar{q}}|{\hole}
&\\&
  A[f] 
  \ar[rr]
  \ar[dl]
&&
  A
  \ar[dl]^{p}
\\
  Y
  \ar[rr]_{f}
&&
  X \, ,
&
}
\]
is a weak equivalence in $\SSet_{/ Y}$. Then there exist a fibration $\bar{q} \co \bar{B} \to X$, a weak equivalence $v \co \bar{B} \to A$ in $\SSet_{/X}$ and a map $B \to \bar{B}$ such that both squares in the diagram above are pullbacks. 
\end{proposition}

\begin{proof} We define the required object $\bar{B}$ as the following pullback:
\[\xymatrix{
\bar{B} \ar[d] \ar[r] \drpullback & \Pi_Y(B) \ar[d] \\
A \ar[r]_-{\eta_{A}} & \Pi_Y \big( A[f]  \big) \, ,
}\]
where $\eta_{A}$ is  a component of the unit of adjunction  $f^* \dashv \Pi_f$. An application of the pullback $f^* \co \SSet_{/X} \to \SSet_{/Y}$ to this square gives the commutative square
\[\xymatrix{
\bar{B}[f] \ar[d] \ar[r]  & B \ar[d] \\
A[f] \ar@{=}[r] &A[f] \, ,
}\]
which is a pullback since $f^* \Pi_f \iso 1$ by part~(ii) \cref{Lemma:ForTheExtProperty}. Hence 
$B \iso \bar{B}[f]$, as required.


Since $B$ is cofibrant, we have that $\Pi_Y(B)$ is cofibrant by part~(iii) of  \cref{Lemma:ForTheExtProperty}. Hence, the simplicial set~$\bar{B}$  is also cofibrant by \cref{lem:cofibrant_fiber_product}. Furthermore, the maps $B \rightarrow \bar{B}$ and~$A[f] \rightarrow A$ are cofibrations by~\cref{thm:cof-pbk}, as they are pullback of the cofibration~$f \co Y \rightarrow X$.





It remains to prove that $v \co \bar{B} \rightarrow A$ is a weak equivalence and that $\bar{q} \co \bar{B} \rightarrow X$ is a fibration. The map $u$ can be factored into a trivial cofibration followed by a trivial fibration, and it is sufficient to prove these claims for each half of the factorization separately, \ie when~$u$ is a trivial fibration and when it is a trivial cofibration.

If $u$ is a trivial fibration, then its image under $\Pi_f$ is a trivial fibration by 
part~(i) of \cref{Lemma:ForTheExtProperty}. Since the map $\bar{B} \rightarrow A$ is a pullback of this map,
it is also a trivial fibration. This also implies that the composite $\bar{B} \rightarrow A \rightarrow X$ is a fibration.


We now assume that $u \co B \rightarrow A[f]$ is a trivial cofibration. Using that the maps from $\bar{B}$ and $A[f]$ to $Y$ are fibrations, we can show that $u$ is a strong deformation retract over $Y$, \ie there is a retraction $r \co A[f] \rightarrow B$ of $u$ in $\SSet_{/Y}$ and a homotopy 
\[
H \co \Delta[1] \times A[f] \rightarrow A[f]
\] 
between 
$u \circ r$ and $1_{A[f]}$, whose composite with $A[f] \rightarrow Y$ is the trivial homotopy. Indeed, $r$ and  $H$ are respectively constructed as the dotted diagonal liftings in the squares:

\[\xymatrix{
B \ar[d] \ar@{=}[r]  & B \ar[d] \\
A[f] \ar[r] \ar@{.>}[ur]^r  & Y 
} \quad \xymatrix{
\displaystyle A[f] +_B \left( B \times \Delta[1] \right) +_B A[f] \ar[d] \ar[r]  & A[f] \ar[d] \\
A[f] \times \Delta[1] \ar[r] \ar@{.>}[ur]_H  & Y 
}\]

We want to show that $\bar{B} \rightarrow A$ is also a deformation retract by constructing a similar homotopy 
\[
H' \co \Delta[1] \times A \rightarrow A \, .
\] 
This homotopy will be constructed so that it is $H$ on $\Delta[1] \times A[f]$ ,  it is the map 
\[
\Delta[1] \times \bar{B} \rightarrow \Delta[0]  \times \bar{B} \iso \bar{B} \rightarrow A
\] 
on $\Delta[0] \times \bar{B} $ (indeed they agree on $\Delta[1] \times B$) and it is the identity on $\Delta[0] \times A$.  This is achieved by taking a diagonal filling in the square:
\[
\xymatrix@C=1.5cm{
\big( \Delta[1] \times (\bar{B} \cup A[f]) \big)  \cup \big( \Delta[0] \times A \big) \ar[d] \ar[r] & A \ar[d] \\
\Delta[1] \times A \ar[r] \ar@{.>}[ur]^{H'} & X
}\]
Such a diagonall filler exists since the map on the left-hand side is a trivial cofibration, being the 
 pushout-product of $Y_0 \co \Delta[0] \rightarrow \Delta[1]$ and the cofibration $\bar{B} \cup A[f] \rightarrow A$, and the map on the right-hand side is a fibration by assumption.

It remains to see that the map $H_{1} \co A \rightarrow A$ is indeed a retraction of $\bar{B} \rightarrow A$. We already know that the restriction of $H_{1}$ to $\bar{B}$ is  the inclusion of $\bar{B}$ in $A$, so it is enough to show that $H_{1}$ has values in $\bar{B}$. We also know that $H_{1}$ restricted to $A[f]$ takes values in $B \subseteq \bar{B}$. By definition of $\bar{B}$, the map $H_1$ factor into $\bar{B}$ if and only if it takes values in $\Pi_Y(B)$ when seen as a map to $\Pi_Y(A[f])$, and by adjunction this is the case if and only if the map corresponding to $H_1$, $A[f]= f^*(A) \rightarrow A[f]$ takes values in $B$, but already mentioned above that this was indeed the case.

Since $\bar{B} \rightarrow A$ is a deformation retract, it is invertible in the homotopy category and so it is a weak equivalence. The construction above also shows that~$\bar{B}$ is retract of $A$ in $\SSet_{/X}$ and hence $\bar{q} \co \bar{B} \rightarrow X$ is a fibration because $p \co A \rightarrow X$ is.
\end{proof}

 

\section{A  classifying small fibration between bifibrant objects}
\label{sec:unifbb}


The aim of this section is to begin establishing the results necessary to interpret a type-theoretic universe
\`a la Tarski satisfying the univalence axiom. \cref{tab:U-type} gives the basic rules for the type-theoretic universe,
while we refer to~\cite{voevodsky-simplicial-model} for the statement of the univalence axiom and the
relationship to its simplicial counterpart.

\begin{table}[htb]
\fbox{\begin{minipage}{8cm}  \smallskip
\[
\begin{gathered}
\U \co \type
\end{gathered}  \qquad  \qquad 
\begin{gathered}
\begin{prooftree}
\Gamma \vdash a \co \U
\justifies
\Gamma  \vdash \mathsf{T}(a) \co \type
\end{prooftree}
\end{gathered} 
\]
\end{minipage} }
\smallskip
\caption{Rules for the type-theoretic universe.} 
\label{tab:U-type}
\end{table}

In order to interpret these rules, we need to define a cofibrant Kan complex $\U_c$ and a Kan 
fibration with cofibrant domain $\pi_c \co \UU_c \to \U_c$. With these, the elimination rule of the
type theoretic universe will correspond to taking a pullback of $\pi_c$ along a map $a \co X
\to \U_c$ with cofibrant domain:
\[
\xymatrix{
\mathsf{T}(a) \ar[d] \ar[r] & \UU_c \ar[d]^{\pi_c} \\
X \ar[r]_a & \U_c}
\]
Furthermore, the closure of the type-theoretic universe under various type-formation operations
corresponds to the closure of the class of maps arising as pullbacks of $\pi_c$ under the 
operations necessary to intepret the corresponding types. For this, it is convenient to consider
$\pi_c$ to be a fibration that classifies small fibrations between cofibrant simplicial
sets, \ie such that for every such fibration $p \co A \to X$ there exists a map $a \co X \to \U_c$ fitting in a pullback diagram of the form
\[
\xymatrix{
A \ar[r] \ar[d]_p   & \UU_c \ar[d]^{\pi_c} \\
X \ar[r]_-a &  \U_c \, .}
\]
Here, note that we make no requirement for
the map $a$ to be unique, in contrast for example with the situation of the subobject classifier
in an elementary topos. Indeed, the map $a$ is not unique, but only unique  up to a contractible space
of choices, a fact that will be expressed by showing that $\pi_c$ is univalent. 

In this section, we limit ourselves to define $\pi_c \co \UU_c \to \U_c$ and prove that it
classifies small fibrations between cofibrant simplicial sets. We will then show 
that $\U_c$ is bifibrant, that $\UU_c$ is cofibrant  and that $\pi_c$ is univalent 
in~\cref{sec:fibrancy-and-univalence}. For the goal of this section, we proceed in two steps. First, we modify  the construction of the weak classifier for small fibrations in~\cite{voevodsky-simplicial-model} to obtain a small fibration $\pi \co \UU \to \U$ which classifies small fibrations with cofibrant fibers. Since the base of this fibration does not appear to be cofibrant,
we then consider a suitable cofibrant replacement  of~$\U$ and obtain the required fibration $\pi_c \co \UU_c \to \U_c$ via a pullback. 

\medskip

As a preliminary step, let us recall that a simplicial set $A$ is \emph{small}  if $A_n$ is a small set for every $[n] \in \Delta$ and that a map $p \co A \to X$ of simplicial sets is \emph{small} if for every $x \co \Delta[n] 
\to X$ the simplicial set~$A[x]$ given by the pullback square
\[
\xymatrix{
A[x] \ar[r] \ar[d] \drpullback & A \ar[d]^{p} \\
\Delta[n] \ar[r]_-{x} & X }
\]
is small. Let us also recall the  construction of a map of simplicial sets $\rho \co \VV \to \V$ that classifies
small maps of simplicial sets, which is a special case of the results in~\cite{hofmann-streicher-universes} for arbitrary presheaf categories.  For this, we use the equivalence in~\eqref{equ:psh-slice-sset} and the notation associated to it.
The simplicial set $\V$ is defined by letting
\[
\mathsf{V}_n \defeq \{ F \co {\Delta_{/[n]}}^{\op} \to \Set \ | \ \pi_F \co \textstyle{\int F} \to \Delta[n] \text{ is a small
map} \}
\]
for $[n] \in \Delta$. The object $\VV$ and the map $\rho \co \VV \to \V$ are then defined in an evident way. 


\bigskip

We now come to our first step, in which we define a small fibration $\pi \co \UU \to \U$ which 
classifies the class of small fibrations with cofibrant fibers. 

\begin{definition} We say that a map $p \co A \to X$ \emph{has cofibrant fibers} if for every $x \co \Delta[n] 
\to X$ the simplicial set~$A[x]$ given by the pullback square
\[
\xymatrix{
A[x] \ar[r] \ar[d] \drpullback & A \ar[d]^{p} \\
\Delta[n] \ar[r]_-{x} & X }
\]
is cofibrant.
\end{definition}


\begin{lemma} \label{lem:Cofib_fiber} \hfill 
\begin{enumerate}[$(i)$]
\item If a map $f \co A \rightarrow X$ has cofibrant domain then it has cofibrant fibers.
\item If $X$ is cofibrant and $f \co A \rightarrow X$ has cofibrant fibers then $A$ is cofibrant.
\end{enumerate}
\end{lemma} 

\begin{proof} Part~(i) follows from \cref{lem:cofibrant_fiber_product}. For part~(ii), let $[n] \in \Delta$, $a \in A_n$. Since $X$ is cofibrant, by the constructive version of the Eilenberg-Zilber lemma we can write $p(a) \in X$ in a unique way as $p(a) =s^*(x)$, where $s \co [n] \to [k]$ is a degeneracy and 
$x \in X_k$ is a non-degenerate cell. Let $x \co \Delta[k] \rightarrow X$ be the corresponding map. We now form the pullback
\[
\xymatrix{
A[x] \ar[r]^w \ar[d] \drpullback & A  \ar[d]^{p}  \\
\Delta[k] \ar[r]_{x} & X }
\]
By the universal property of the pullback, there is a unique cell $e \in A[x]_n$ such that $w(e)=a$, and the image of $e$ in $\Delta[k]$ is the cell $s \co [n] \to [k]$, whose image in $X$ are both equal to $p(a)=s^*(x)$.

By the assumption that $p$ has cofibrant fibers, the simplicial set $A[x]$ is cofibrant and hence it is decidable whether $e$ is degenerate or not. We claim that $a$ is degenerate if and only if $e$ is, which implies that it is decidable whether $a$ is degenerate.

 Indeed as $a = w(e)$ then if $e$ is degenerate so $a$ is. Conversely, assume that $a=\sigma^*(y_1)$ for a non-trivial degeneracy $\sigma$. Then $p(a)=\sigma^*(x_1)$, hence by the uniqueness part of the Eilenberg-Zilber lemma for $X$ one has that $s=s_1 \circ \sigma$ for some degeneracy $s_1$, and $x_1 = s_1^*(x')$. In particular, we get a unique cell $e_1$ of $A[x]$ whose image in $\Delta[n]$ and $X$ are  $s_1$ and $a_1$, respectively, whose images in $X$ are both equal to $x_1=s_1^*(x')$. Finally, the image of $p^*(e_1)$ in $\Delta[n]$ and~$A$ are  $p^* y_1 =a$ and $s_1 \circ p =s$, respectively, and hence $p^*(e_1) =e$, which proves that $e$ is degenerate as soon as $a$ is.
\end{proof} 


Define a subobject $\U \subseteq \V$ by letting, for $[n] \in \Delta$, 
\[
\U_n = \{ F \in \V_n \ | \ \pi_F \co \textstyle{\int F} \to \Delta[n] \text{ is a small fibration and 
$ \textstyle{\int F}$ is cofibrant} \} \, .
\]
We then define the map $\pi \co \UU \to \U$ via the pullback 
\begin{equation}
\label{equ:def-of-pi}
\begin{gathered}
\xymatrix{
\UU \ar[r] \ar[d]_\pi \drpullback  & \VV \ar[d]^\rho \\
\U \ar@{>->}[r] & \V }
\end{gathered}
\end{equation}




\begin{proposition} \label{thm:universe-u}  \hfill 
\begin{enumerate}[(i)] 
\item The map $\pi \co \UU \to \U$ is a small fibration with cofibrant fibers.
\item The map $\pi \co \UU \to \U$ classifies small fibrations with cofibrant fibers, \ie 
a map $p \co A \to X$ is a small fibrations with cofibrant fibers if and only if 
there is a map $a \co X \to \U$ and a pullback  of the form
\[
\xymatrix{
A \ar[r] \ar[d]_p \drpullback & \UU \ar[d]^\pi \\
X \ar[r]_{a} & \U }
\]
\end{enumerate}
\end{proposition}

\begin{proof} We prove the two claims separately.
For part~(i), consider a map $a \co \Delta[n] \rightarrow \U$ and the pullbacks
\[
\xymatrix{
A \ar[r] \ar[d]_p  & \UU \ar[d]^{\pi} \ar[r]  & \VV \ar[d]^{\rho} \\
\Delta[n] \ar[r]_{a} & \U \ar@{>->}[r] & \V }
\]
This shows that $p \co A \rightarrow \Delta[n]$ is isomorphic to $\pi_F \co \int F \rightarrow \Delta[n]$ in $\SSet_{/ \Delta[n]}$, where $F$ corresponds under the equivalence in~\eqref{equ:psh-slice-sset} to 
$a \co \Delta[n] \rightarrow \V$. Therefore, by definition of $\U$, $A$ is cofibrant and $p \co A \rightarrow \Delta[n]$ is a small fibration. This implies that $\pi \co \UU \rightarrow \U$ is has cofibrant fibers. To show it is a fibration, we rewrite a general lifting problem against a horn inclusion $h^n_k \co \Lambda^k[n] \rightarrow \Delta[n]$ as follows:
\[
\xymatrix{
\Lambda^k[n] \ar[r] \ar[d]_{h^k_n} & A  \ar[d]^{p} \ar[r] & \UU \ar[d]^\pi \\
\Delta[n] \ar@{=}[r]  & \Delta[n] \ar[r]_a & \U  }
\]
and then use that $p \co A  \to \Delta[n]$ is a fibration.

For part~(ii), if a map $p \co A \to X$ fits in a diagram as in the statement, then it is clearly a small fibration with
cofibrant fibers. Conversely, let $p \co A \to X$ be a small fibration with cofibrant fibers. Being a small 
map, $p$ fits into a pullback of the form 
\[
\xymatrix{
A \ar[r] \ar[d]_p  &  \VV \ar[d]^{\rho} \\
X \ar[r]_{a} & \V }
\]

For each cell of $X$,  $x \co \Delta[n] \rightarrow X$ the pullback $A[x] \rightarrow \Delta[n]$ is a small fibration with cofibrant domain by the assumptions on $p$, which means, by the definition of $\U$ that the image of the cell $x$ in $\V$ belongs to $\U$. Hence $X \rightarrow \V$ factors in $\U$, and hence $A \rightarrow X$ is actually the pullback of $\UU \rightarrow \U$. \qedhere
\end{proof} 



\bigskip

We now  construct a fibration $\pi_c \co \UU_c \to \U_c$ that classifies small fibrations between 
cofibrant objects. In particular, the base $\U_c$ of this fibration will be cofibrant.  In order to do this, let $\U_c$ be a cofibrant replacement of~$\U$,  which comes with a trivial fibration
\begin{equation}
\label{equ:ucu}
\tau \co \U_c \rightarrow \U \, .
\end{equation}
We then define $\UU_c$ via the pullback
\begin{equation}
\label{equ:def-of-pi-c}
\begin{gathered}
\xymatrix{
\UU_c \ar[d]_{\pi_c} \ar[r] \drpullback & \UU \ar[d]^{\pi}  \\
\U_c \ar[r]_\tau & \U}
\end{gathered}
\end{equation}
We now prove that $\pi_c \co \UU_c \to \U_c$ has the required properties.

\begin{theorem} \label{thm:universe-uc} 
\hfill 
\begin{enumerate}[(i)] 
\item $\pi_c \co \UU_c \to \U_c$ is a small fibration with cofibrant fibers and cofibrant codomain. 
\item The map $\pi_c \co \UU_c \to \U_c$ classifies small fibrations between cofibrant
objects, \ie if a map $p \co A \to X$, with $X$ cofibrant, is a small fibration between cofibrant objects
if and only there exists a pullback diagram of the form
\[
\xymatrix{
A \ar[r] \ar[d]_p & \UU_c \ar[d]^{\pi_c} \\
X \ar[r]_a & \U_c }
\]
\end{enumerate}
\end{theorem}

\begin{proof} For part~(i), $\U_c$ is cofibrant by construction and the rest of the claim follows from part~(ii) of~\cref{thm:universe-u}. 

%Part~(ii) follows from the cofibrancy of $X$ and $\UU_c$ by~\cref{lem:cofibrant_fiber_product}.

For part~(ii), let $p \co A \to X$ be a small fibration between cofibrant objects, by part~(i) of \cref{lem:Cofib_fiber}, it has cofibrant fibers and hence by part~(ii) of \cref{thm:universe-u} there is a pullback diagram of the form 
\[
\xymatrix{
A \ar[r] \ar[d]_p \drpullback & \UU \ar[d]^{\pi} \\
X \ar[r]_a & \U }
\]
Since $X$ is cofibrant, we have a diagonal filler in the diagram
\[
\xymatrix{
0 \ar[r] \ar[d] & \U_c \ar[d]^{\tau} \\
X \ar[r]_a \ar@{.>}[ur]^{a_c} & \U }
\]
 We then obtain the diagram
\[
\xymatrix{
A \ar[r] \ar[d]_p &  \UU_c \ar[r]  \ar[d]^{\pi_c}  & \UU \ar[d]^{\pi} \\
X \ar[r]_{a_c} & \U_c \ar[r]_{a} &  \U }
\]
Here, the right-hand side square and the rectangle are pullbacks and therefore the left-hand
side square is also a pullback. Hence $p$ is indeed a pullback of $\UU_c \rightarrow \U_c$. Conversely, any pullback  $p \co A \rightarrow X$ of $\UU_c \rightarrow \U_c$ is a small fibrations with cofibrant fibers, hence by part~(ii) of \ref{lem:Cofib_fiber}, if $X$ is cofibrant, $p$ is small fibration between cofibrant objects.
\end{proof} 

\section{Fibrancy and univalence of the universe}
\label{sec:fibrancy-and-univalence}


The aim of this section is to show that $\U_c$ is a cofibrant Kan complex and that $\pi_c \co
\UU_c \to \U_c$ is univalent.  For this, let us first return to consider the fibration $\pi \co \UU \to \U$ defined
via the pullback in~\eqref{equ:def-of-pi}. Let $\U^{\rightarrow}$ be the simplicial set whose $n$-simplices are triples of the form $(F_0, F_1, \phi)$, where $F_0$ and $F_1$ are $n$-simplices of $\U$, \ie functors
 \[
F_0, F_1 \co {\Delta_{/[n]}}^{\op} \rightarrow \Set
\]
and $\phi \co F_0 \Rightarrow F_1$ is a natural transformation. By the equivalence in~\eqref{equ:psh-slice-sset},
such triples correspond to commutative diagrams of the form
\begin{equation}
\label{equ:corresp}
\begin{gathered}
\xymatrix{ 
A_1 \ar[rr]^{f} \ar[dr]_{p_1} & & A_2 \ar[dl]^{p_2} \\
& \Delta[n] & }
\end{gathered}
\end{equation}
where $p_1$ and $p_2$ are fibrations with cofibrant domain.


\begin{lemma} $\U^{\rightarrow} \rightarrow \U \times \U$ is a fibration
\end{lemma}

\begin{proof} Observe that $\U^{\rightarrow}$ is exactly $\Pi_p(\UU \times \UU)$, 
where $p \co \UU \times \U \rightarrow \U \times \U$ is the evident map. 
It follows from \cref{cor:Pi_types_are_fibrant} that $\U^{\rightarrow} \rightarrow \U \times \U$ is a fibration.
More precisely, \cref{cor:Pi_types_are_fibrant} implies that any pullback of  $\U^{\rightarrow} \rightarrow \U \times \U$ to a cofibrant
$X \rightarrow \U \times \U$ is a fibration (due to the cofibrancy assumption of  \cref{cor:Pi_types_are_fibrant}),
but this is sufficient to prove that $\U^{\rightarrow} \rightarrow \U \times \U$ is a fibration,
as in the argument for part~(i) of  \cref{thm:universe-u}.
\end{proof}


We define $\mathsf{Weq}(\U)$ as the simplicial subset of $\U^\to$ whose $n$-simplices are the $n$-simplices $(F_1, F_2, \phi)$ of $\U^\to$ such that the corresponding map~$f$ in~\eqref{equ:corresp} is a weak equivalence.


In order for $\mathsf{Weq}(\U)$ to actually be a simplicial subset of $\U^{\to}$, one needs to check that given two fibrations with cofibrant domain $F_1 \co X_1 \rightarrow \Delta[n]$ and $F_2 \co X_2 \rightarrow \Delta[n]$ and a weak equivalence $\phi:F_1 \rightarrow F_2$ over $\Delta[n]$, the pullback of $\phi$ along any morphisms $\Delta[m] \rightarrow \Delta[n]$ is again an equivalence. This is the case because the pullback  $\SSet_{/\Delta[n]} \rightarrow \SSet_{/\Delta[m]}$ preserves fibrations and trivial fibrations, hence also preserves weak equivalences between fibrant objects (either by Ken Brown's lemma or because it is a right Quillen functor).



\begin{lemma}
\label{prop:Weq_classify_Weq}
For any cofibrant object $X$, a map $a \co X \rightarrow \U^{\rightarrow}$ factors via 
 $\mathsf{Weq}(\U)$ if and only the map in $\BFFib_{/X}$ classified by $a$,
\[
\xymatrix{
A_1 \ar[rr]^w  \ar[dr]_{p_1} & & A_2 \ar[dl]^{p_2} \\
 & X \, , & }
 \]
is a weak equivalence.
\end{lemma}

\begin{proof} By definition of $\mathsf{Weq}(\U)$, $a$ factors via $\mathsf{Weq}(\U)$ if and only if the pullback of $w \co A_1 \to A_2$ along any simplex $x \co \Delta[n] \rightarrow X$  is a weak equivalence. As observed above, pullbacks preserves equivalences between fibrations so this is indeed the case if $w$ is an equivalence.

Conversely,
we let $w \co A_1 \rightarrow A_2$ be a map between bifibrant objects of $\SSet_{/X}$, assume that the pullback of $w$ along every $x \co \Delta[n] \rightarrow X$ is a weak equivalence and show that $w$ is also a weak equivalence. One factors $w = p j$ as a trivial cofibration $j$ followed by a fibration $p$ and we will show that $p$ is a trivial fibration. For any $x \co \Delta[n] \rightarrow X$, the pullback of $x^* w$ is an equivalence by assumption and the pullback of $x^* j$ is an equivalence by the first part of the proof, hence by the 3-for-2 property for weak equivalences, the pullback of $x^* p$ is also an equivalence, and hence a trivial fibration.

To conclude, let $p : Y_1 \rightarrow Y_2$ be a fibration in $\SSet_{/X}$ such that for all cell $x: \Delta[n] \rightarrow X$ the morphism $x^* p \co \Delta[n]\times_X Y_1 \rightarrow \Delta[n] \times_X Y_2$ is a trivial fibration, we show that $p$ is a trivial fibration. Indeed, given any lifting problem of $\partial \Delta[n] \rightarrow \Delta[n]$ against $p$, one can consider the composite of $x: \Delta[n] \rightarrow Y_2 \rightarrow X$, take the pullback $x^* p$ of $p$ along this map, using that this pullback is a trivial fibration one obtains the lift:

\[\xymatrix{
\partial \Delta[n] \ar[d] \ar[r] & Y_1 \times_X \Delta[n] \drpullback \ar[r] \ar@{->>}[d] & Y_1 \ar@{->>}[d] \\
\Delta[n] \ar[r] \ar@{.>}[ur] \ar@{=}[dr] & Y_2 \times_X \Delta[n] \drpullback \ar[r] \ar[d] & Y_2 \ar[d] \\
& \Delta[n] \ar[r] & X
}
\]
which solves the given lifting problem.
\end{proof}


%Former proof: We do so using~\cite[\S 2.5.7]{henry2018wms}, and show that $w$ has the weak right lifting property against all $i_n \co \partial \Delta[n] \rightarrow \Delta[n]$. So let us consider the diagram
% \begin{equation}
% \label{equ:before-pullback}
% \begin{gathered}
% \xymatrix{\partial \Delta[n] \ar[rr] \ar[d] & & A_1 \ar[d]^w \\
% \Delta[n] \ar[rr] \ar[dr]_{x}
%  & & A_2 \ar[dl] \\
% & X \, .&}  
% \end{gathered}
% \end{equation}
% By pulling back everything to $\Delta[n]$ we obtain the diagram
% \begin{equation}
% \label{equ:after-pullback}
% \begin{gathered}
% \xymatrix{
% \partial \Delta[n] \ar[r] \ar[d] & A_1 \times_X \Delta[n] \ar[d]^{x^*(w)} \\
% \Delta[n] \ar[r] & A_2 \times_X \Delta[n]  \, .
% } 
% \end{gathered}
% \end{equation}
% By assumption $x^*(w)$ is a weak equivalence between fibrant objects hence it has the weak right lifting property against $i^n \co \partial \Delta[n] \rightarrow \Delta[n]$. A weak diagonal filler for~\eqref{equ:after-pullback} then gives  a weak diagonal filler for~\eqref{equ:before-pullback}, as required.






\begin{lemma} \label{thm:lemma-for-u} The map $\mathsf{Weq}(\U) \rightarrow \U^\to$ is a fibration.
\end{lemma}


\begin{proof}  Let $f \co Y \to X$ be a trivial cofibration between cofibrant objects and consider the diagram
\[ 
\xymatrix{ 
Y \ar[d]_f \ar[r]^-{w} & \mathsf{Weq}(\U) \ar[d] \\
X \ar[r]_-{a} & \U^{\rightarrow} }
\]
Such a lifting exists if and only if the map $\overline{a} \co A_1 \to A_2$ in $\SSet_{/X}$ classified by $a$ is a weak equivalence. Its pullback $f^*(\overline{a})$, fitting in the diagram:
\[ 
\xymatrix{ 
f^*(A_1) \ar[d]_{f^*(\overline{a})}  \ar[r]  & A_1 \ar[d]^{\overline{a}} \\
f^*(A_2) \ar[r] \ar[d]   & A_2 \ar[d] \\
Y \ar[r]_f & X }
\]
is a weak equivalence, because the map $Y \rightarrow \U^{\to}$ corresponding to it factors into $\mathsf{Weq}(\U)$.

Since the maps $p_i \co A_i \rightarrow X$ (for $i = 1, 2$) are fibrations with cofibrant domain, \cref{cor:Pi_types_are_fibrant} implies that pullbacks of trivial cofibrations between cofibrant objects along such 
a map are trivial cofibrations. This implies that all the horizontal maps of the diagram above are weak equivalence, and so 
$\bar{a}$ is as well. This shows that~$\mathsf{Weq}(\U) \rightarrow \U^{\rightarrow}$ is a fibration.
\end{proof}


\begin{theorem} \label{thm:fibrancy-of-u-and-uc} \hfill 
\begin{enumerate}[(i)] 
\item The simplicial set $\U$ is fibrant. 
\item The simplicial set $\U_c$ is bifibrant. 
\end{enumerate}
\end{theorem}

\begin{proof} We prove part~(i). Since
$(s, t) \co \mathsf{Weq}(\U) \rightarrow \U \times \U$ is a fibration, for any cofibrant 
simplicial set $X$,  maps $a_1 \, , a_2 \co X \rightrightarrows \U$ and homotopy $h \co \Delta[1] \times X \rightarrow \U$ from $a_1$ to $a_2$, there is a weak equivalence in $\SSet_{/X}$ between the objects classified by $a_1$ and $a_2$, constructed as follows. For this, we first consider a diagonal filler in the
diagram
\[
\xymatrix{ X \ar[r]^{i_1} \ar[d]_{\delta^0} & \mathsf{Weq}(\U) \ar@{->>}[d] \\
\Delta[1] \times X \ar[r]_{(a_1,h)} \ar@{.>}[ur] & \U \times \U
}
\]
Here, $i_1$ denotes a map classifying the identify of the object classified by $a_1$. By $a_1$ in the first component of the bottom arrow we mean the composite $\Delta[1] \times X \rightarrow X \rightarrow \U$. Composing the dotted arrow with $\delta^1$ gives us a map $X \rightarrow  \mathsf{Weq}(\U)$ whose projection to~$\U \times \U$ if $(a_1,a_2)$, \ie it classifies a weak equivalence between the objects classified by $a_1$ and $a_2$. One can do the same thing with $\delta^0$ and $\delta^1$ exchanged to get a weak equivalence in the other direction.

Using this fact, we can now prove that $\U$ is fibrant. A map $h^k_n \co \Lambda^k[n] \rightarrow \U $ classifies a fibration $q \co B \rightarrow \Lambda^k[n]$ with cofibrant domain. The horn inclusions $h^k_n \co \Lambda^k [n] \rightarrow \Delta[n]$ fits into retract diagrams:
\[
\xymatrix{\Lambda^k[n] \ar[d] \ar[r] & \big( \Delta[1] \times \Lambda^k[n] \big) \cup \Delta[n] \ar[d] \ar[r] & \Lambda^k[n] \ar[d]  \\
\Delta[n] \ar[r] & \Delta[1] \times \Delta[n] \ar[r] & \Delta[n] \, .
}\]

Where the map $\Delta[0] \rightarrow \Delta[1]$ can be either $\delta^0$ or $\delta^1$ depending on whether $0<k$ or $k<n$. See for example the last part of the proof of theorem 3.2.3 in \cite{joyal-tierney:simplicial-homotopy-theory}.

By the observation above, the composite map $\big( \Delta[1] \times \Lambda^k[n]  \big) \cup \Delta[n] \rightarrow \Lambda^k[n] \to \U$  gives a solid diagram of the form
\[ 
\xymatrix{
  B
  \ar@{.>}[rr]
  \ar[dr]
  \ar@{->>}[dd]
&&
  \bar{B}
  \ar@{.>}[dr]
  \ar@{.>>}[dd]|{\hole}
&\\&
  A'
  \ar[rr]
  \ar@{->>}[dl]
&&
  A
  \ar@{->>}[dl]
\\
  \Lambda^k[n]
  \ar[rr]_{h^k_n}
&&
  \Delta[n]
&
}
\] 
So we can construct a fibration  $\bar{q} \co \bar{B} \to \Delta[n]$ with cofibrant domain whose pullback 
along~$h^k_n$ is isomorphic to $q \co B \to \Lambda^k_n$. The map $b \co \Delta[n] \rightarrow \U$ classifying $\bar{q}$ gives the lift we are looking for. More precisely, we can use $q  \co \bar{B} \to
\Delta[n]$ to construct  a map $b \co \Delta[n] \rightarrow \U$ which extend the one we started from and classifies 
an object isomorphic to $\bar{B}$.

Part~(ii) follows immediately from part~(i) since $\tau \co \U_c \to \U$ is a trivial fibration.
\end{proof}




We now wish to define  what it means for a small fibration with cofibrant fibers, and in particular for a small fibration between cofibrant objects, to  be univalent. For this, we fix such a fibration $p \co A \to X$ and construct an object $\Weq(p) \to X \times X$ that represents weak
equivalences between fibers of $p$, in the sense that maps $Y \to \Weq(p)$ in $\SSet_{/X \times X}$ are in bijective correspondence with triples $(x_1, x_2, w)$ consisting of two map $x_1 \, , x_2 \co Y \to X$ and a weak equivalence $w \co A[x_1]
\to A[x_2]$ in $\SSet_{/Y}$. The required object $\Weq(p)$ can be constructed as the pullback
\[
\xymatrix{
\Weq(p) \ar[r] \ar[d] & \Weq(\U) \ar[d]^{(s,t)} \\
X \times X \ar[r]_{a \times a} & \U \times \U \, ,}
\]
where $a \co X \to \U$ is a classifying map for the small fibration $p \co A \to X$, which exists by our assumption
that $A$ and $X$ are cofibrant and part~(iii) of~\cref{thm:universe-u}. The verification that $\Weq(p)$
has the required universal property is an easy calculation, which we leave to the readers. There is an
evident map $i \co X \to \Weq(p)$ corresponding via the universal property of $\Weq(p)$ to the triple of identity maps $(1_X, 1_X, 1_A)$.




\begin{definition}  \label{equ:characterisations-of-univalence} Let $p \co A \to X$ be a small fibration with cofibrant fibers. We say that $p$ is \myemph{univalent} if the map $i \co X \to \Weq(p)$ is a weak equivalence. 
\end{definition}

\smallskip

For a small fibration with cofibrant fibers $p \co A \to X$, being univalent is equivalent to
either $s \co \Weq(p) \to X$ or $t \co \Weq(p) \to X$ being a trivial fibration.  Also note that, when $X$
is cofibrant, we have a map $j \co \Path(X) \to \Weq(p)$ fitting in the diagram
\[
\xymatrix@C=1.5cm{
X \ar[r]^i \ar[d]_r & \Weq(p) \ar[d]^{(s,t)} \\
\Path(X) \ar[r]_{\partial_X}  \ar@{.>}[ur]_{j} &  X \times X }
\]
In this case, $p$ is univalent if and only if $j$ is a weak equivalence, mirroring the type-theoretic
definition of univalence. This will be the case for $\pi_c \co \UU_c \to \U_c$, for example.





\begin{theorem}  \label{thm:univalence-of-u-and-uc} \hfill 
\begin{enumerate}[(i)]
\item The fibration $\pi \co \UU \to \U$ is univalent.
\item The fibration $\pi_c \co \UU_c \to \U_c$ is univalent.
\end{enumerate}
\end{theorem}

\begin{proof} For part (i), we prove that $t \co  \mathsf{Weq}(\U) \to \U$ has the right lifting property with respect
to all cofibrations. So let $f \co Y \rightarrow X$ be a cofibration and consider the diagonal
filling problem
\[
\xymatrix{Y \ar[d]_f \ar[r] & \mathsf{Weq}(\U) \ar[d] \\
X \ar[r] \ar@{.>}[ur]  & \U 
}
\]
By  \cref{prop:Weq_classify_Weq}, this corresponds exactly to a diagram as in the equivalence extension property as in \cref{Prop:Homotopy_ext_prop}. Indeed, the map $X \rightarrow \U$ gives us
$p \co A \to X$, the composite of~$Y \rightarrow  \mathsf{Weq}(\U)$ with the first projection
gives us $q \co B \to Y$, while the rest of the data and the commutativity of the square 
gives us a weak equivalence $u$ between $B$ and $A[f]$ over~$X$. The completion of this diagram given by \cref{Prop:Homotopy_ext_prop} is exactly what one needs to produce the required diagonal filler.

For part~(ii), we prove that $t \co \Weq(\pi_c) \to \U_c$ is a trivial fibration. For this, let us
observe that we have a diagram 
\[
\xymatrix@C=1.5cm{
\Weq(\U_c) \ar[r]^{\sigma}  \ar[d]_{(s,t)} & \Weq(\U) \ar[d]^{(s,t)} \\
\U_c \times \U_c \ar[r]^{\tau \times \tau} \ar[d]_{\pi_2} & \U \times \U \ar[d]^{\pi_2} \\
\U_c \ar[r]_\tau & \U }
\]
The composite  on the left-hand side is the map $t$ that we wish to show to be a trivial cofibration.
First, using part (i), observe that it is a fibration since it is the composite of two fibrations. Secondly,
observe that the top square is a pullback and so $\sigma$ is a trivial fibration since $\tau \times \tau$ 
is so. Thus, applying 3-for-2 to the outer square, we obtain that $t$ is a weak equivalence and hence
a trivial fibration, as required.
\end{proof} 


\section{Conclusion}
\label{sec:conclusion}

We can now sumarize our results in the following theorem:

\begin{theorem} \label{th:main_ContextualCat}

The comprehension category

\[ \xymatrix{
\mathbf{Fib}_{ \mathsf{cof}} \ar[dr]_{p} \ar[rr]^{\chi} & & \SSet^{\to}_{\mathsf{cof}} \ar[dl]^{\mathrm{cod}} \\ 
 & \SSet_{\mathsf{cof}} &  }
 \]


of ??? has, in the terminology of \cite{LumsdaineP:locuoc}:

\begin{itemize}

\item Pseudo-stable $\Sigma$-types.

\item Partially pseudo-stable $\Id$-types.

\item Weakly stable $\Pi$-types.

\item A type universe $\pi_c: \UU_c \rightarrow \U_c$ (weakly) à la Tarski, closed under the $\Sigma$-types, $\Id$-types and $\Pi$-types constructions.

\end{itemize}

Furthermore the fibration $\pi_c: \UU_c \rightarrow \U_c $ is univalent.

\end{theorem}


\begin{proof}

We have checked that this is indeed a comprehension category in ???. $\Sigma$-types have been constructed in ???. Identity types are constructed in section \ref{sec:pats}, more precisely see \cref{thm:MainPathObject} and above. The construction of $\Id$-types themselves invole only categorical dependants products and pullbacks, which are all pseudo-stable, but the construction of the $j$-elimination involves a lifting properties which might not be pseudo-stable, this corresponds to what is called a partially pseudo-stable identity types in \cite[definition 2.3.4]{LumsdaineP:locuoc}.
The $\Pi$-types have been constructed in section \ref{sec:Pi-types} as cofibrant replacement $\mathbb{L} \Pi_A B$ of the categorical dependant products $\Pi_A B$. The categorical dependant products are pseudo-stable, so given such a $\pi$-type $\mathbb{L} \Pi_A B$, in context $\Gamma$ its pullback $f^*( \mathbb{L} \Pi_A B)$ along a context morphism $f:\Gamma' \rightarrow \Gamma$ might be different from $\mathbb{L} \Pi_{f^* A} f^* B$ but is also a cofibrant replacement of $\Pi_{f^* A} f^* B$ hence also has the property of being a $\Pi$-type.

Finally, in section \ref{sec:unifbb} we constructed the universe $\UU_c \rightarrow \U_c$ as a classifier for all small fibrations between cofibrant objects. As small fibrations are stable under all the constructor above, this gives the stability of the universe. It has been shown in \ref{thm:univalence-of-u-and-uc} in that this fibration is indeed univalent.

\end{proof}



\begin{remark}
The $\Pi$-types constructed in the present paper have stronger properties than being weakly stable $\Pi$-types as stated above. First, has explained in remark \ref{rem:pi-types} they satisfies a propositional $\eta$-rule (while \cite{LumsdaineP:locuoc} require no $\eta$-rule at all). Moreover, given a $\Pi$-type $\mathbb{L} \Pi_A B$ in context $\Gamma$ in the sense of theorem \ref{th:main_ContextualCat}, and $f: \Gamma' \rightarrow \Gamma$ a morphism of context, then as observed in the proof above, both the pullback $f^*( \mathbb{L} \Pi_A B )$ and $\Pi$-type $\mathbb{L} \Pi_{f^* A} f^* B$ are both cofibrant replacement of the categorical dependent product $f^* \Pi_A B \simeq \Pi_{f^*A} f^* B $, hence they are homotopy equivalent. 


\end{remark}



\begin{remark}
The main result of \cite{LumsdaineP:locuoc} assert that if a contextual category $\mathcal{C}$, satisfying a certain additional condition called (LF), (see \cite[definition 3.1.3]{LumsdaineP:locuoc}), has such weakly stable type constructor, then its universal split replacement $\mathcal{C}_!$ has strictly stable similar type constructor. This solves the coherence problem, but, unfortunately, condition (LF) is $\emph{not}$ satisfied in our framework. It involves the existence of some dependent products in the contextual category, but as cofibrant objects are generally not stable under dependent products it does not seems possible to apply this techniques in our framework.  New ideas are required to solve the coherence problem in this case.
\end{remark}










\bibliographystyle{alpha}
\bibliography{../Auxiliary/bibliography}




\end{document}

% Type theory rules

\section{Type-theoretic rules}



\subsection*{Identity types}


\[
\begin{gathered}
\begin{prooftree}
\Gamma \vdash A \co \type \qquad
\Gamma \vdash a \co A \qquad
\Gamma \vdash b \co A 
\justifies
\Gamma \vdash \mathsf{Id}_A(a,b) \co \type
\end{prooftree} \\[2ex]
\begin{prooftree}
a \co A 
\justifies
\mathsf{refl}(a) \co \mathsf{Id}_A(a,a)
\end{prooftree} \\[2ex]
\begin{prooftree}
\Gamma \vdash p \co \Id_A(a,b) \qquad
\Gamma, x, y \co A, z \co \Id_A(x,y) \vdash B \co \type \qquad
\Gamma, x \co A \vdash d \co B[x/y, \mathsf{refl}(x)/z] 
\justifies
\Gamma \vdash \mathsf{J}(a,b,p,d) \co B[a/x, b/y, p/z]
\end{prooftree} \\[2ex]
\begin{prooftree}
\Gamma \vdash a \co A \qquad
\Gamma, x, y \co A, z \co \Id_A(x,y) \vdash B \co \type \qquad
\Gamma, x \co A \vdash d \co B[x/y, \mathsf{refl}(x)/z] 
\justifies
\Gamma \vdash \mathsf{J}(a,a,\mathsf{refl}(a),d) = d[a/x] \co B[a/x, a/y, \mathsf{refl}(a)/z]
\end{prooftree}
\end{gathered}
\]



\subsection*{$\Sigma$-types}

\[
\begin{gathered}
\begin{prooftree}
\Gamma \vdash A \co \type \qquad
\Gamma, x \co A \vdash B \co \type
\justifies
\Gamma \vdash (\Sigma x \co A) B 
\end{prooftree}  \\[1ex]
\begin{prooftree}
\Gamma \vdash a \co A  \qquad
\Gamma \vdash b \co B[a/x] 
\justifies
\Gamma \vdash \mathsf{pair}(a,b) \co (\Sigma x \co A) B 
\end{prooftree}  \\[2ex]
\begin{prooftree}
\Gamma \vdash c \co (\Sigma x \co A) B 
\justifies
\Gamma \vdash  \pi_1(c) \co A 
\end{prooftree} \qquad
\begin{prooftree}
\Gamma \vdash c \co (\Sigma x \co A) B 
\justifies
\Gamma \vdash \pi_2(c) \co B[ \pi_1(c)/x] 
\end{prooftree} \\[2ex]
\begin{prooftree}
\Gamma \vdash a \co A  \qquad
\Gamma \vdash b \co B[a/x] 
\justifies
\Gamma \vdash \pi_1 \big( \mathsf{pair}(a,b) \big)  = a \co A
\end{prooftree}  \qquad
\begin{prooftree}
\Gamma \vdash a \co A  \qquad
\Gamma \vdash b \co B[a/x] 
\justifies
\Gamma \vdash \pi_2\big( \mathsf{pair}(a,b) \big) = b \co B[a/x]
\end{prooftree} \\[2ex]
\begin{prooftree}
\Gamma \vdash c \co (\Sigma x \co A) B 
\justifies
\Gamma \vdash  c = \mathsf{pair}(\pi_1(c), \pi_2(c))  \co   (\Sigma x \co A) B 
\end{prooftree}
\end{gathered}
\]







\subsection*{$\Pi$-types}
 
\[ 
\begin{gathered} 
\begin{prooftree}
\Gamma \vdash A \co \type \qquad
\Gamma, x \co A \vdash B \co \type
\justifies
\Gamma \vdash (\Pi x \co A) B 
\end{prooftree}  \\[2ex]
\begin{prooftree}
\Gamma, x \co A \vdash b \co B
\justifies
\Gamma \vdash (\lambda x \co A) b \co (\Pi x \co A) B 
\end{prooftree} \\[2ex]
\begin{prooftree}
\Gamma \vdash f \co (\Pi x \co A) B \quad
\Gamma \vdash a \co A 
\justifies
\Gamma \vdash \mathsf{app}(f,a) \co B[a/x]
\end{prooftree} \\[2ex]
\begin{prooftree}
\Gamma, x \co A \vdash b \co B \qquad
\Gamma \vdash a \co A 
\justifies
\Gamma \vdash \mathsf{app}( (\lambda x \co A)b, a) = b[a/x] \co B[a/x] 
\end{prooftree} \\[2ex]
\begin{prooftree}
\Gamma \vdash f \co (\Pi x \co A) B 
\justifies
\Gamma \vdash \eta_f \co \Id_{(\Pi x \co A)B} \big( f, (\lambda x \co A) \mathsf{app}(f,x)\big)
\end{prooftree}
\end{gathered}
\]






\subsection*{Universe type}


\[
\U \co \type \qquad 
\begin{prooftree}
\Gamma \vdash a \co \U
\justifies
\Gamma \vdash \mathsf{T}(a) \co \type
\end{prooftree}
\]

\appendix
