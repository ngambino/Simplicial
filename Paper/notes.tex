\documentclass[reqno,10pt,a4paper,oneside,draft]{amsart}
\setcounter{tocdepth}{1}
\usepackage{../Auxiliary/prelude}
\input{../Auxiliary/prooftree}


\title[]{Constructive aspects of the simplicial model of univalent foundations}

\begin{document}

\begin{abstract}
We establish constructive counterparts of the main results of simplicial homotopy theory
used in the definition of Voevodsky's simplicial model of univalent foundations. In particular, 
 we
show constructively that the weak factorisation system of trivial cofibrations and Kan fibrations satisfies a
restricted version of the Frobenius property, that the model structure satisfies the weak equivalence extension property
for weak equivalences between fibrations with cofibrant domain and that there exists a univalent 
 fibration with bifibrant base that classifies small fibrations between bifibrant objects.
\end{abstract}

\author{Nicola Gambino}
\address{School of Mathematics, University of Leeds, Leeds LS2 9JT, United Kingdom}
\email{n.gambino@leeds.ac.uk}


\author{Simon Henry}
\address{Department of Mathematics and Statistics, Masaryk University, Brno, Czeck Republic}
\email{henrys@math.muni.cz} 


 \date{\today}
 
 

\maketitle



\section*{Introduction} 

The aim of this paper is to advance our understanding of the connections between type theory
and homotopy theory that have given rise to the areas of Homotopy Type Theory~\cite{hottbook}
and Univalent Foundations of Mathematics~\cite{voevodsky:library}. A cornerstone of these connections is Voevodsky's
discovery that Martin-L\"of type theory admits a model in the category of simplicial sets~\cite{voevodsky-simplicial-model}. The
study of this model suggested a new approach to the formalisation of mathematics in type
theory, following the idea that types have homotopy levels, and the formulation of the Univalence Axiom, 
which makes precise the idea that isomorphic structures
should be considered as equal.

Voevodsky's definition of the simplicial model of type theory was carried out working within 
classical Zermelo-Fraenkel
set theory extended with two inaccessible cardinals. Here, the first cardinal is used to interpret a type-theoretic universe, while the other is used to address coherence issues of the interepretation. 
Give that Martin-L\"of type theory is a constructive theory and its proof-theoretic strength is 
much lower than that of Zermelo-Fraenkel set theory, it is natural to ask whether the simplicial model could be defined working in a constructive setting, such as 
Martin-L\"of type theory itself or Constructive Zermelo-Fraenkel set theory~\cite{Aczel-Rathjen}. This
question has a direct application of mathematical logic, by allowing to prove the consistency of
the extension of Martin-L\"of type theory with the Univalence Axiom relative to Martin-L\"of type theory.

The aim of this paper is to provide a partial answer to this long-standing open problem.  In order to
explain our results, let us recall that the main results of simplicial homotopy theory necessary 
to define the simplicial model of Univalent Foundations are 
the existence of a Quillen
model structure whose fibrations are the Kan fibrations, which underpins a lot of the development, 
 the Frobenius property for the weak factorisation system of trivial cofibrations and fibrations, which
implies that pushforward along fibrations preserves fibrations, 
 the existence of a small fibration $\pi \co \UU \to \U$ that classifies small fibrations, 
the fibrancy of $\U$, and the univalence of $\pi$.

A serious obstruction to a constructive development of the simplicial model was discovered by
Bezem, Coquand and Parmann in~\cite{coquand-non-constructivity-kan}, where they showed that
the exponentiability of Kan complexes is not provable constructively.
Because of this, Coquand and several collaborators switched and considered the possibility of
defining homotopy-theoretic models of type theory in categories of cubical sets~\cite{coquand-cubical-sets}, starting a very profitable line of enquiry, cf.~\cite{awodey-cubical,PittsAM:aximct}. On top of switching 
from simplicial to cubical, they also switched from ordinary fibrations to uniform ones. While they
succeeded in interpreting constructively type theory with univalence in cubical sets, none of the
categories of cubical sets considered are known to be Quillen equivalent to topological spaces.

As shown in~\cite{gambino2017frobenius}, it is possible to develop a theory of uniform Kan fibrations
also in simplicial sets and prove the Frobenius property. However, the notion of a uniform Kan fibration
in simplicial sets is not as well-behaved as in cubical sets in that it does not admit a classifier, a fact that
ultimately follows from the representables are closed under products in cubical sets, while they are not
in simplicial sets. This situation, in which attempting to solve one problem immediately leads to problems with another problem, reminds closely of the game of `whack-a-mole', is summarised in~\cref{tab:whack}.



\begin{table}[htb]
\begin{tabular}{|c|c|c|c|}
\hline
& $\SSet$ &  $\mathbf{CSet}$ & $\SSet$    \\ 
& \cite{voevodsky-simplicial-model} & \cite{cohen-et-al:cubicaltt}  & \cite{gambino2017frobenius}   \\ \hline \hline 
Dependent type & Kan fibration & Uniform Kan fibration & Uniform Kan fibration    \\ 
$\mathsf{Id}$-types & \checkmark & \checkmark  &  \checkmark  \\
$\Pi$-types & \checkmark  & \checkmark  & \checkmark  \\
Universe type & \checkmark & \checkmark   &  $\times$  \\
Univalence axiom & \checkmark  & \checkmark &   $\times$  \\ 
Constructive & $\times$ & \checkmark   & \checkmark  \\ 
Equivalent to $\mathbf{Top}$ &  \checkmark & $\times$  & \checkmark    \\
Coherence & \checkmark   & \checkmark &  \checkmark  \\ 
\hline
\end{tabular}
\medskip
\caption{The univalent foundations whack-a-mole game.} 
\label{tab:whack} 
\end{table}



A breakthrough has been obtained in~\cite{henry2019qms}, in which the second-named author has
shown constructively that the category of simplicial sets admits a Quillen model structure in which the
fibrations are the Kan fibrations. Crucially, the cofibrations in this constructive model structure are not
all the monomorphisms, but rather the morphisms that are levelwise complemented with decidability of
degeneracies on the completement of the image. In particular, the cofibrant objects are not all simplicial
sets but only those with decidable degeneracies.




\newpage

\subsection*{Context} 
\begin{itemize}
\item The problem
\item Obstructions:
\begin{itemize}
\item Minimal fibrations
\item Exponentiability of Kan complexes
\end{itemize}
\item Progress so far:
\begin{itemize}
\item Coquand and others:  
\begin{itemize}
\item cubical sets
\item uniform fibrations
\item issues: de Morgan version is known not to be equivalent to $\mathbf{Top}$.
\end{itemize}
\item Gambino and Sattler
\begin{itemize}
\item General approach; uniform fibrations
\item Gets $\Pi$-types
\item Issue: if you work algebraically, notion of fibration does not support universe; if you work non-algebraically then assumptions are not known to be constructively valid.
\end{itemize}
\item Sattler: 
\begin{itemize}
\item gets a model structure
\item same issues as Gambino-Sattler. 
\end{itemize}
\end{itemize}
\end{itemize}


\subsection*{Main results}

\begin{itemize}
\item We build on [Henry 2018]
\item General approach
\item The crucial role of cofibrancy
\item We obtain 
\begin{itemize}
\item $\Pi$-types
\item Universe
\item Univalence
\end{itemize}
\item Coherence issues:
\begin{itemize}
\item We consider them seriously
\item But they are a separate issue, left for future work
\end{itemize}
\end{itemize}






We will interpret contexts as bifibrant simplicial sets. In particular,
the empty context will be interpreted as the terminal simplicial set $1$.  Dependent types 
will be interpreted as fibrations $p \co A \to X$ where $X$ is a bifibrant simplicial and $A$ is cofibrant. In particular, types in the empty context will be interpreted as bifibrant simplicial sets. 
This choice is motivated by the fact that, for a bifibrant simplicial set $X$, the slice category~$\SSet_{/X}$ admits a weak model structure in which fibrations and cofibrations are the maps that become fibrations and cofibrations, respectively, in~$\SSet$. Then, bifibrant objects in $\SSet_{/X}$
are exactly the fibrations with codomain $X$ and cofibrant domain.





This paper adds a new column to~\cref{tab:whack}.

\begin{table}[htb]
\begin{tabular}{|c|c|}
\hline
& $\SSet$ \\
 \hline \hline 
 Dependent type & Kan fibration with cofibrant domain \\
 $\mathsf{Id}$-types &   \checkmark  \\
$\Pi$-types  & \checkmark  \\
Universe type & \checkmark  \\
Univalence axiom & \checkmark  \\ 
Constructive & \checkmark  \\ 
Equivalent to $\mathbf{Top}$ & \checkmark  \\
Coherence  &  ? \\ 
\hline
\end{tabular}
\medskip
\caption{The contributions of this paper.} 
\end{table}


When working constructively, fibrations as defined above maintain many of their usual classical properties,
such as stability under pullbacks. However, other properties no longer hold. For example, it is not possible to prove constructively that
for a Kan complex $B$, the simplicial set $B^A$ is a Kan complex for every simplicial set $A$. In order
to remedy this in a constructive setting, one needs some additional decidabilty assumptions, which can be 
conveniently encapsulated in the notion of a cofibration and of a cofibrant object introduced in~\cite[\S 5.1.7]{henry2018wms}. 


 \medskip



\section{Preliminaries} 
\label{sec:preliminaries}




We write $\Delta$ for the simplicial category. The objects of $\Delta$ will be written as $[n]$, for $n \geq 0$.
We write $\SSet \defeq [\Delta^{\op}, \Set]$ for the category of simplicial sets. For $n \geq 0$, $\Delta[n] \in \SSet$ is the representable simplicial set associated to $[n] \in \Delta$. For $[n] \in \Delta$, we write $i^n \co  \partial \Delta[n] \to \Delta[n]$ for the boundary inclusion into the $n$-simplex and, for $1 \leq k \leq n$,  
$h^k_n  \co \Lambda^k[n] \to \Delta[n]$  for the $k$-th horn inclusion into the $n$-simplex. The simplicial set $\Delta[1]$ is an interval object in $\SSet$, with endpoint inclusions~$\kcyl \co \braces{ k} \to \Delta[1]$ defined by~$\kcyl \defeq h^1_k$. Throughout this paper, we shall work  with the constructive version of the
Kan-Quillen model structure on $\SSet$ defined in~\cite{henry2019qms} . 
For the convenience of the reader, we recall some of the main aspects of this model structure.


\medskip

Let $\cal{I} \defeq \{ i^n \co  \partial \Delta[n] \to \Delta[n] \ | \ n \geq 0 \}$ be the set of boundary 
inclusions.  By a constructive version of the small object argument, the set~$\cal{I}$ generates a weak factorisation system
\[
\big( \Cof, \TrivFib \big) \defeq ( \mathsf{Sat}(\cal{J}) \, , \cal{J}^\pitchfork)
\]
on $\SSet$. The maps in the left class will be called \myemph{cofibrations} and those in the right
class will be called \myemph{trivial fibrations}. 
As shown in \cite{henry2018wms} a map $f \co Y \to X$ is a cofibration if  and only if 
it is a levelwise complemented monomorphism and the degeneracy of the simplices of $X~\setminus~\mathsf{Im}(f)$ is decidable. A simplicial set $X$ will be said to be~\myemph{cofibrant} if the unique map $0 \to X$ is a cofibration, 
\ie degeneracy of the simplices of $X$ is decidable.
Note that a map between cofibrant objects is a cofibration
if and only if it is a levelwise complemented monomorphism. 
Cofibrant simplicial sets are of particular importance for our development because of their decidability property, which can be used to establish counterparts of classical results valid for all simplicial sets. An example is the Eilenberg-Zilber lemma~\cite{henry2018wms}, asserting that in a cofibrant simplicial set~$X$, any cell $x \in X$ can be written uniquely as $p^*(y)$, where $y$ is a non-degenerate cell of $X$ and $p$ is a degeneracy. Clearly, if one assumes the law of excluded middle, then every monomorphism is a cofibration and our notion of a cofibration coincides with the classical one. 

\medskip

Let $\cal{J} \defeq \{ h^k_n  \co \Lambda^k[n] \to \Delta[n]  \ | \ 0 \leq k \leq n \}$ be the set of horn 
inclusions. By another application of the constructive version of the small object argument, the set $\cal{J}$
generates a weak factorisation system 
\[
(\TrivCof, \Fib) = ( \mathsf{Sat}(\cal{J}) \, , \cal{J}^\pitchfork)
\] 
on $\SSet$. The maps in the right class will be called \emph{fibrations} and the maps in the left class will be called \emph{trivial cofibrations}. As usual, we say that a simplicial set~$A$ is \myemph{fibrant} if  \ie the unique map~$A \to 1$ is a fibration. For $X \in \SSet$, we write $\Fib_{/X}$ for the full subcategory of the slice category $\SSet_{/X}$ spanned by the fibrations with codomain $X$. We say that a simplicial set is \myemph{bifibrant} if it is both fibrant and cofibrant.  For $X \in \SSet$, we write $\BFFib_{/X}$ for  the full subcategory of  $\Fib_{/X}$ spanned by fibrations with cofibrant domain.

\medskip
For a simplicial set $X$, we write $\mathbb{L}(X)$ for its cofibrant replacement and $\mathbb{R}(X)$ for its
fibrant replacement (as given by Kan's $\mathrm{Ex}^\infty$ functor, for example). These objects come equipped with a trivial fibration $\varepsilon_X \co \mathbb{L}(X) \to X$ and a trivial cofibration $\eta_X \co 
X \to \mathbb{R}(X)$, respectively.

\medskip

As shown in~\cite{henry2019qms}, the two weak factorisation systems
$(\Cof, \TrivFib)$ and $(\TrivCof, \Fib)$ are part of a proper model structure on the category $\SSet$.
Furthermore,  the weak factorization systems $(\Cof, \TrivFib)$ and $(\TrivCof, \Fib)$
satisfy the so-called pushout product property~\cite{henry2018wms}. Recall that, given two maps $f \co Y \rightarrow X$ and $g \co B \rightarrow A$ their \emph{pushout product} $f \hattimes g$ is defined as the unique dotted map in the diagram
\[
\xymatrix{
Y \times B \ar[r] \ar[d] &  X \times B \ar[d] \ar@/^1.5pc/[ddr] \\
Y \times A \ar[r]  \ar@/_1.5pc/[drr] & \displaystyle \big( Y \times A ) +_{Y \times B} \big( X \times B ) \ar@{.>}[dr]  \\
 & & X \times A }
 \]
The pushout-product property is  the statement that:

\begin{itemize}
\item if $f$ and $g$ are cofibrations then so is $f \hattimes g$.
\item if additionally either $f$ and $g$ is a weak equivalence, then so is $f \hattimes g$.
\end{itemize}
Note that when $f$ and $g$ are monomorphisms, so in particular when they are cofibrations, the pushout in the diagram above is just an union of subobjects and the pushout product of~$f$ and~$g$ is just the inclusion:
\[  
f \hattimes g \co (Y \times A) \cup (X \times B) \rightarrow X \times A \, .
\]
Dually, using  exponentials instead of products and pullbacks instead of pushouts, for maps $f \co Y \rightarrow X$ and~$p \co B \rightarrow A$, the \myemph{pullback exponential}  $\langle f \, , p \rangle$ is defined as the unique dotted arrow in the diagram:
\[
\xymatrix{
 B^X \ar@{.>}[dr] \ar@/^1.5pc/[drr] \ar@/_1.5pc/[ddr] \\
& B^Y \times_{A^Y} A^X \ar[r] \ar[d] &  A^X \ar[d]  \\
& B^Y \ar[r] & A^Y  \\
 }
 \]
By adjointness (see~\cite{joyal-tierney-segal} for details), a map $f$ has the left lifting property agains $\langle p \, , q \rangle $ if and only if $f \hattimes p$ has the left lifting property against $q$:
\[
f \pitchfork \langle p \, , q \rangle \Leftrightarrow f \hattimes p \pitchfork q 
\]
Therefore, the pushout-property  implies its dual version: 
\begin{itemize}
\item if $f \co Y \to X$ is a cofibration and $p \co B \to A$ a fibration then $\langle f \, , p \rangle$ is a fibration.
\item if additionally either $p$ or $f$ is a weak equivalence, then so so is $\langle f \, , p \rangle$.
\end{itemize}

\medskip

Given a map $\sigma \co Y \to X$ in $\SSet$, to be thought of as a context morphism, we write 
\[
\sigma^* \co \SSet_{/X} \to \SSet_{/Y}
\] 
for the associated pullback functor and denote its action on a map $p \co A \to X$, to be thought of as a dependent type, as
\[
\xymatrix{
A[\sigma] \ar[r] \ar[d]_{p[\sigma]} & A \ar[d]^{p} \\
Y \ar[r]_\sigma & X \, .}
\]
Here, one can think of $A[\sigma]$ as the dependent type in context $Y$ obtained via the
substitutions in $\sigma$. Since $\SSet$ is locally cartesian closed, the pullback functor 
$\sigma^*$ has both a  left and a right adjoint, written $\sigma_{!} \co \SSet_{/Y} \to \SSet_{/X}$ and
$\sigma_* \co \SSet_{/Y} \to \SSet_{/X}$, respectively. 

\medskip

Given a fibration $p \co A \to X$, the pullback functor $p^* \co \SSet_{/X} \to \SSet_{/A}$ is
to be thought of as weakening, rather than substitution, operation. Because of this, we shall
introduce special notation both to denote its action and for its adjoints. 
The effect of $p^*$ on a fibration $p' \co A \to X$ is written
\[
\xymatrix{
A \times_X A'  \ar[r] \ar[d] & A' \ar[d]^{p'} \\ 
A \ar[r]_p  & X \, ,} 
\]
while the left and right adjoints of $p^* \co \SSet_{/X} \to \SSet_{/A}$ will be written
\[
\Sigma_p \co \SSet_{/A} \to \SSet_{/X} \,  ,  \Pi_p \co \SSet_{/A} \to \SSet_{/X} 
\]
respectively. Their action  on a
map $q \co B \to A$ will be written
\[
 \Sigma_p(q) \co \Sigma_A(B) \to X \, , \quad \Pi_p(q) \co \Pi_A(B) \to X \, ,
 \]
respectively. Since the left adjoint $\Sigma_p$ is defined by composition, $\Sigma_A(B) \defeq B$ and $\Sigma_p(q) = q p$.


 

\medskip

Finally, let us recall that, as a special case of a general result for presheaf categories, for every $[n] \in \Delta$ there is an equivalence of categories
\begin{equation}
\label{equ:psh-slice-sset}
\SSet_{/\Delta[n]} \simeq  [ {\Delta_{/[n]}}^{\op}, \Set]   \, .
\end{equation}
For $F \co {\Delta_{/[n]}}^{\op} \to \Set$, we write $\pi_F \co \int F \to \Delta[n]$
for the corresponding object of~$\SSet_{/ \Delta[n]}$. Here, $\int F$ is the
simplicial set whose $m$-simplices are pairs $(\theta, x)$ where $\theta \co [m] \to [n]$
is a map in~$\Delta$ and $x \in F(\theta)$. The component of $\pi_F$ at $[m]$
is simply the first projection.







\section{Basic results on pullbacks}


We begin our development with some elementary observations on pullbacks of cofibrant objects
and cofibrations. 




\begin{proposition}\label{lem:cofibrant_fiber_product} \hfill 
\begin{enumerate}[$(i)$] 
\item Let $A \, , B$ be cofibrant simplicial sets. Then their product $A \times B$ is cofibrant.
\item Let $f \co A \to X$ and $g \co B \to X$ be maps with $A$ and $B$ cofibrant. Then their
fiber product $A \times_X B$, fitting in the pullback diagram
\[
\xymatrix{
A \times_X B \ar[r]^-{q} \ar[d]_-{p} & B \ar[d]^g \\
A \ar[r]_f & X \, ,}
\]
is cofibrant.
\end{enumerate}
\end{proposition}

\begin{proof} For part~(i), let $x = (a,b) \in (A \times B)_n$, if $A$ and $B$ are cofibrant one can apply the Eilenberg-Zilber lemma and write $a = p^*(a')$ and $b = q^*(b')$ with $a'$ and $b'$ non-degenerate cells and $p \co [n] \to [m]$ and $q \co [n] \to [k]$ are two degeneracies. We claim that $(a,b)$ is degenerate if and only if the map $(a ,b )  \co [n] \rightarrow [m] \times [k]$ is not monic, which is a decidable condition since $(a,b)$ is a map between finite decidable sets.

Indeed, the pair of map $(a,b)$ corresponding to a degenerate cell $x=r^*(x')$ for any non-trivial degeneracy $r \co [n] \to [n']$, is  $\big( p'(r),q'(r) \big)$ where $(p',q')$ are the maps corresponding to $x'$, and this is never monic as it factor into $r$, hence for a degenerate cell the condition above is always satisfied. Conversely, if $(p,q)\co [n] \rightarrow [m] \times [k]$ is not monic then as it is an increasing map there is some $i$ such that $p(i)=p(i+1)$ and $q(i)=q(i+1)$, hence $(p,q)$ can be factored into the $i$-th degeneracy map: $p=p' d_i$ and $q=q' d_i$ and in this case $x= d_i^* (p'^*(b'),q'^*(c'))$ is indeed a degenerate cell.


For part~(ii), $A \times_X B$ is a sub-simplicial set of $A \times B$ hence a cell of~$A \times_X B$ is degenerate if and only if it is degenerate as a cell of $A \times B$, hence degeneratness 
in~$A \times_X B$ is indeed decidable. 
\end{proof}






We use pullback functors to intepret substitution of terms into types. Given a map $f \co \Gamma' \to \Gamma$, we will show that the pullback functor $f^* \co \SSet_{/\Gamma} \to \SSet_{/\Gamma'}$ restricts to a functor
\[
f^* \co \BFFib_{/\Gamma}  \to \BFFib_{/\Gamma'} \, .
\]





\begin{proposition} \label{thm:cof-pbk}  Let $p \co A \to X$  be a map with cofibrant domain and codomain.
Then the pullback functor $p^* \co \SSet_{/X} \to \SSet_{/A}$ preserves cofibrations. 
\end{proposition}

\begin{proof} Let $f \co Y \to X$ be a cofibration and consider the pullback
\[
\xymatrix{
B \ar[r]^-{q} \ar[d]_-{g} &  Y \ar[d]^{f} \\
A \ar[r]_{p} & X}
\]
Since $A$ is cofibrant, the claim is that $g \co B \rightarrow A$ is a levelwise complemented monomorphism. For $a \in A_n$, we have that $a \in B_n$ if and only if $p(a) \in Y_n$. Since $f \co Y \rightarrow X$ is a levelwise complemented monomorphism, this is decidable.
\end{proof} 


\begin{remark} 
We conclude this section by observing that it is straightforward to define a naive interpretation of
$\Sigma$-types so that the computation rule and the $\eta$-rule hold as judgemental equalities. 
Let  $p \co A \to X$ be a fibration with cofibrant domain. The
functor~$\Sigma_p \co \SSet_{/A} \to \SSet_{/X}$ defined by
composition with $p$ restricts to a functor
$\Sigma_p \co \BFFib_{/A}  \to \BFFib_{/X}$
 Since fibrations are defined by a right lifting property, they are closed
under composition. Therefore, if $q \co B \to A$ is a fibration with cofibrant domain, then
$\Sigma_p(q) \co \Sigma_A(B) \to X$ is again a fibration with cofibrant domain.
\end{remark} 



\section{Path spaces}

The existence of the weak factorisation system $(\TrivCof, \Fib)$ 
implies that for every fibration with cofibrant domain $p \co A \to X$ the diagonal map $\delta_p \co A \to A \times_X A$ admits a factorisation as a trivial cofibration  followed by a fibration. Here, we show that such a factorisation can be defined via path objects. 








\begin{lemma} \hfill 
 \label{thm:exponentials}
\begin{enumerate}[(i)] 
\item Let $X$ be cofibrant and $A$ be fibrant.  Then $A^X$ is fibrant.
\item Let $f \co Y \rightarrow X$ be a cofibration and $A$ be fibrant. Then $A^f \co A^X \rightarrow A^Y$ is a fibration.
\item Let $f \co Y \rightarrow X$ be a trivial cofibration and $A$ be fibrant.  Then $A^f \co A^X \rightarrow A^Y$ is a trivial fibration.
\item Let $X$ be cofibrant and $p \co B \rightarrow A$ be a (trivial) fibration. Then $p^X \co B^X \rightarrow A^X$ is also a trivial fibration.
\end{enumerate}
\end{lemma}

\begin{proof} The claims follow easily from the pushout product property of the model structure.
\end{proof}

Part~(i) of~\cref{thm:exponentials} can also be established directly, by inspecting the classical proofs, exploiting the decidability of degeneracy in $X$ instead of appealing to the law of excluded middle.



\medskip

For a simplicial set $A$, we define its path object by letting
\[
 \Path(A) \defeq A^{\Delta[1]} \, . 
\]
There are evident boundary maps $(\partial_0, \partial_1)  \co \Path(A) \to A \times A$, giving
the endpoints of a path.



\begin{proposition} \label{thm:id-types-for-types}
Assume that $A \in \SSet$ is fibrant. Then,
\begin{enumerate}[(i)] 
\item $\Path(A)$ is fibrant,
\item the boundary map $\partial  \co \Path(A) \rightarrow A \times A$ is a fibration.
\item the composite of $\partial \co \Path(A) \rightarrow A \times A$ with either projection is a trivial fibration,
\item the map $r \co A \rightarrow \Path(A)$ induced by the unique map $\Delta[1] \rightarrow \Delta[0]$ is a weak equivalence.
\end{enumerate}
\end{proposition} 

\begin{proof}
Part~(i) is just a special case of part~(i) of \cref{thm:exponentials}. For part~(ii), apply part~(ii) of \cref{thm:exponentials} to the cofibration $\partial \Delta[1]  \hookrightarrow \Delta[1]$. For part~(iii), apply part~(iii) of \cref{thm:exponentials} to the horn inclusions $\Lambda^k[1]  \rightarrow \Delta[1]$. Part~(iv) follows from the 3-for-2 property for weak equivalences applied to $A \rightarrow A^{\Delta[1]} \rightarrow A$. Indeed, the
composite is the identity and the second factor has just been proved to be a trivial fibration.
\end{proof}


In order to have a path object, we need $\Path(A)$ to be cofibrant and the map $r \co A \rightarrow \Path(A)$ to be a trivial cofibration. In a general setting, these properties are not to be expected
but in the special case of $\SSet$, this is achieved by the following proposition.

\begin{proposition}\label{proposition:PathObjectCofibrant}
Let $A$ be a cofibrant simplicial set. Then the simplicial set $\Path(A)$ is cofibrant and the map~$r \co A \rightarrow \Path(A)$ is a cofibration.
\end{proposition}



\begin{proof}
An $n$-cell of $\Path(A) = A^{\Delta[1]}$ is the same as a morphism $\Delta[1] \times \Delta[n] \rightarrow A$. It is classical simplicial combinatorics that $\Delta[1] \times \Delta[n]$ is a simplicial set generated by $(n+1)$ distinct non-degenerate $(n+1)$-dimensional cell $\alpha_i: \Delta[n+1] \rightarrow \Delta[1] \times \Delta[n]$ for $0 \leqslant i \leqslant n$ which are defined as (the nerve of) the map $\alpha_i$ from $\{0,\dots,(n+1)\}$ to $\{0,1\} \times \{0,\dots,n\}$ such that $\alpha_i(k)=(0,k)$ if $k \leqslant i$ and $\alpha_i(k)=(1,k-1)$ if $k>i$. They satisfies only the relation $d_i^* \alpha_i = d_{i}^* \alpha_{i-1}$ for each $1 \leqslant i \leqslant n$.

So an $n$-cell $\sigma$ of $A^{\Delta[1]}$ is the same as collection of $n+1$-cells $\sigma(\alpha_i) \in A([n])$ for $i=0,\dots,n$ subject to the relations $d_i^* \sigma \alpha_i = d_{i}^* \sigma \alpha_{i-1}$ for $i=1,\dots,n$.


It can then be checked that:

\begin{itemize}

\item $\sigma$ is ``degenerate at $j$'', i.e. is of the form $s_j ^* \sigma'$ for $s_j : [n] \rightarrow [n-1]$ the map that repeat $j$ twice if and only if for each $i$, $\sigma \alpha_i $ is degenerate at $j+1$ for $i \leqslant j+1$ and at $j$ for $i \geqslant j$ (and at both if $i=j$ of $j+1$).

\item $\sigma$ is in the image of $A \rightarrow A^{\Delta[1]}$ if and only if $\sigma(\alpha_i)$ is degenerate at $i$ for all $i$.

\end{itemize}


For example, for the second condition, if $\sigma$ is the image of an $n$-cell $t \in A[n]$ then $\sigma \alpha_i = s_i^* x$ for all $i$, and conversely, if $\sigma \alpha_i$ are all of the form $s_i^* x_i$ then the relation $d_i^* \sigma \alpha_i = d_{i}^* \sigma \alpha_{i-1}$ implies that $x_{i-1}=x_i$ for all $i$ and hence $\sigma$ is indeed the image of the cell $x=x_0= \dots = x_n$. The proof of the first claim is very similar.

Now because $A$ is cofibrant for any given cell of $A$ one can decide whether it is degenerated or not, and by which degeneracy, so these the two question above are decidable as finite conjunction of decidable questions. As $\sigma$ is degenerated if and only if it is degenerated at $j$ for some $j$ this concludes the proof.
\end{proof}


\begin{remark}
We expect the cofibrancy of $\Path(A)$ to extend to the case of $A^{X}$ when $X$ is any finite decidable simplicial set. But this is definitely not going to holds for infinite $X$ in general. For this reason, the definition of the interpretation of function types will involve a cofibrant replacement.
\end{remark} 


\bigskip


We now define mapping path objects and  extend~\cref{thm:id-types-for-types} and~\cref{proposition:PathObjectCofibrant}. Given a map $p \co A \to X$, we define $\Path(p)$ 
via the pullback diagram
\[
\xymatrix{
\Path(p) \ar[r] \ar[d] & \Path(A) \ar[d] \\
X \ar[r]_-{r_X} & \Path(X) }
\]
The structural maps $r_p \co A \rightarrow \Path(p)$ and $\partial \co \Path(p) \rightarrow A \times_{X} A$ are produced by the diagram:
\[
\xymatrix{
& A \ar[rr] \ar[dd] & & \Path(A) \ar[rr] \ar[dd] & & A \times A \ar[dd] \\
A \ar[ur] \ar[rr] \ar[dd] & & \Path(p) \ar[ur] \ar[rr] \ar[dd] & & A \times_{X} A \ar[ur] \ar[dd] \\
& X \ar[rr] & & \Path(X) \ar[rr] & & X \times X \\
X \ar[ur] \ar[rr] & & X \ar[rr] \ar[ur] & & X \ar[ur] \\ 
}
\]
where the three square in the vertical/diagonal direction are pullbacks.


\begin{proposition}
\label{proposition:MainPathObject}
Assume $p \co A \rightarrow X $ is a fibration between bifibrant objects. Then,
\begin{enumerate}[(i)] 
\item \label{proposition:MainPathObject:IdBifib} $\Path(p)$ is bifibrant, 
\item the map $\Path(p) \rightarrow X$ is a fibration
\item the map $\partial \co \Path(p) \rightarrow A \times_{X} A$ is a fibration,
\item the composite of $s \, , t \co \Path(p) \rightarrow A \times_{X} A$  with either of the two projections is a trivial fibration,
\item the map $r_p \co A \rightarrow \Path(p)$ is a trivial cofibration.
\end{enumerate}
\end{proposition}

\begin{proof}

The map $\Path(p) \rightarrow X$ is a pullback of the maps $\Path(A) \rightarrow \Path(X)$ along $X \rightarrow \Path(X)$. Hence as $\Path(A) \rightarrow \Path(X)$ is a fibration (by the last point of lemma \ref{thm:exponentials}), the map $\Path(p) \rightarrow X$ is a fibration, in particular $\Path(p)$ is fibrant. Since $X$ is cofibrant by assumption and $\Path(A)$ is  cofibrant by \cref{proposition:PathObjectCofibrant}, we have that $\Path(p)$ is also cofibrant by~\cref{lem:cofibrant_fiber_product}. 

Because of the dual of the pushout-product property, the map $\langle \partial \Delta[n] \hookrightarrow \Delta[n] ,  A \rightarrow X \rangle$ is a fibration. This map is 
\[ 
\Path(A) \rightarrow (A \times A) \times_{X \times X} \Path(X)
\] 
Moreover, in the diagram
\[
\xymatrix{
\Path(A) \ar[r] \ar[d] & A \times_{X} A \ar[d] \ar[r] & X \ar[d] \\
\Path(p) \ar[r] & \Path(X) \times_{X \times X}  (A \times A) \ar[r]  & \Path(X)
}
 \]
the right hand square is easily seen to be a pullback and the total rectangle is the pullback defining $\Path(p)$, hence the left hand square is also a pullback. As we just showed that the bottom left map is a fibration, this implies that $\Path(p) \rightarrow A \times_{X} A$ is a fibration as well.

A very similar argument gives that for $k=0 \, , 1$, the map $\langle \Lambda^k[n] \hookrightarrow \Delta[n] ,  A \rightarrow X\rangle$ is a trivial fibration. Indeed, this is the map
\[ 
\Path(A) \rightarrow  A  \times_X \Path(X)
\] 
which fits into the pullback diagrams
\[
\xymatrix{
\Id_A \ar[r] \ar[d] & A  \ar[d] \ar[r] & X \ar[d] \\
\Path(A) \ar[r] & A \times_X \Path(X) \ar[r]  & \Path(X)
}
 \]
which  shows that any of the two (dependings on whether $k=0$ or $k=1$) canonical maps $\Path(p) \rightarrow A$ is a trivial fibration.

The map $A \rightarrow \Path(p)$ is levelwise complemented. Indeed, it fits into a factorization 
\[
A \rightarrow \Path(p) \rightarrow \Path(A)
\] 
of a map which has been proved to be a levelwise complemented inclusion in~\cref{proposition:PathObjectCofibrant} and therefore for any cell of $\Path(p)$ one can decide if it is in $A$ or not by considering it as a cell of $\Path(A)$. Since $A$ and $\Path(A)$ are cofibrant, this shows that $A \rightarrow \Path(p)$ is a cofibration. The 3-for-2 property applied to $A \rightarrow \Path(p) \rightarrow A$ show that  $A \rightarrow \Path(p)$ is moreover a weak equivalence, hence a trivial cofibration.
\end{proof}


\section{The restricted Frobenius property}
\label{sec:Pi-types}



We wish to show that the weak factorisation system of fibrations and trivial cofibrations satisfies
a restricted version of the Frobenius property, asserting that pullback along fibrations with cofibrant
domains preserves trivial cofibrations. 
As is well known, by adjointness this implies that, for a fibration $p \co A \rightarrow X$ with cofibrant
domain, the dependent product functor $\Pi_p \co \SSet_{/A} \to \SSet_{/X}$ preserves fibrations. The proof of the restricted Frobenius property follows essentially~\cite{gambino2017frobenius}. In order to combine our development with that argument, however, we need the following preliminary proposition, in which we use the pushout product $\kcyl \hattimes f$ of an endpoint inclusion $\kcyl \co \braces{ k } \to \Delta[1]$ and a map~$f \co Y \to X$, which is defined as the unique dotted arrow  in the diagram:
\[
\xymatrix{
\braces{ k } \times Y \ar[r] \ar[d] &  \Delta[1]  \times X  \ar[d] \ar@/^1.5pc/[ddr] \\
\braces{ k} \times X \ar[r]  \ar@/_1.5pc/[drr] & \big( \Delta[1] \times Y ) \cup X \ar@{.>}[dr]  \\
 & &\Delta[1] \times  X \, .}
 \]



\begin{proposition} For a map $p \co B \to A$, the following conditions are equivalent.
\begin{enumerate}[(i)] 
\item The map $p$ is a fibration.
\item The map $p$ has the right lifting property with respect to the pushout products $ \kcyl  \hattimes f$, for every cofibration $i \co Y \to X$.
\item The map $p$ has the right lifting property with respect to the pushout products $ \kcyl  \hattimes h^n_k$, for every  
horn inclusion $h^n_k$. \noten{Need to fix indices to avoid clash of use of $k$}.
\end{enumerate}
\end{proposition} 

\begin{proof}The proof of \cite[Theorem~3.2.3]{joyal-tierney:simplicial-homotopy-theory} is completely constructive and sufficient to imply this result.
For more details, readers may also refer to the last claim in \cite[Corollary~5.3.2]{henry2018wms}
and \cite[Proposition~5.2.6]{henry2018wms}.
\end{proof}



Let $p \co A \rightarrow X$ be a fibration. By adjointness, $\Pi_p \co \SSet_{/A} \to \SSet_{/X}$  preserves fibrations if and only if the pullback functor $p^* \co \SSet_{/X} \to \SSet_{/A}$ preserves trivial cofibrations. Let us now assume that  $A$ is cofibrant. 
In this case $p^*$ preserves cofibrations by \cref{thm:cof-pbk}, so we only need to show that it preserves
trivial cofibrations. We achieve this by following closely \cite[Section~3]{gambino2017frobenius}. Note that we cannot apply directly the result therein since 
the assumption that every object is cofibrant does not hold in our setting. However, only minor modifications are sufficient.


% \begin{lemma} \label{thm:missing-1}
% \hfill 
% \begin{enumerate}[(i)] 
% \item $\mathcal{J} \subset \Cof \cap \mathcal{S}$.
% \item $\Cof \cap \mathcal{S} \subseteq \TrivCof$.
% \end{enumerate}
% \end{lemma} 


\begin{definition} \label{def:strhtpyequiv} Let $k \in \braces{0 \, , 1 }$.
A map $f \co Y \rightarrow X$ in $\SSet$ is a \myemph{strong $k$-oriented homotopy equivalence} if there are maps $H$ and $H_X$ which exhibit $f$ as a retract of $\delta^k \times ' f$ as follows:

\[
\xymatrix@C=2cm{
Y \ar[d]_{f} \ar[r]^-{\delta^{1-k} \times Y} & 
( \Delta[1] \times Y ) \cup X \ar[d]^{\delta^k \times' f} \ar[r]^-{H_X} & 
Y \ar[d]^{f} \\
X \ar[r]_-{\delta^{1-k} \times X}  & 
\Delta[1] \times X \ar[r]_{H} &
X  }
\]
\end{definition}

\begin{remark} \label{rem:below-strhtpyequiv}
The definition of strong $k$-oriented homotopy equivalence in~\cref{def:strhtpyequiv} is equivalent to the one given in \cite{gambino2017frobenius} by Lemma~3.3 therein.
With the definition given here, it is immediate to observe that a cofibration which is a strong $k$-oriented homotopy equivalence is a trivial cofibration. Indeed, if $f \co Y \to X$ is a cofibration then $\delta^k \times ' f$ is 
a weak equivalence by the pushout-product property and so $f$ is also a weak equivalence, since it is a retract of $\delta^k \times ' f$.
\end{remark}

\begin{lemma}\label{lemma:genTcof_strongHequiv} \hfill 
\begin{enumerate}[$(i)$]
\item For $i < n$, the horn inclusions $h^n_i \co \Lambda^i[n] \rightarrow \Delta[n]$ are strong $0$-oriented homotopy equivalences,
\item For $0 < i $, the horn inclusions $h^n_i \co \Lambda^i[n] \rightarrow \Delta[n]$ are strong $1$-oriented homotopy equivalences.
\end{enumerate}
\end{lemma}

\begin{proof}
This is shown as part of~\cite[Theorem 3.2.3]{joyal-tierney:simplicial-homotopy-theory}. The proof given there can be easily checked to be constructive. This argument as also been reproduced (in the context of complicial sets) in the first part of the proof of \cite[Proposition~5.2.6]{henry2018wms} which is developed in constructive settings.
\end{proof}

%I've only added the reference to my paper to avoid having Joyal and Tierney's notes as unique reference for this. But If you think it is fine, you can remove it.



\begin{lemma} 
\label{lemma:pb_of_StrongHomotopyEq}
Let $p \co A \rightarrow X$ be a fibration with cofibrant domain. Then, for $k \in \{0,1\}$, 
the pullback functor $p^* \co \SSet_{/X} \to \SSet_{/A}$ preserves strong $k$-oriented homotopy equivalences.
\end{lemma}



\begin{proof} This is essentially \cite[Lemma~3.7]{gambino2017frobenius}, but we provide some details
for the convenience of the reader.
Let $f \co Y \rightarrow X$ be a strong $k$-oriented homotopy equivalence. Let $H$ and $H_X$  maps 
as in~\cref{def:strhtpyequiv}. Let $p \co A \rightarrow X$ be a fibration with cofibrant domain and consider the pullback
\[
\xymatrix@C=1.5cm{
A[f] \ar[r]^-{p^*(f)}  \ar[d]_{p_{A[f]}} & A \ar[d]^{p} \\
Y \ar[r]_{f} & X \, .\\
}
\] 
To show that $p^*(f)$ is a strong $k$-oriented homotopy equivalence, we let $K \co \Delta[1] \times A \rightarrow A$ be a diagonal filler in the square:
\[
\xymatrix@C=1.5cm{
A \ar[d]_{\delta^{1-k}} \ar@{=}[rr] & & A \ar[d]^p \\ 
\Delta[1] \times A \ar[r]_{\Delta[1] \times p}  & \Delta[1] \times X \ar[r]_H & X
}\]
Where the map on the left-hand side is a trivial cofibration because $A$ is a cofibrant.
It remains to construct a map $K_Y$  fitting into a retract diagram of the form
\[
\xymatrix@C=1.5cm{
A[f]  \ar[d]_{p^*(f)} \ar[r] &  (\Delta[1] \times A[f])  \cup Y \ar[d]^{\delta^k \times' p^*(f)} \ar[r]^-{K_Y} & A[f] \ar[d]^{p^*(f)} \\
A \ar[r] & \Delta[1]  \times A \ar[r]_-{K} & A
}
\]
We define it using the universal property of $A[f]$ as the unique map to $A[f]$ such that the image in $X$ is the one specified by the diagram above, and the value in $Y$ is the one given by $H_X$ composed with the map $(\Delta[1] \times A[f]) \cup Y$ to $(\Delta[1] \times A[f]) \cup X$. These indeed have the same image in $X$ exactly because of the commutation of lower triangle in the filler diagram defining~$H'$. The commutation of the diagram and the fact that the upper line is a retract are immediate with this definition.
\end{proof}



\begin{proposition}\label{prop:Frobenius}
Let $p \co A \rightarrow X$ is a fibration with cofibrant domain. Then the pullback functor 
\[
p^* : \SSet_{/X} \rightarrow \SSet_{/A}
\] 
preserves cofibrations and trivial cofibrations.
\end{proposition}


\begin{proof} Since the pullback functor $p^*$ has a right adjoint,  it is enough to check that pullbacks of generating cofibrations and generating trivial cofibrations are cofibrations and trivial cofibrations, respectively. For cofibrations, this is \cref{thm:cof-pbk}. For trivial cofibrations, note that the trivial cofibrations in $\SSet_{/Y}$ are generated by the horn inclusions $h^k_n \co \Lambda^{k}[n] \rightarrow \Delta[n]$ for all possible choices of $\Delta[n] \rightarrow A$. By \cref{lemma:genTcof_strongHequiv} they are all strong $k$-oriented homotopy equivalences. Moreover, their pullback to $X$ is also their pullback along the map $A \times_X \Delta[n] \rightarrow \Delta[n]$ which is again a fibration with cofibrant domain by \cref{lem:cofibrant_fiber_product}. Hence  \cref{lemma:pb_of_StrongHomotopyEq} implies that the pullback 
to~$X$ are also strong $k$-oriented homotopy equivalences. Since they are cofibrations, they are 
trivial cofibrations by \cref{rem:below-strhtpyequiv}.
\end{proof}



\begin{corollary}\label{cor:Pi_types_are_fibrant}
Let $p \co A \rightarrow X$ be a fibration with cofibrant domain. The functor $\Pi_p \co \SSet_{/A} \to \SSet_{/X}$ restricts to a functor
\[
\Pi_p \co \Fib_{/A}  \to \Fib_{/X} \, .
\]
Furthermore, this restriction preserves  trivial fibrations.

\end{corollary}

\begin{proof}
Since $p^*$ preserves cofibrations and trivial cofibrations, its right adjoint $p_*$ preserves trivial fibrations and fibrations. In particular, it preserves fibrant objects.
\end{proof}


Given a cofibrant simplicial set $X$ and a fibrant simplicial set, the special case of \cref{cor:Pi_types_are_fibrant} with $p$   the unique map $A \to 1$ and $q$ the second projection 
$\pi_2 \co A \times A \to A$, which is a fibration if~$A$ is a  fibrant, implies that $A^X$ is fibrant, as we already established in part~(i) 
of~\cref{thm:exponentials}. 




% \begin{theorem} 
% \label{thm:restricted-frobenius}
% The semi-model structure for Kan complexes on $\SSet$ has the restricted Frobenius condition.
% \end{theorem} 

% \begin{proof}  Since the semi-model structure in which we 
% are working is cofibrantly generated, it is sufficient [TO CHECK] that $p^*$ sends generating trivial cofibrations to trivial cofibrations. So, let $p \co B \to \Delta[n]$ be a fibration, $i \co \Lambda^k[n]
% \to \Delta[n]$ be a horn inclusion, and define $ j \defeq p^*(i)$, given by 
% the pullback diagram
% \[
% \xymatrix{
% \bullet  \ar[r] \ar[d]_j \drpullback & \Lambda^k[n] \ar[d]^{i} \\
% B \ar[r]_-{p} & \Delta[n] }
% \]
% We need to show that $j$ is a trivial cofibration.  
% First, since $i$ is a trivial cofibration, it is in particular
% a cofibration and therefore $j$ is again a cofibration by~\cref{thm:cof-pbk}. Secondly, since~$i \in \cal{J}$,
% by part~(i) of \cref{thm:missing-1}, it is a cofibration and a strong homotopy equivalence. Since its codomain is cofibrant, $j$ is a strong homotopy equivalence by~\cref{thm:missing-2}.
% But now $j$ is both a cofibration and a strong homotopy equivalence
% and hence it is a trivial cofibration, as required, by part~(ii) of \cref{thm:missing-2}.
% \end{proof} 



 


\begin{remark}[$\Pi$-types] \label{rem:pi-types}
We can define a naive interpretation of $\Pi$-types via a cofibrant replacement of dependent product. In order to explain this, let
us recall that, for maps $p \co A \to X$ and $q \co B \to A$,  the dependent product $\Pi_p(q) \co \Pi_A(B)
\to X$ is equipped with a map
\[
\mathsf{app} \co \Pi_A(B) \times_A  A \to B
\] 
in $\SSet_{/A}$ which is universal in the sense that, for every  $Y \to X$, the function
\[
\begin{array}{rcl} 
 \SSet_{/X}[ Y , \Pi_A(B)] & \longrightarrow &  \SSet_{/A}[Y \times_A A, B]  \\
  f & \longmapsto & \mathsf{app}(f \times_A 1_A) 
  \end{array} 
 \]
 is a bijection. This means that we have a function $\lambda$ in the opposite direction such that  
 \begin{equation}
 \label{equ:betaeta}
 \mathsf{app}(\lambda(b) \times_A 1_A) = b   \, , \quad
 \lambda( \mathsf{app}(f \times_A 1_A)) = f \, ,
 \end{equation}
 for every $b \co Y \times_A A \to B$ and $f \co Y \to \Pi_A(B)$.  These equations correspond to the
 well-known $\beta$-rule and $\eta$-rule for $\Pi$-types, respectively.
 
 When $p$ and $q$ are fibrations and $A$ is cofibrant, the map 
 $\Pi_p(q) \co \Pi_A(B) \to X$ is a fibration by \cref{cor:Pi_types_are_fibrant} but $\Pi_A(B)$ is not cofibrant
 in general. Thus, we interpret  $\Pi$-types as the 
 cofibrant replacement  of $\Pi_A(B)$, which is given by a cofibrant simplicial set
 $\mathbb{L}(\Pi_A(B)$  equipped with
 a trivial fibration $t \co \mathbb{L}(\Pi_A(B)) \to \Pi_A(B)$. 
We then define $\widetilde{\mathsf{app}} \co   \mathbb{L}(\Pi_A(B)) \times_A A \to B$ by letting
\[
\widetilde{\mathsf{app}}  = \mathsf{app} \circ (t \times_A 1_A) \, .
\]
For a bifibrant simplicial set $Y$ and maps $Y \to X$,  $b \co Y \times_A A \to B$, we define $\widetilde{\lambda}(b) \co Y \to \mathbb{L}(B^A)$ to be the
diagonal filler
\[
\xymatrix{
0 \ar[r] \ar[d] & \mathbb{L}(\Pi_A(B))  \ar[d]^t \\
Y \ar[r]_{\lambda(b)} \ar@{.>}[ur] & \Pi_A(B)}
\]
which exists since $Y$ is cofibrant and $t$ is a trivial fibration. It follows immediately that
\[
 \widetilde{\mathsf{app}}(\widetilde{\lambda}(b) \times_A 1_A) = b \, ,
\]
so the $\beta$-rule holds as an equality. Instead, for $f \co X \to \mathbb{L}(\Pi_A(B))$, we have a homotopy
\[
\eta_f  \co \widetilde{\lambda}( \widetilde{\mathsf{app}}(f \times_A 1_A)) \sim  f  \, ,
\]
which is constructed as the diagonal filler in the following diagram
\[
\xymatrix@C=2cm{
\partial \Delta[1] \times Y \ar[r]^-{[f, \widetilde{\lambda}(f \times 1_A)]} \ar[d] & \mathbb{L}(\Pi_A(B)) \ar[d]^t \\
\Delta[1] \times Y \ar[r] \ar@{.>}[ur] & \Pi_A(B) }
\]
where the bottom map is given by the equality in the $\eta$-rule in~\eqref{equ:betaeta}.
\end{remark}



\section{The weak equivalence extension property}
\label{sec:equep}

The main goal of this section is to prove the so-called weak equivalence extension property, which will be the key to prove the existence and univalence of a weakly universal fibration.  For this, we follow closely the approach in \cite{voevodsky-simplicial-model}, but exploiting crucially the cofibrancy requirements that are part of our set-up.



\begin{lemma}\label{Lemma:ForTheExtProperty} Let $f \co Y \rightarrow X$ be a cofibration with $Y$ cofibrant. 
\begin{enumerate}[$(i)$]
\item The functor $\Pi_f \co \SSet_{/Y} \rightarrow \SSet_{/X}$ preserves trivial fibrations.
\item The counit of the adjunction $f^* \dashv \Pi_f$ is a natural isomorphism.
\item If $g \co Z \to Y$  is cofibrant in $\SSet_{/Y}$, then $\Pi_f(g) \co \Pi_f(Z) \to X$  is cofibrant in~$\SSet_{/X}$.
\item Trivial fibrations extend along cofibrations with cofibrant domain, \ie given a trivial fibration $q
 \co B \to Y$  as in the solid diagram:
\[
\xymatrix{
B \ar@{.>}[r]^{g} \ar[d]_{q} \drpullback  & A \ar@{.>}[d]^{p} \\
Y \ar[r]_f &  X \, ,}
\]
then there exists a trivial fibration $p \co A \rightarrow X$ which fits in the dotted pullback square above. Moreover if $B$ is cofibrant  then 
$A$ can be chosen to be 
cofibrant as well.
\end{enumerate}
\end{lemma}

 
\begin{proof} We prove the different parts separately. 
\begin{enumerate}[$(i)$] 
\item 

Since the functor $\Pi_f$ is the right adjoint to the pullback functor $f^*$ and trivial fibrations are the maps with the the right lifting property with respect to cofibrations, $\Pi_f$ preserves trivial fibrations if and only if $f^*$ preserve cofibrations. But this follows by~\cref{thm:cof-pbk}.

\item As $f$ is a monomorphism, then the forgetful functor $\Sigma_{f} \co \SSet_{/X} \rightarrow \SSet_{/Y}$ is fully faithful and hence the unit $\eta \co 1_{\SSet_{/A}} \rightarrow f^* \Sigma_{f}$ is an isomorphism. By adjointness, the counit $\varepsilon \co f^* \Pi_f \rightarrow 1_{\SSet_{/Y}}$ is also an isomorphism.

\item \hfill 

\begin{center}
\noten{To be revised} 
\end{center}

Let $v$ be a $n$-cell in $\Pi_i(X)$.
If the image of $v$ is in $A$ then $v$ is a cell of $X \subset \Pi_i(X)$, in which case it is decidable whether $v$ is degenerate or not.
As $A$ is levelwise complemented in $B$, one can assume that $v$ is not in the image of $A$. In this case it is decidable if the image of $v$ in $B$ is degenerate or not.
Infact, by the Eilenberg-Zilber lemma one can also decide for each given degeneracy if the image of $v$ is degenerate for this precise degeneracy or not. 

Let $\sigma \co [n] \to [k]$ be any degeneracy, we will show that it is decidable whether $v$ is ``$\sigma$-degenerate, i.e. if $v =\sigma^* v'$ for some $v'$. Note that if $v$ is $\sigma$-degenerate then its image in $B$ is as well. As this is a decidable question, one can freely assume that the image of $v$ in $B$ is $\sigma$-degenerate, i.e. for the form $\sigma^* b$ for some $b \in B$ (and not in $A$).

 One can form the pullback square:


\[
\xymatrix{
V \ar@{^{(}->}[r] \drpullback \ar[d] & \Delta[n] \ar@{->}[d]^{\sigma} \\
V_{\sigma} \ar@{^{(}->}[r] \drpullback \ar[d] & \Delta[k] \ar@{->}[d]^b \\
A \ar@{^{(}->}[r] &  B }
\]

Given its image in $B$, the cell $v: \Delta[n] \rightarrow \Pi_i X$ is uniquely determined by the data of a morphism $\lambda: V\rightarrow X$. The cell $v$ is $\sigma$-degenerate if and only if $\lambda$ factors in $V_{\sigma}$ (such a factorization being unique if it exists). For any $J \subset [n]$, the $J$-face of a cell is said to be $\sigma$-degenerate if and only if it is degenerate for the (potentially trivial) degeneracy: $\sigma_{|J} : J \rightarrow \sigma(J)$. We claim that $\lambda$ factor into $V_{\sigma}$ if and only for all $i:[f] \hookrightarrow [n]$ that belong to $V$ (and $V$ is decidable so there is only a finite cardinal of them), $i^* \lambda$ is $\sigma$-degenerate (which is dediable). Indeed $V$ is the gluing of all the $\sigma \circ i$ for such faces, for each individual face $i$ one has a factorization into its image in $V_{\sigma}$ if and only if $i^* \lambda$ is $\sigma$-degenerate, and as such factorization are unique they patch together on $V_{\sigma}$ is they all exists.

\item One can simply take $p \co A \to X$ to be $\Pi_f(q) \co \Pi_f(B) \to X$. Indeed, it is a trivial fibration by part~$(i)$ and the square is a pullback by part $(ii)$. The final remark about the cofibrancy of 
$A$ follows from $(iii)$. \qedhere
\end{enumerate}
\end{proof}


\notesh{I remember we proved $(iii)$ explicitly as this was something I was worried about. But the proof above is a lot harder than in my memories. Do you have any notes about this claims ? If not just erase this note.}



\begin{proposition}[Weak equivalence extension property]
\label{Prop:Homotopy_ext_prop}
Let 
\[
\xymatrix{
B \ar[r]^g \ar[d]_q & A \ar[d]^p \\
Y \ar[r]_f & X}
\]
be a commutative diagram with $p \co A \to X$ and $q \co B \to Y$ be fibrations, $f \co Y \to X$ 
a cofibration and such that the map $u \co B \to A[f]$ defined by $u \defeq (q, g)$, fitting the diagram 
of solid maps
\[ 
\xymatrix{
 B
  \ar@{.>}[rr]
  \ar[dr]^{u}
  \ar[dd]_(.3){q}
&&
  \bar{B}
  \ar@{.>}[dr]^{v}
  \ar@{.>}[dd]_(.3){\bar{q}}|{\hole}
&\\&
  A[f] 
  \ar[rr]
  \ar[dl]
&&
  A
  \ar[dl]^{p}
\\
  Y
  \ar[rr]_{f}
&&
  X \, ,
&
}
\]
is a weak equivalence in $\SSet_{/ Y}$. Then there exist a fibration $\bar{q} \co \bar{B} \to X$, a weak equivalence $v \co \bar{B} \to A$ in $\SSet_{/X}$ and a map $B \to \bar{B}$ such that all the squares in the diagram above are pullbacks. 
\end{proposition}

\begin{proof} We define the required object $\bar{B}$ as the following pullback:
\[\xymatrix{
\bar{B} \ar[d] \ar[r] \drpullback & \Pi_Y(B) \ar[d] \\
A \ar[r]_-{\eta_{A}} & \Pi_Y \big( A[f]  \big) \, ,
}\]
where $\eta_{A}$ is  a component of the unit of adjunction  $f^* \dashv \Pi_f$. An application of the pullback functor $f^* \co \SSet_{/X} \to \SSet_{/Y}$ to this pullback square gives the commutative square
\[\xymatrix{
\bar{B}[f] \ar[d] \ar[r]  & B \ar[d] \\
A[f] \ar@{=}[r] &A[f] 
}\]
This is a pullback since $f^* \Pi_f \iso 1$ by part~(ii) \cref{Lemma:ForTheExtProperty}. Hence 
$B \iso \bar{B}[f]$, as required.


Since $B$ is cofibrant, we have that $\Pi_Y(B)$ is cofibrant by part~(iii) of  \cref{Lemma:ForTheExtProperty}. Hence, the simplicial set~$\bar{B}$  is also cofibrant by \cref{lem:cofibrant_fiber_product}. Furthermore, the maps $B \rightarrow \bar{B}$ and~$A[f] \rightarrow A$ are cofibrations by~\cref{thm:cof-pbk}, as they are pullback of the cofibration~$f \co Y \rightarrow X$.





It remains to prove that $v \co \bar{B} \rightarrow A$ is a weak equivalence and that 
$\bar{q} \co \bar{B} \rightarrow X$ is a fibration. Since the map $u$ can be factored into a trivial cofibration followed by a trivial fibration, and our construction are functorial, it is sufficient to prove these claims when~$u$ is a trivial fibration or a trivial cofibration.

If $u$ is a trivial fibration, then its image under $f_*$ is a trivial fibration by 
part~(i) of \cref{Lemma:ForTheExtProperty}. Since the map $\bar{B} \rightarrow A$ is a pullback of this map,
it is also a trivial fibration. This also implies that the composite $\bar{B} \rightarrow A \rightarrow X$ is a fibration.


We now assume that $u \co B \rightarrow A[f]$ is a trivial cofibration. Using that the maps from $\bar{B}$ and $A[f]$ to $Y$ are fibrations between fibrant objects, we can show that $u$ is a strong deformation retract over $Y$, \ie there is a retraction $r \co A[f] \rightarrow B$ of $u$ in $\SSet_{/Y}$ and a homotopy 
\[
H \co \Delta[1] \times A[f] \rightarrow A[f]
\] 
between 
$u \circ r$ and $1_{A[f]}$, whose composite with $A[f] \rightarrow Y$ is the trivial homotopy.

We want to show that $\bar{B} \rightarrow A$ is also a deformation retract by constructing a similar homotopy 
\[
H' \co \Delta[1] \times A \rightarrow A \, .
\] 
This homotopy will be constructed so that it is $H$ on $I \times A[f]$ ,  it is the map 
\[
\Delta[1] \times \bar{B} \rightarrow \Delta[0]  \times \bar{B} \iso \bar{B} \rightarrow A
\] 
on $\Delta[0] \times \bar{B} $ (indeed they agree on $\Delta[1] \times B$) and it is the identity on $\Delta[0] \times A$.  This is achieved by taking a diagonal filling in the square:
\[
\xymatrix@C=1.5cm{
\big( \Delta[1] \times (\bar{B} \cup A[f]) \big)  \cup \big( \Delta[0] \times A \big) \ar[d] \ar[r] & A \ar[d] \\
\Delta[1] \times A \ar[r] \ar@{.>}[ur]^{H'} & X
}\]
Such a diagonall filler exists since the map on the left-hand side is a trivial cofibration, being the 
 pushout-product of $Y_0 \co \Delta[0] \rightarrow \Delta[1]$ and the cofibration $\bar{B} \cup A[f] \rightarrow A$, and the map on the right-hand side is a fibration by assumption.

It remains to see that the map $H_{1} \co A \rightarrow A$ is indeed a retraction of $\bar{B} \rightarrow A$. We already know that the restriction of $H_{1}$ to $\bar{B}$ is  the inclusion of $\bar{B}$ in $A$, so it is enough to show that $H_{1}$ has values in $\bar{B}$. We also know that $H_{1}$ restricted to $A[f]$ takes values in $B \subseteq \bar{B}$. By definition of $\bar{B}$, the map $H_1$ factor into $\bar{B}$ if and only if it takes values in $\Pi_Y(B)$ when seen as a map to $\Pi_Y(A[f])$, and by adjunction this is the case if and only if the map corresponding to $H_1$, $A[f]= f^*(A) \rightarrow A[f]$ takes values in $B$, but already mentioned above that this was indeed the case.

The fact that $\bar{B} \rightarrow A$ is a deformation retract show that it is invertible in the homotopy category, in particular it is indeed a weak equivalence. The construction above also shows that~$\bar{B}$ is retract of $A$ in $\SSet_{/X}$ and hence $\bar{q} \co \bar{B} \rightarrow X$ is a fibration because $p \co A \rightarrow X$ is.
\end{proof}

 

\section{A universal fibration with cofibrant fibers}





The goal of this section is to define a small fibration $\pi_c \co \UU_c \to \U_c$ between
cofibrant simplicial sets that weakly classifies small fibrations between cofibrant simplicial sets, in the sense that for every such fibration $p \co A \to X$ there exists a map $a \co X \to \U_c$ fitting in a pullback diagram of the form
\[
\xymatrix{
A \ar[r] \ar[d]_p   & \UU_c \ar[d]^{\pi_c} \\
X \ar[r]_a &  \U_c \, .}
\]
We will show in~\cref{sec:fibrancy-and-univalence} that $\U_c$ is actually bifibrant and $\pi_c$ is
univalent.

\bigskip

For the goal of this section, we proceed in two steps. First, we modify  the construction of the weak classifier for small fibrations in~\cite{voevodsky-simplicial-model} to obtain a small fibration $\pi \co \UU \to \U$ which weakly classifies  small fibrations between cofibrant objects. Since the base of this fibration does not appear to be cofibrant,
we then consider a suitable cofibrant replacement  of~$\U$ and obtain the required fibration $\pi_c \co \UU_c \to \U_c$ via a pullback. 

\medskip

As a preliminary step, let us recall that a simplicial set $A$ is \emph{small}  if $A_n$ is a small set for every $[n] \in \Delta$ and that a map $p \co A \to X$ of simplicial sets is \emph{small} if for every $x \co \Delta[n] 
\to X$ the simplicial set~$A[x]$ given by the pullback square
\[
\xymatrix{
A[x] \ar[r] \ar[d] \drpullback & A \ar[d]^{p} \\
\Delta[n] \ar[r]_-{x} & X }
\]
is small. Let us also recall the  construction of the weakly universal small map of simplicial sets $\rho \co \VV \to \V$, which is a special case of the results in~\cite{hofmann-streicher-universes} for arbitrary presheaf categories.  For this, we use the equivalence in~\eqref{equ:psh-slice-sset} and the notation associated to it.
The simplicial set $\V$ is defined by letting
\[
\mathsf{V}_n \defeq \{ F \co {\Delta_{/[n]}}^{\op} \to \Set \ | \ \pi_F \co \textstyle{\int F} \to \Delta[n] \text{ is a small
map} \}
\]
for $[n] \in \Delta$. The object $\VV$ and the map $\rho \co \VV \to \V$ are then defined in an evident way. 


\bigskip

We now come to our first step, in which we define a small fibration $\pi \co \UU \to \U$ which weakly classifies  small fibrations with cofibrant fibers. 

\begin{definition} We say that a fibration $p \co A \to X$ has cofibrant fibers if for every $x \co \Delta[n] 
\to X$ the simplicial set~$A[x]$ given by the pullback square
\[
\xymatrix{
A[x] \ar[r] \ar[d] \drpullback & A \ar[d]^{p} \\
\Delta[n] \ar[r]_-{x} & X }
\]
is cofibrant.
\end{definition}


\begin{lemma} Let $p \co A \to X$ be a fibration with cofibrant fibers. If $X$ is cofibrant, then
so is $A$.
\end{lemma} 

\begin{proof} Let $[n] \in \Delta$, $a \in A_n$. Since $X$ is cofibrant, by the Eilenberg-Zilber lemma we can write $p(a) \in X$ in a unique way as $p(a) =s^*(x)$, where $s \co [n] \to [k]$ is a degeneracy and 
$x \in X_k$ is a non-degenerate cell. Let $x \co \Delta[k] \rightarrow X$ be the corresponding map. We now form the pullback
\[
\xymatrix{
A[x] \ar[r]^w \ar[d] \drpullback & A  \ar[d]^{p}  \\
\Delta[k] \ar[r]_{x} & X }
\]
By the universal property of the pullback, there is a unique cell $e \in A[x]_n$ such that $w(e)=a$, and the image of $e$ in $\Delta[k]$ is the cell $s \co [n] \to [k]$, whose image in $X$ are both equal to $p(a)=s^*(x)$.

By the assumption that $p$ has cofibrant fibers, the simplicial set $A[x]$ is cofibrant and hence it is decidable whether $e$ is degenerate or not. We claim that $a$ is degenerate if and only if $e$ is, which implies that it is decidable whether $a$ is degenerate.

\caution{Notation to be fixed.}  Indeed as $a = w(e)$ then if $e$ is degenerate so $a$ is. Conversely, assume that $a=\theta^*(y_1)$ for a non-trivial degeneracy $\theta$. Then $p(a)=\theta^*(x_1)$, hence by the uniqueness part of the Eilenberg-Zilber lemma for $X$ one has that $s=q \circ \theta$ for some degeneracy $q$, and $x_1 = q^*(x')$. In particular, we get a unique cell $e_1$ of $A[x]$ whose image in $\Delta[n]$ and $X$ are  $q$ and $a_1$, respectively, whose images in $X$ are both equal to $x_1=q^*(x')$. Finally, the image of $p^*(e_1)$ in $\Delta[n]$ and~$A$ are  $p^* y_1 =a$ and $q \circ p =s$, respectively, and hence $p^*(e_1) =e$, which proves that $e$ is degenerate as soon as $a$ is.
\end{proof} 


Define a subobject $\U \subseteq \V$ by letting, for $[n] \in \Delta$, 
\[
\U_n = \{ F \in \V_n \ | \ \pi_F \co \textstyle{\int F} \to \Delta[n] \text{ is a small fibration and 
$ \textstyle{\int F}$ is cofibrant} \} \, .
\]
We then define the map $\pi \co \UU \to \U$ via the pullback 
\[
\xymatrix{
\UU \ar[r] \ar[d]_\pi \drpullback  & \VV \ar[d]^\rho \\
\U \ar@{>->}[r] & \V }
\]




\begin{proposition} \label{thm:universe-u}  \hfill 
\begin{enumerate}[(i)] 
\item The map $\pi \co \UU \to \U$ is a small fibration with cofibrant fibers.
\item The map $\pi \co \UU \to \U$ classifies small fibrations with cofibrant fibers, \ie 
a map $p \co A \to X$ is a small fibrations with cofibrant fibers if and only if 
there is a map $a \co X \to \U$ and a pullback  of the form
\[
\xymatrix{
A \ar[r] \ar[d]_p \drpullback & \UU \ar[d]^\pi \\
X \ar[r]_{a} & \U }
\]
\end{enumerate}
\end{proposition}

\begin{proof} We prove the two claims separately.
\begin{enumerate}[(i)] 
\item The map $\pi$ is a fibration since we can rewrite a general lifting problem against a horn inclusion $h^n_k \co \Lambda^k[n] \rightarrow \Delta[n]$ as follows:
\[
\xymatrix{
\Lambda^k[n] \ar[r] \ar[d]_{h^k_n} & A  \ar[d]^{p} \ar[r] & \UU \ar[d]^\pi \\
\Delta[n] \ar@{=}[r]  & \Delta[n] \ar[r]_a & \U  }
\]
and then use that $p \co A  \to \Delta[n]$ is a fibration. To show that it is a small fibration with
cofibrant fibers, consider a map $a \co \Delta[n] \rightarrow \U$ and the pullbacks
\[
\xymatrix{
A \ar[r] \ar[d]_p  & \UU \ar[d]^{\pi} \ar[r]  & \VV \ar[d]^{\rho} \\
\Delta[n] \ar[r]_{a} & \U \ar@{>->}[r] & \V }
\]
This shows that $p \co A \rightarrow \Delta[n]$ is isomorphic to $\pi_F \co \int F \rightarrow \Delta[n]$ in $\SSet_{/ \Delta[n]}$, where $F$ corresponds under the equivalence in~\eqref{equ:psh-slice-sset} to 
$a \co \Delta[n] \rightarrow \V$. Therefore, by definition of $\U$, $A$ is cofibrant and $p \co A \rightarrow \Delta[n]$ is a small fibration. This implies that $\pi \co \UU \rightarrow \U$ is a small fibration with cofibrant
fibers. 
\item If a map $p \co A \to X$ fits in a diagram as in the statement, then it is clearly a small fibration with
cofibrant fibers. Conversely, let $p \co A \to X$ be a small fibration with cofibrant fibers. Being a small 
map, $p$ fits into a pullback of the form 
\[
\xymatrix{
A \ar[r] \ar[d]_p  &  \VV \ar[d]^{\rho} \\
X \ar[r]_{a} & \V }
\]
But the map $a \co X \rightarrow \V$ factors via the inclusion $\U \to \V$ by the definition of
$\U$, since $p$ is a small fibration with cofibrant fibers.  \caution{More details needed.} \qedhere
\end{enumerate}
\end{proof} 



\bigskip

We now  construct a fibration $\pi_c \co \UU_c \to \U_c$ that classifies small fibrations between 
cofibrant objects. In particular, the base $\U_c$ of this fibration will be cofibrant.  In order to do this, let $\U_c$ be a cofibrant replacement of~$\U$,  which comes with a trivial fibration
\begin{equation}
\label{equ:ucu}
\tau \co \U_c \rightarrow \U \, .
\end{equation}
We then define $\UU_c$ via the pullback
\[
\xymatrix{
\UU_c \ar[d]_{\pi_c} \ar[r] \drpullback & \UU \ar[d]^{\pi}  \\
\U_c \ar[r]_\tau & \U}
\]
We now prove that $\pi_c \co \UU_c \to \U_c$ has the required properties.

\begin{theorem} \label{thm:universe-uc} 
\hfill 
\begin{enumerate}[(i)] 
\item $\pi_c \co \UU_c \to \U_c$ is a small fibration with cofibrant fibers and cofibrant codomain. 
\item The map $\pi_c \co \UU_c \to \U_c$ classifies small fibrations between with cofibrant fibers and cofibrant codomain, \ie a map $p \co A \to X$ is in that class if and only if $X$ is cofibrant and 
there exists a pullback diagram of the form
\[
\xymatrix{
A \ar[r] \ar[d]_p & \UU_c \ar[d]^{\pi_c} \\
X \ar[r]_a & \U_c }
\]
\end{enumerate}
\end{theorem}

\begin{proof} \hfill
\begin{enumerate}[(i)] 
\item $\U_c$ is cofibrant by construction and the rest of the claim follows from part~(ii) of~\cref{thm:universe-u}. 
\item This follows from the cofibrancy of $X$ and $\UU_c$ by~\cref{lem:cofibrant_fiber_product}.
\item First, let us assume that  $X$ is cofibrant, $p \co A \to X$ is a small fibration with cofibrant fibers
and show that $p$ fits into a pullback as in the statement.  By part~(ii) of \cref{thm:universe-u} there is a pullback diagram of the form 
\[
\xymatrix{
A \ar[r] \ar[d]_p \drpullback & \UU \ar[d]^{\pi} \\
X \ar[r]_a & \U }
\]
Since $X$ is cofibrant, we have a diagonal filler in the diagram
\[
\xymatrix{
0 \ar[r] \ar[d] & \U_c \ar[d]^{\tau} \\
X \ar[r]_a \ar@{.>}[ur]^{a_c} & \U }
\]
 We then obtain the diagram
\[
\xymatrix{
A \ar[r] \ar[d]_p &  \UU_c \ar[r]  \ar[d]^{\pi_c}  & \UU \ar[d]^{\pi} \\
X \ar[r]_{a_c} & \U_c \ar[r]_{a} &  \U }
\]
Here, the right-hand side square and the rectangle are pullbacks and therefore the left-hand
side square is also a pullback, as required. \qedhere
\end{enumerate} 
\end{proof} 

\section{Fibrancy and univalence of the universe}
\label{sec:fibrancy-and-univalence}




Let us return to consider the fibration $\pi \co \UU \to \U$. Let $\U^{\rightarrow}$ be the simplicial set whose $n$-simplices are triples of the form $(F_0, F_1, \phi)$, where $F_0$ and $F_1$ are $n$-simplices of $\U$, \ie functors
 \[
F_0, F_1 \co {\Delta_{/[n]}}^{\op} \rightarrow \Set
\]
and $\phi \co F_0 \Rightarrow F_1$ is a natural transformation. By the equivalence in~\eqref{equ:psh-slice-sset},
such triples correspond to commutative diagrams of the form
\begin{equation}
\label{equ:corresp}
\begin{gathered}
\xymatrix{ 
A_1 \ar[rr]^{f} \ar[dr]_{p_1} & & A_2 \ar[dl]^{p_2} \\
& \Delta[n] & }
\end{gathered}
\end{equation}
where $p_1$ and $p_2$ are fibrations with cofibrant domain.


\begin{lemma} $\U^{\rightarrow} \rightarrow \U \times \U$ is a fibration
\end{lemma}

\begin{proof} Observe that $\U^{\rightarrow}$ is exactly $\Pi_p(\UU \times \UU)$, 
where $p \co \UU \times \U \rightarrow \U \times \U$ is the evident map. 
It follows from \cref{cor:Pi_types_are_fibrant} that $\U^{\rightarrow} \rightarrow \U \times \U$ is a fibration.
More precisely, \cref{cor:Pi_types_are_fibrant} implies that any pullback of  $\U^{\rightarrow} \rightarrow \U \times \U$ to a cofibrant
$X \rightarrow \U \times \U$ is a fibration (due to the cofibrancy assumption of  \cref{cor:Pi_types_are_fibrant}),
but this is sufficient to prove that $\U^{\rightarrow} \rightarrow \U \times \U$ is a fibration,
as in the argument for part~(i) of  \cref{thm:universe-u}.
\end{proof}


We define $\mathsf{Weq}(\U)$ as the simplicial subset of $\U^\to$ whose $n$-simplices are the $n$-simplices $(F_1, F_2, \phi)$ of $\U^\to$ such that the corresponding map~$f$ in~\eqref{equ:corresp} is a weak equivalence.
The simplicial set $\mathsf{Weq}(\U)$ is well defined because pullback functors are right Quillen functors \cite{henry2019qms} and therefore preserve weak equivalences. Hence, 
$\mathsf{Weq}(\U)$ as defined here is indeed a subobject of $\U^{\rightarrow}$. \caution{To be checked again!}




\begin{lemma}
\label{prop:Weq_classify_Weq}
For any cofibrant object $X$, a map $a \co X \rightarrow \U^{\rightarrow}$ factors via 
 $\mathsf{Weq}(\U)$ if and only the map in $\BFFib_{/X}$ classified by $a$,
\[
\xymatrix{
A_1 \ar[rr]^w  \ar[dr]_{p_1} & & A_2 \ar[dl]^{p_2} \\
 & X \, , & }
 \]
is a weak equivalence.
\end{lemma}

\begin{proof} By definition of $\mathsf{Weq}(\U)$, $a$ factors via $\mathsf{Weq}(\U)$ if and only if the pullback of $w \co A_1 \to A_2$ along any simplex $x \co \Delta[n] \rightarrow X$  is a weak equivalence. \caution{As observed above, this is indeed the case if $w$ is a weak equivalence.} Conversely,
we let $w \co A_1 \rightarrow A_2$ be a map between bifibrant objects of $\SSet_{/X}$, assume that the pullback of $w$ along every $x \co \Delta[n] \rightarrow X$ is a weak equivalence and show that $w$ is also equivalence.  We do so using~\cite[\S 2.5.7]{henry2018wms}, and show that $w$ has the weak right lifting property against all $i_n \co \partial \Delta[n] \rightarrow \Delta[n]$. So let us consider the diagram
\begin{equation}
\label{equ:before-pullback}
\begin{gathered}
\xymatrix{\partial \Delta[n] \ar[rr] \ar[d] & & A_1 \ar[d]^w \\
\Delta[n] \ar[rr] \ar[dr]_{x}
 & & A_2 \ar[dl] \\
& X \, .&}  
\end{gathered}
\end{equation}
By pulling back everything to $\Delta[n]$ we obtain the diagram
\begin{equation}
\label{equ:after-pullback}
\begin{gathered}
\xymatrix{
\partial \Delta[n] \ar[r] \ar[d] & A_1 \times_X \Delta[n] \ar[d]^{x^*(w)} \\
\Delta[n] \ar[r] & A_2 \times_X \Delta[n]  \, .
} 
\end{gathered}
\end{equation}
By assumption $x^*(w)$ is a weak equivalence between fibrant objects hence it has the weak right lifting property against $i^n \co \partial \Delta[n] \rightarrow \Delta[n]$. A weak diagonal filler for~\eqref{equ:after-pullback} then gives  a weak diagonal filler for~\eqref{equ:before-pullback}, as required.
\end{proof}




\begin{lemma} \label{thm:lemma-for-u} The map $\mathsf{Weq}(\U) \rightarrow \U^\to$ is a fibration.
\end{lemma}


\begin{proof}  Let $f \co Y \to X$ be a trivial cofibration between cofibrant objects and consider the diagram
\[ 
\xymatrix@C=1.5cm{ Y \ar[d]_f \ar[r]^-{w} & \mathsf{Weq}(\U) \ar[d] \\
X \ar[r]_-{a} & \U^{\rightarrow} }
\]
\caution{The map on the right-hand side is a monomorphism}, so a lifting is unique if it exists. It exists if the map $\bar{w}$ in $\SSet_{/X}$ classified by $w$ is a weak equivalence. 
\caution{But its pullback $f^*(\bar{w})$  in the diagram}
\[ 
\xymatrix{ 
f^*(A_1) \ar[d]_{f^*(w)}  \ar[r]  & A_1 \ar[d]^{w} \\
f^*(A_2) \ar[r] \ar[d]   & A_2 \ar[d] \\
Y \ar[r]_f & X }
\]
\caution{is a weak equivalence}.
Since the maps $p_i \co A_i \rightarrow X$ (for $i = 1, 2$) are fibrations with cofibrant domain, \cref{prop:Frobenius} implies that pullbacks of trivial cofibration between cofibrant objects 
\caution{along such 
a map} are trivial cofibrations. This implies that all the horizontal maps of the diagram above are weak equivalence, and so the upper right map also is. This shows that~$\mathsf{Weq}(\U) \rightarrow \U^{\rightarrow}$ is a fibration.
\end{proof}


\begin{theorem} \label{thm:fibrancy-of-u-and-uc} \hfill 
\begin{enumerate}[(i)] 
\item The simplicial set $\U$ is fibrant. 
\item The simplicial set $\U_c$ is bifibrant. 
\end{enumerate}
\end{theorem}

\begin{proof} We prove part~(i). Since
$(s, t) \co \mathsf{Weq}(\U) \rightarrow \U \times \U$ is a fibration, for any cofibrant 
simplicial set $X$,  maps $a_1 \, , a_2 \co X \rightrightarrows \U$ and homotopy $h \co \Delta[1] \times X \rightarrow \U$ from $a_1$ to $a_2$, there is a weak equivalence in $\SSet_{/X}$ between the objects classified by $a_1$ and $a_2$, constructed as follows. For this, we first consider a diagonal filler in the
diagram
\[
\xymatrix{ X \ar[r]^{i_1} \ar[d]_{\delta^0} & \mathsf{Weq}(\U) \ar@{->>}[d] \\
\Delta[1] \times X \ar[r]_{(a_1,h)} \ar@{.>}[ur] & \U \times \U
}
\]
Here, $i_1$ denotes a map classifying the identify of the object classified by $a_1$. By $a_1$ in the first component of the bottom arrow we mean the composite $\Delta[1] \times X \rightarrow X \rightarrow \U$. Composing the dotted arrow with $\delta^1$ gives us a map $X \rightarrow  \mathsf{Weq}(\U)$ whose projection to~$\U \times \U$ if $(a_1,a_2)$, \ie it classifies a weak equivalence between the objects classified by $a_1$ and $a_2$. One can do the same thing with $\delta^0$ and $\delta^1$ exchanged to get a weak equivalence in the other direction.

We can now prove that $\U$ is fibrant. A map $h^k_n \co \Lambda^k[n] \rightarrow \U $ classifies a fibration $q \co B \rightarrow \Lambda^k[n]$ with cofibrant domain. As proved in \cref{lemma:genTcof_strongHequiv}, the horn inclusions $h^k_n \co \Lambda^k [n] \rightarrow \Delta[n]$ are strong $k$-oriented homotopy equivalence and so this diagram can be extended by the retract diagram, where the precise maps depend on whether $k<n$ or $0 < k$,
\[
\xymatrix{\Lambda^k[n] \ar[d] \ar[r] & \big( \Delta[1] \times \Lambda^k[n] \big) \cup \Delta[n] \ar[d] \ar[r] & \Lambda^k[n] \ar[d]  \\
\Delta[n] \ar[r] & \Delta[1] \times \Delta[n] \ar[r] & \Delta[n] \, .
}\]
By the observation above, the composite map $\big( \Delta[1] \times \Lambda^k[n]  \big) \cup \Delta[n] \rightarrow \Lambda^k[n] \to \U$  gives a solid diagram of the form
\[ 
\xymatrix{
  B
  \ar@{.>}[rr]
  \ar[dr]
  \ar@{->>}[dd]
&&
  \bar{B}
  \ar@{.>}[dr]
  \ar@{.>>}[dd]|{\hole}
&\\&
  A'
  \ar[rr]
  \ar@{->>}[dl]
&&
  A
  \ar@{->>}[dl]
\\
  \Lambda^k[n]
  \ar[rr]_{h^k_n}
&&
  \Delta[n]
&
}
\] 
So one can construct a fibration  $\bar{q} \co \bar{B} \to \Delta[n]$ with cofibrant domain whose pullback 
along~$h^k_n$ is isomorphic to $q \co B \to \Lambda^k_n$. The map $b \co \Delta[n] \rightarrow \U$ classifying $\bar{q}$ gives the lift we are looking for. More precisely, one can use $q  \co \bar{B} \to
\Delta[n]$ to construct  a map $b \co \Delta[n] \rightarrow \U$ which extend the one we started from and which classifies something isomorphic to $\bar{B}$.

Part~(ii) follows immediately from part~(i) since $\tau \co \U_c \to \U$ is a trivial fibration.
\end{proof}




We now wish to define  what it means for a small fibration with cofibrant fibers, and in particular for a small fibration between cofibrant objects, to  be univalent. For this, we fix such a fibration $p \co A \to X$ and construct an object $\Weq(p) \to X \times X$ that represents weak
equivalences between fibers of $p$, in the sense that maps $\sigma \co Y \to \Weq(p)$ in $\SSet_{/X \times X}$ are in bijective correspondence with triples $(x_1, x_2, w)$ consisting of two map $x_1 \, , x_2 \co Y \to X$ and a weak equivalence $w \co A[x_1]
\to A[x_2]$ in $\SSet_{/Y}$. The required object $\Weq(p)$ can be constructed as the pullback
\[
\xymatrix{
\Weq(p) \ar[r] \ar[d] & \Weq(\U) \ar[d]^{(s,t)} \\
X \times X \ar[r]_{a \times a} & \U \times \U \, ,}
\]
where $a \co X \to \U$ is a classifying map for the small fibration $p \co A \to X$, which exists by our assumption
that $A$ and $X$ are cofibrant and part~(iii) of~\cref{thm:universe-u}. The verification that $\Weq(p)$
has the required universal property is an easy calculation, which we leave to the readers. There is an
evident map $i \co X \to \Weq(p)$ corresponding via the universal property of $\Weq(p)$ to the triple of identity maps $(1_X, 1_X, 1_A)$.




\begin{definition}  \label{equ:characterisations-of-univalence} Let $p \co A \to X$ be a small fibration with cofibrant fibers. We say that $p$ is \myemph{univalent} if the map $i \co X \to \Weq(p)$ is a weak equivalence. 
\end{definition}

\smallskip

For a small fibration with cofibrant fibers $p \co A \to X$, being univalent is equivalent to
either $s \co \Weq(p) \to X$ or $t \co \Weq(p) \to X$ being a trivial fibration.  Also note that, when $X$
is cofibrant, we have a map $j \co \Path(X) \to \Weq(p)$ fitting in the diagram
\[
\xymatrix@C=1.5cm{
X \ar[r]^i \ar[d]_r & \Weq(p) \ar[d]^{(s,t)} \\
\Path(X) \ar[r]_{\partial_X}  \ar@{.>}[ur]_{j} &  X \times X }
\]
Then, $p$ is univalent if and only if $j$ is a weak equivalence.





\begin{theorem}  \label{thm:fibrancy-of-u-and-uc} \hfill 
\begin{enumerate}[(i)]
\item The fibration $\pi \co \UU \to \U$ is univalent.
\item The fibration $\pi_c \co \UU_c \to \U_c$ is univalent.
\end{enumerate}
\end{theorem}

\begin{proof} For part (i), we prove that $t \co  \mathsf{Weq}(\U) \to \U$ has the right lifting property with respect
to all cofibrations. So let $f \co Y \rightarrow X$ be a cofibration and consider the diagonal
filling problem
\[
\xymatrix{Y \ar[d]_f \ar[r] & \mathsf{Weq}(\U) \ar[d] \\
X \ar[r] \ar@{.>}[ur]  & \U 
}
\]
By  \cref{prop:Weq_classify_Weq}, this corresponds exactly to a diagram as in the equivalence extension property as in \cref{Prop:Homotopy_ext_prop}. Indeed, the map $X \rightarrow \U$ gives us
$p \co A \to X$, the composite of~$Y \rightarrow  \mathsf{Weq}(\U)$ with the first projection
gives us $q \co B \to Y$, while the rest of the data and the commutativity of the square 
gives us a weak equivalence $u$ between $B$ and $A[f]$ over~$X$. The completion of this diagram claimed by \cref{Prop:Homotopy_ext_prop} is exactly what one needs to produce the required diagonal filler.

For part~(ii), we prove that $t \co \Weq(\pi_c) \to \U_c$ is a trivial fibration. For this, let us
observe that we have a diagram 
\[
\xymatrix@C=1.5cm{
\Weq(\U_c) \ar[r]^{\sigma}  \ar[d]_{(s,t)} & \Weq(\U) \ar[d]^{(s,t)} \\
\U_c \times \U_c \ar[r]^{\tau \times \tau} \ar[d]_{\pi_2} & \U \times \U \ar[d]^{\pi_2} \\
\U_c \ar[r]_\tau & \U }
\]
The composite  on the left-hand side is the map $t$ that we wish to show to be a trivial cofibration.
First, using part (i), observe that it is a fibration since it is the composite of two fibrations. Secondly,
observe that the top square is a pullback and so $\sigma$ is a trivial fibration since $\tau \times \tau$ 
is so. Thus, applying 3-for-2 to the outer square, we obtain that $t$ is a weak equivalence and hence
a trivial fibration, as required.
\end{proof} 













\bibliographystyle{alpha}
\bibliography{../Auxiliary/bibliography}




\end{document}

% Type theory rules

\section{Type-theoretic rules}



\subsection*{Identity types}


\[
\begin{gathered}
\begin{prooftree}
\Gamma \vdash A \co \type \qquad
\Gamma \vdash a \co A \qquad
\Gamma \vdash b \co A 
\justifies
\Gamma \vdash \mathsf{Id}_A(a,b) \co \type
\end{prooftree} \\[2ex]
\begin{prooftree}
a \co A 
\justifies
\mathsf{refl}(a) \co \mathsf{Id}_A(a,a)
\end{prooftree} \\[2ex]
\begin{prooftree}
\Gamma \vdash p \co \Id_A(a,b) \qquad
\Gamma, x, y \co A, z \co \Id_A(x,y) \vdash B \co \type \qquad
\Gamma, x \co A \vdash d \co B[x/y, \mathsf{refl}(x)/z] 
\justifies
\Gamma \vdash \mathsf{J}(a,b,p,d) \co B[a/x, b/y, p/z]
\end{prooftree} \\[2ex]
\begin{prooftree}
\Gamma \vdash a \co A \qquad
\Gamma, x, y \co A, z \co \Id_A(x,y) \vdash B \co \type \qquad
\Gamma, x \co A \vdash d \co B[x/y, \mathsf{refl}(x)/z] 
\justifies
\Gamma \vdash \mathsf{J}(a,a,\mathsf{refl}(a),d) = d[a/x] \co B[a/x, a/y, \mathsf{refl}(a)/z]
\end{prooftree}
\end{gathered}
\]



\subsection*{$\Sigma$-types}

\[
\begin{gathered}
\begin{prooftree}
\Gamma \vdash A \co \type \qquad
\Gamma, x \co A \vdash B \co \type
\justifies
\Gamma \vdash (\Sigma x \co A) B 
\end{prooftree}  \\[1ex]
\begin{prooftree}
\Gamma \vdash a \co A  \qquad
\Gamma \vdash b \co B[a/x] 
\justifies
\Gamma \vdash \mathsf{pair}(a,b) \co (\Sigma x \co A) B 
\end{prooftree}  \\[2ex]
\begin{prooftree}
\Gamma \vdash c \co (\Sigma x \co A) B 
\justifies
\Gamma \vdash  \pi_1(c) \co A 
\end{prooftree} \qquad
\begin{prooftree}
\Gamma \vdash c \co (\Sigma x \co A) B 
\justifies
\Gamma \vdash \pi_2(c) \co B[ \pi_1(c)/x] 
\end{prooftree} \\[2ex]
\begin{prooftree}
\Gamma \vdash a \co A  \qquad
\Gamma \vdash b \co B[a/x] 
\justifies
\Gamma \vdash \pi_1 \big( \mathsf{pair}(a,b) \big)  = a \co A
\end{prooftree}  \qquad
\begin{prooftree}
\Gamma \vdash a \co A  \qquad
\Gamma \vdash b \co B[a/x] 
\justifies
\Gamma \vdash \pi_2\big( \mathsf{pair}(a,b) \big) = b \co B[a/x]
\end{prooftree} \\[2ex]
\begin{prooftree}
\Gamma \vdash c \co (\Sigma x \co A) B 
\justifies
\Gamma \vdash  c = \mathsf{pair}(\pi_1(c), \pi_2(c))  \co   (\Sigma x \co A) B 
\end{prooftree}
\end{gathered}
\]







\subsection*{$\Pi$-types}
 
\[ 
\begin{gathered} 
\begin{prooftree}
\Gamma \vdash A \co \type \qquad
\Gamma, x \co A \vdash B \co \type
\justifies
\Gamma \vdash (\Pi x \co A) B 
\end{prooftree}  \\[2ex]
\begin{prooftree}
\Gamma, x \co A \vdash b \co B
\justifies
\Gamma \vdash (\lambda x \co A) b \co (\Pi x \co A) B 
\end{prooftree} \\[2ex]
\begin{prooftree}
\Gamma \vdash f \co (\Pi x \co A) B \quad
\Gamma \vdash a \co A 
\justifies
\Gamma \vdash \mathsf{app}(f,a) \co B[a/x]
\end{prooftree} \\[2ex]
\begin{prooftree}
\Gamma, x \co A \vdash b \co B \qquad
\Gamma \vdash a \co A 
\justifies
\Gamma \vdash \mathsf{app}( (\lambda x \co A)b, a) = b[a/x] \co B[a/x] 
\end{prooftree} \\[2ex]
\begin{prooftree}
\Gamma \vdash f \co (\Pi x \co A) B 
\justifies
\Gamma \vdash \eta_f \co \Id_{(\Pi x \co A)B} \big( f, (\lambda x \co A) \mathsf{app}(f,x)\big)
\end{prooftree}
\end{gathered}
\]






\subsection*{Universe type}


\[
\U \co \type \qquad 
\begin{prooftree}
\Gamma \vdash a \co \U
\justifies
\Gamma \vdash \mathsf{T}(a) \co \type
\end{prooftree}
\]

\appendix


