
\subsection*{Second approach (\`a la Shulman)}


\notesh{[Work in progress] I prefer to start again this section from scratch, but I'm leaving the old univalence section as long as this is not finished}

We recall what it means for a fibration $p \co A \to X$  to be a univalent fibration. Using the $\Pi$-types that we have already defined, one can to construct a fibration $(s, t) \co A^{\to} \to X \times X$ that classifies morphisms between two fibers of $p$. More precisely, $A^{\to} \to X \times X$ is the result of applying
the functor
\[
\Pi_{p \times 1_X} \co \SSet_{/A \times X} \to \SSet_{X \times X}
\]
to $1_A \times X \co A \times A \to A \times X$. 
A morphisms $Y \to A^{\to}$ in $\SSet_{/ X \times X}$ is the same thing as a morphism $(x_1, x_2) \co Y \to  X \times X$ together with a morphism $A[x_1] \rightarrow A[x_2]$ over $Y$. We now need to construct an object $\Weq(p) \to E^{\to}$ that similarly classifies weak equivalences $w \co E[f] \to E[g]$. \caution{We have not been able to show constructively that weak equivalences are stable under pullback along fibrations, so it is not clear if such an object $\Weq(p)$ can be defined as a subobject od $E^{\to}$ or not}. \noten{Hasn't right properness been proved now? If so, maybe some of what is below is not needed.}
Instead we will follow Voevodsky's definition from type theory, see also section $4$ of \cite{shulman:reedy}, and hence one needs to start by discusing the type $\iscontr_B(E)$



Given a fibration $p \co A \to X$ we define
\[ 
\iscontr_X(A) \defeq \Sigma_p \Pi_{\pi_2} \Id_A 
\]
where $\Id_A \to A \times_X A$ and $\pi_2 \co A \times_X A \rightarrow A$.
Thus, $\iscontr_X(A) \to A$ is a fibration  which encodes the idea that the fibers of $p$ are contractible, in the sense that the data of an element of $\iscontr_X(A)$ over $x \in X$ is an element $a_0 \in A(x)$ and
a function mapping $a \in A(x)$ to a path from~$a_0$ to $a$.

We now need to use this to construct an object that classifies weak equivalences. 
Let fibrations $p \co A \to X$ and $q \co B \to X$ be fibrations and $f \co A \to B$ be
a map between them, 
\[
\xymatrix{
A \ar[dr]_{p} \ar[rr]^f & & B \ar[dl]^{q} \\
& X & 
}
\]
We can then form the homotopy fiber of $f$ as the pullback:
\[
\xymatrix@C=1.5cm{
\hfiber(f) \ar@{->>}[d] \ar[r]  & \Id_{B} \ar@{->>}[d] \\
A \times_X B \ar[r]_{f \times_X 1_B} & B \times_X B 
}
\]
which  produces a factorization of $f$,
\[
A \to \hfiber(f) \to B  \, .
\]

\begin{lemma}
If $A$ and $B$ are cofibrant, then $A \rightarrow \hfiber(f)$ is a weak equivalence.
\end{lemma}

\begin{proof}
The projections $\Id_{B} \to B$ are trivial fibrations because of part~(iv) of \cref{proposition:MainPathObject}, as the proof of this claim uses  only that $p \co A \to X$ is a fibration, and none of the other assumptions), so the pullback $\hfiber(f) \rightarrow B$ is also a trivial fibration. The simplicial set $\Id_{B}$ is cofibrant because  of part~(i) of \cref{proposition:MainPathObject}, which only uses that $B$ is cofibrant, so $\hfiber(f)$ is cofibrant by \cref{lem:cofibrant_fiber_product} as $A$ and $B$ are cofibrant. It follows that $\hfiber(f) \rightarrow B$ is a weak equivalence. By 3-for-2 for weak equivalences, $A \rightarrow \hfiber(f)$ is a weak equivalence.
\end{proof}




For any morphisms $f \co A \to B$ in $\Fib/X$, we define
\[ 
\isequiv_X(f) = \Pi_{q}  \, \iscontr_X( \hfiber(f)) 
\]
where $q \co B \to X$. The definition of $\isequiv$ formalizes the idea that $\eta$ is an equivalence if and only all its homotopy fibers are contractible. Given $p \co A \to X$ be a fibration, one defines
\[ 
\Weq(p) = \isequiv_{A^{\to}}(\eta) 
\]
where $\eta$ denotes the universal map over $A^{\to}$ corresponding to its universal property. 

\medskip

We now prove  that these definitions indeed behave as expected. These lemma mostly comes from  \cite[Section 4]{shulman:reedy}, but we need to reprove them to track of the  ecessary fibrancy and cofibrancy assumptions.

\begin{lemma}\label{lem:IsContr1}For a fibration $p \co A \to X$ with $A$ and $X$ bifibrant, the following are equivalent:

\begin{enumerate}[(i)]
\item The map $p \co A \to X$ is a trivial fibration.
\item The map $\iscontr_X (A) \to X$ has a section.
\item There is a map $ 1 \rightarrow \Pi_X  \iscontr_X (A) $.
\item The map $\iscontr_X (A) \to X$ is a trivial fibration.
\end{enumerate}
\end{lemma}



\begin{proof} A section of $\iscontr_X (A) \to X$ is  the data of a section $s \co X \rightarrow A$, and a map $h \co A \to \Id_A$ whose image by $\Id_A \to A \times_X A$ is $(s \circ p, 1_A)$. Such a section $s$ and homotopy $h$ produces a homotopy inverse to $p$ and hence implies that $p$ is a weak equivalences, and hence a trivial fibration as $X$ is assumed fibrant.  This proves (ii) $\Rightarrow$ (i).
Note that if we only assumes that $A$ and $X$ are cofibrant and $p$ is a fibration, it is still the case that $\Id_A \to A$ is a trivial fibration between cofibrant objects and hence a weak equivalence. This is then sufficient to show that the map $h$ induces an equality in the homotopy category between $s \circ p $ and the identity of $A$.


Conversely, assuming (i), one can construct a section $s$ as above using the lifting property of $X$ (cofibrant) against $p$. To construct the homotopy $h$ one needs to use a lifting property of $A$ against the map $Id_A \to A \times_X A$, which is a trivial fibration because $\Id_A \to A$ and $A \times_X A \to A$ both are, and trivial fibrations satisfy 3-for-2 among fibrations.

The equivalence between (ii) and (iii) follows from the universal property of $\Pi$-types.
The implication (vi) $\Rightarrow$ (ii) is clear when $B$ is cofibrant. We conclude by showing (i) $\Rightarrow$ (vi).

As above, as $X$ is fibrant, one can deduce assuming (i) that $\Id_A \to A \times_X A$ is a trivial fibration. $\pi_2 \co A \times_X A \to A$ is a fibration with cofibrant domain, hence $\Pi_{\pi_2} \Id_A \rightarrow A$ is a trivial fibration by \cref{cor:Pi_types_are_fibrant}, and finally, $\iscontr(A)_X \rightarrow X$ is just the composite $\Pi_{\pi_2} \Id_A \rightarrow A \rightarrow X$ and so it is  a trivial fibration.
\end{proof}

\begin{remark} \label{lem:IsContr1-sharp}
 \cref{lem:IsContr1} can be stated more sharply observing that (ii) implies that $p$ is a weak equivalence assuming either that $X$ is fibrant or that $A$ and $X$ are cofibrant, so (ii) $\Rightarrow$ (i) when $X$ is fibrant, (i) $\Rightarrow$ (ii) when $X$ is cofibrant and $A$ is bifibrant.
(ii) $\Leftrightarrow$ (iii) holds without assumption on $A$ and $X$, (vi) $\Rightarrow$ (ii) holds as soon as $X$ is cofibrant. And finally (i) $\Rightarrow$ (vi) holds when $A$ is bifibrant. 
\end{remark} 

\begin{lemma} \label{lem:isContr2}
Let $p \co A \to X$ be a fibration, and $f \co Y \rightarrow X$ be a morphism. Then there is a natural isomorphisms
\[ 
 \iscontr_X(A) [f]  \cong \iscontr_Y(A[f])  \, .
\]
In particular, if $A$ is cofibrant and $Y$ bifibrant, then the following conditions are equivalent: 
\begin{enumerate}[(i)] 
\item $A[f] \to Y$ is a trivial fibration.
\item The map $f \co Y \rightarrow X$ factors via $\iscontr_X(A)$.
\end{enumerate}
\end{lemma}

\begin{proof} The stability of $\iscontr$ under pullback is a formal property which follows from the similar stability property of $\Pi$-types. The equivalence is an immediate consequence of this pullback stability and \cref{lem:IsContr1}.
\end{proof}


Similarly, one also has the following.

\begin{lemma} \label{isEquiv1} Let $f \co A \rightarrow B$ be a map in $\BFFib/X$ for $X$ bifibrant, i.e. $p \co A \to X$ and $q \co B \to X$ are fibrations and $A$ and $B$ are cofibrant. The following are equivalent:
\begin{enumerate}[(i)]
\item $f$ is a weak equivalence.
\item $\isequiv(f) \to X$ has a section.
\item There is a map $1 \rightarrow \Pi_X \isequiv(f)$.
\item The map $\isequiv(f) \to X$ is a trivial fibration.
\end{enumerate}
\end{lemma}



\begin{proof} As $B$ is cofibrant, $\Id_{B}$ is cofibrant as well because of part~(\ref{proposition:MainPathObject:IdBifib}) of \cref{proposition:MainPathObject}. And \cref{lem:cofibrant_fiber_product} implies that as $A$ and $B$ are cofibrant, then $\hfiber(f)$ is cofibrant as well. So as $\isequiv(f)$ is defined as $\Pi_{p_2}\iscontr( \hfiber(f))$ the equivalence above mostly follows from \cref{lem:IsContr1}. More precisely, $f$ factors as $A \to \hfiber(f) \to B$ so $f$ is an equivalence if and only if $\hfiber(f) \to B$ is an equivalence, which by \cref{lem:IsContr1} is equivalent to the fact that $\iscontr( \hfiber(f)) \to B$ has a section, which by universal property of $\Pi$-types is equivalent to condition (ii). The equivalence between (ii) and (iii) is formal, exactly as in the proof of \cref{lem:IsContr1}. One still has that (iv) implies (ii) simply because $X$ is cofibrant. Assuming (i) one has that $\hfiber(f) \to B$ is a trivial fibration, hence $\iscontr( \hfiber(f)) \to E_2$ is a trivial fibration by \cref{lem:IsContr1}, and \cref{cor:Pi_types_are_fibrant} implies that $\Pi_{p_2} \iscontr( \hfiber(f)) = \isequiv(f) \to X$ is a trivial fibration.

As observed in~\cref{lem:IsContr1-sharp}, one has that a section of $\isequiv(f) \to X$ shows that $\hfiber(f) \to A$ is an equivalence only assuming that $A$ and $B$ are cofibrant, so it is indeed the case that (ii) $\Rightarrow$ (i) holds without assuming $X$ is bifibrant.
\end{proof}

Note that in~\cref{isEquiv1} the implication (ii) $\Rightarrow$ (i) holds without assuming that $X$ is bifibrant. Similarly to \cref{lem:isContr2}, one has the following.

\begin{lemma} \label{lem:isEquiv2}
Let $f \co A \rightarrow B$ be a map in $\BFFib/X$  and $\sigma \co Y \rightarrow X$ a morphism with $Y$ and $X$ bifibrant. Then there is a natural isomorphisms
\[ 
\isequiv_Y(f[\sigma]) \cong \iscontr_X(f) [\sigma]
\]
where $f[\sigma] \co A[\sigma] \rightarrow B[\sigma]$ is pullback of $f$ along $\sigma$. 
In particular, the following conditions are equivalent: 
\begin{enumerate}[(i)]
\item $f[\sigma] \co A[\sigma] \rightarrow B[\sigma]$ is an equivalence.
\item The maps $\sigma$ factors via $\isequiv_X(f)$.
\end{enumerate}
\end{lemma}





Let $p \co A \to X$ be a fibration with $A$ and $X$ bifibrant. By the universal property of $A^\to$, there
is a map $A^{\to} \to X \times X$ corresponding to the identity $A[1_X] \rightarrow A[1_X]$. This map
fits into a factorization of the diagonal map $X \rightarrow A^{\to} \to X \times X$. Since the identity
$A[1_X] \rightarrow A[1_X]$ is a weak equivalence, \cref{lem:isEquiv2} implies that this maps factors into $X \rightarrow \Weq(p) \to X \times X$. Then, using the lifting
\[
\xymatrix{
X \ar[d] \ar[r] & \Weq(p) \ar@{->>}[d] \\
\Id_X \ar@{->>}[r] \ar@{..>}[ur] & X \times X \\ 
}
\]
one obtains a map $j \co \Id_X \rightarrow \Weq(p)$.

\begin{definition} \label{def:univalentFib}
A fibration $p \co A \to X$  is said to be \emph{univalent} if the map $j \co \Id_X \rightarrow \Weq(p)$ is an equivalence.
\end{definition}

This is clearly equivalent to the fact that the map $X \rightarrow \Weq(p)$ is an equivalence, or to the fact that any of the two projection $\Weq(p) \to X$ is a trivial fibration.


\bigskip

\notesh{The next proposition assumes $\UU_c \to \U_c$ is a fibration between bifibrant objects.}

\begin{proposition}
The fibration $\pi_c \co \UU_c \to \U_c$ is univalent.
\end{proposition}



\begin{proof} It is enough to show that the target projection $\Weq(\UU_c) \to \U_c$ is a weak equivalence. Consider a diagram:
\[ 
\xymatrix{
Y \ar[d]_{f} \ar[r]^-{\beta} & \Weq(\UU_c) \ar[d]^{\pi_c} \\
X \ar[r]_-{\alpha} & \U_c
}
\]
where $f$ is any cofibration between $Y$ and $X$ two bifibrant objects. We will show that there exists a diagonal filling making the diagram commutes up to homotopies. This is sufficient to show $\Weq(\UU_c) \rightarrow \U_c$ is invertible in the homotopy category, hence is a weak equivalence.

The map to $\alpha$ corresponds to a fibration $A \to X$ with $A$ cofibrant, and the map $\beta$ gives in particular a (solid) diagram as in \cref{Prop:Homotopy_ext_prop}:
\[ 
\xymatrix{
 B
  \ar@{.>}[rr]
  \ar[dr]^{u}
  \ar[dd]_(.3){q}
&&
  \bar{B}
  \ar@{.>}[dr]^{v}
  \ar@{.>}[dd]_(.3){\bar{q}}|{\hole}
&\\&
  A[f] 
  \ar[rr]
  \ar[dl]
&&
  A
  \ar[dl]^{p}
\\
  Y
  \ar[rr]_{f}
&&
  X
&
}
\]
which by \cref{Prop:Homotopy_ext_prop} admits a dotted extention where both square are pullbacks.

This filling in turn allows us to construct a map $X \rightarrow \Weq(\UU_c)$ that makes the lower triangle commutes. The upper triangle does not quite commute though, one only have that the composite with $\Weq(\UU_c) \to \UU^{\to}_c$ commutes ``up to isomorphisms'' in the sense that the two maps $Y \rightrightarrows \UU^{\to}_c$ classifies arrows in $\BFFib_{/Y}$ that are isomorphic.

Hence the two maps $Y \rightrightarrows \UU^{\to}_c$ are homotopic.

\notesh{This is probably not very hard to prove, but I'm not sure what is the best way to do that... In the worst case scenario one can construct a homotopy explicitely, though that sounds tedious. This has probably been delt with in previous work on the topics ? }


Now, for $Y$ a bifibrant object two maps $Y \rightrightarrows \Weq(\UU_c)$ whose image in $\UU^{\to}_c$ are homotopic are also homotopic which concludes the proof that one has a diagonal map making the two triangles commutes up to homotopy. To prove the last claim consider a bifibrant interval object $Y + Y \hookrightarrow IY \overset{\sim}{\to} Y$. Given a diagram of the form 
\[
\xymatrix{
Y + Y \ar[d] \ar[r] & \Weq(\pi_c) \ar@{->>}[d] \\
I Y \ar[r] & \UU_c^{\to} \\  }
\]
One needs to construct a diagonal lift. One forms the successible pullback $P_1$ and $P_2$:
\[
\xymatrix{P_2 \ar[r] \ar@{->>}[d] & P_1 \ar@{->>}[d] \ar[r] & \Weq(\pi_c) \ar@{->>}[d] \\
Y \ar[r]^{\sim} & I Y \ar[r] & \UU_c^{\to} \\ }
\]
The map $P_2 \rightarrow Y$ is a trivial fibration, because it is of the form $\isequiv_Y( \_) \to Y$ and it has a section. The map $P_1 \rightarrow P_2$ is a weak equivalence as a pullback of a weak equivalence between fibrant objects along a fibrations. This implies that the map $P_1 \to IY$ is a trivial fibration, and this is sufficient to construct a diagonal lift in the initial diagram.
\end{proof}
