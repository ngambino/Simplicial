\documentclass[11pt, oneside]{article}   	% use "amsart" instead of "article" for AMSLaTeX format
\usepackage{geometry}                		% See geometry.pdf to learn the layout options. There are lots.
\geometry{a4paper}                   		% ... or a4paper or a5paper or ... 
%\geometry{landscape}                		% Activate for rotated page geometry
\usepackage[parfill]{parskip}    		% Activate to begin paragraphs with an empty line rather than an indent
\usepackage{graphicx}				% Use pdf, png, jpg, or eps§ with pdflatex; use eps in DVI mode
								% TeX will automatically convert eps --> pdf in pdflatex		
\usepackage{amssymb,amsmath}
\usepackage[all,cmtip]{xypic}

\newcommand{\Nat}{\mathbb{N}}
\newcommand{\defeq}{=_\text{def}}
\newcommand{\co}{\colon}
\newcommand{\SSet}{\mathbf{SSet}}
\newcommand{\Set}{\mathbf{Set}}
\newcommand{\boundary}{\partial}

%SetFonts

%SetFonts


						% Activate to display a given date or no date

\begin{document}

\begin{flushright}
October 3rd, 2018
\end{flushright}

Dear Simon,

\bigskip

I have been looking at how to instantiate the small object argument in simplicial sets to generate a weak
factorisation system of `cofibrations' and `trivial fibrations', following your paper.

I have a few questions.

We start from the set of `generating cofibrations'
\[
I \defeq \{ \partial \Delta[n] \to \Delta[n] \ | \ n \in \Nat \}
\]
We want to show that $(\overline{I}, I^\pitchfork)$ is a weak factorisation system, where $\overline{I}$ is
the saturation of $I$ and $I^\pitchfork$ is the class of maps that have a choice of lifts against all maps
in $I$. 

As usual, the key step in the proof is the construction of the factorisation of a map. In your paper, you suggest that this situation falls under the `good case' of the small object argument 
and that one can even modify it to make it equivalent to Garner's version (see pp.~52-53). 

I'd like to understand concretely how things work in this case.

To fix notation, let us consider a fixed map $f \co X \to Y$ which we want to factor. I will write $X^{(n)}$, for $n \in \Nat$, for the objects generated by the small object argument, fitting in multiple pushout diagrams of the form
\[
\xymatrix{
\bigsqcup \partial \Delta[n] \ar[r] \ar[d]  & X^{(n)} \ar[d] \\
\bigsqcup \Delta[n] \ar[r]. & X^{(n+1)} }
\]
The required factorisation will be given as 
\[
 \xymatrix{
  X \ar[rr]^f \ar[dr] & & Y \\
   & X_\omega \ar[ur] & }
   \]
where $X_\omega = \lim_{n} X^{(n)}$. 

To apply the `good case' of the small object argument and to be able to modify it \`a la Garner, it seems that we need to know the following three facts. 
\begin{enumerate}
\item The maps $X_n \to X_{n+1}$ are complemented monomorphisms. 
\item The domains of the maps in $I$, i.e.~the simplicial sets $\boundary \Delta[n]$ are `small' in the
sense that  the functor
\[
\mathrm{Hom}( \boundary \Delta[n], -) \co \SSet \to \Set
\] 
preserves colimits of increasing $\Nat$-chains.
\item The functions
\begin{equation}
\tag{$\ast$}
\mathrm{Hom}( \boundary \Delta[n], X^{(n)}) \to \mathrm{Hom}(\boundary \Delta[n], X^{(n+1)}) \, , 
\end{equation}
induced by the maps $X^{(n)} \to X^{(n+1)}$, are complemented monomorphisms, i.e.~we have  decidable subsets.
\end{enumerate}

I think (1) and (2) are not difficult, but I have not checked them yet.

Assuming (1) and (2) hold, we obtain that the maps in  $(\ast)$ are injective. So for (3) we only 
need to show that they
are complemented. For this you write in the paper that it is enough that 
the monomorphisms $X_n \to X_{n+1}$ are complemented and that the objects $\partial \Delta[n]$ are 
``finitely generated''.  Could you please explain this a little further? 

I guess the general question is whether, given a complemented monomorphism $A \rightarrowtail B$
in $\SSet$, the induced injective function 
\[
\mathrm{Hom}( \boundary \Delta[n], A) \rightarrowtail \mathrm{Hom}(\boundary \Delta[n], B) \, , 
\]
in $\Set$ is complemented. This amounts to be able to decide whether a given map~$f \co \partial \Delta[n]
\to B$ factors via $A \rightarrowtail B$, as in the diagram
\[
\xymatrix{
& A \ar@{>->}[d]\\
\partial \Delta[n]  \ar[r]_-f \ar@{.>}[ur] & B }
\]
This essentially amounts to checking that all the values of $f$ lie in $A$. I can see that~$\partial \Delta[n]$ being ``finitely generated'' can help to make this decidable, but I am not sure about what you had in
mind exactly and I'd like to see the details to convince myself that everything is constructive. Could you please help me with this? 


\bigskip

Best wishes,

Nicola




\end{document}  